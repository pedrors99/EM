\documentclass[11pt,a4paper]{article}

% Packages
\usepackage[utf8]{inputenc}
\usepackage[spanish, es-tabla]{babel}
\usepackage{parskip}
\usepackage{enumerate}
\usepackage{graphicx}
\usepackage{subfigure}
\usepackage{float}
\usepackage{amsmath}
\usepackage{amssymb}
\usepackage{upgreek}
\usepackage{enumitem}
\usepackage{xfrac}
\usepackage{lmodern}

\usepackage[bookmarks=true,
					 bookmarksnumbered=false,
					 bookmarksopen=false,
					 colorlinks=true,
					 allcolors=blue,
					 urlcolor=blue]{hyperref}

\usepackage[ruled]{algorithm2e}
\SetKwInOut{Parameter}{parameter}

\usepackage{array}
\newcolumntype{N}{>{\centering\arraybackslash}m{0.5cm}}
\newcolumntype{M}{>{\centering\arraybackslash}m{1cm}}

\usepackage[left=2cm, right=2cm, top=2cm, bottom=2cm]{geometry}

\begin{document}\pagenumbering{arabic}
\begin{titlepage}
\centering
\includegraphics[width=0.15\textwidth]{/Users/pedrors/Desktop/DGIIM/Latex/UGR}\par\vspace{1cm}
{\scshape\LARGE Universidad de Granada \par}
\vspace{1cm}
\vspace{1.5cm}
{\huge\bfseries Estadística Multivariante\par}
\vspace{2cm}
{\Large\itshape Pedro Ramos Suárez\par}
\vfill
Doble Grado de Ingeniería Informática y Matemáticas
\vfill
{\large \today\par}
\end{titlepage}

\tableofcontents
\newpage

\section{Distribución Normal Multivariante}

\subsection{Motivación sobre la DNM}

\subsubsection{General}
\begin{itemize}
\item La DNM es una de las distribuciones más conocidas en la estadística, jugando un papel predominante en muchas áreas de aplicación.

\item En el Análisis Multivariante, por ejemplo, la mayor parte de los procedimientos de inferencia para el análisis de datos vector-valuados se han desarrollado bajo la suposición de normalidad.
\end{itemize}

\subsubsection{Aspectos específicos}
\begin{itemize}
\item La DNM constituye la extensión natural, al contexto multivariante, de la distribución normal univariante. En particular, las distribuciones marginales univariantes de la DNM son normales.

Más aún, las distribuciones marginales de cualquier dimensión de la DNM son también DNMs.

\item Muchos procedimientos basados en la suposición de normalidad (multivariante, en general), particularmente procedimientos basados en los momentos de primer y segundo orden, cumplen ciertas propiedades de optimalidad.

En determinados casos, los procedimientos presentan cierta robustez frente a desviaciones de la normalidad en la distribución teórica subyacente o en los propios datos. Una justificación viene dada, por ejemplo, por el TCL acerca de la DNM asintótica del vector de medias muestrales, de modo que, para grandes muestras, la DNM puede usarse como aproximación.

En otros casos, la DNM sirve como modelo de referencia para evaluar posibles desviaciones de la normalidad.

\item Técnicamente, la DNM ha sido estudiada en profundidad y ofrece ciertas ventajas en su tratamiento analítico, también desde el punto de vista estadístico-probabilístico:
\begin{itemize}
\item La DNM viene determinada de forma única por los momentos de primer y segundo orden ($\mu$, $\Sigma$).

\item Si dos subvectores de un vector aleatorio con DNM tienen correlaciones curzadas nulas, entonces dichos subvectores son mutuamente independientes.

\item La familia de DNMs es cerrada bajo transformaciones lineales (...).

\item La familia de DNMs es cerrada bajo combinaciones lineales de vectores independientes (...).

\item Las distribuciones condicionadas (internamente) en una DNM sin también DNMs. Sus momentos de primer y segundo orden se obtienen de forma sencilla a partir de los correspondientes a la DNM de referencia.

\item ...
\end{itemize}
\end{itemize}

\subsubsection{Algunas notas históricas}
\begin{itemize}
\item La DN bivariante comenzó a estudiarse a mediados del s. XIX. Tuvo un gran impacto a partir de una publicación de Francias Galton (1988), sobre análisis de correlación en genética.

\item Karl Pearson (1892), a partir del trabajo de Galton, estableció su formulación matemática de forma rigurosa.

\item Francis Edgeworth (1892) introdujo, en el estudio de la regresión múltiple y el análisis de correlación parcial y múltiple, elementos fundamentales para el desarrollo de la DNM.

\item Durante el s. XX se sucedieron importantes avances en relación con la fundamentación teórica de la DNM, la inferencia estadística con referencia a la DNM (incluyendo el desarrollo y estudio de otras distribuciones relacionadas, e.g. $T^{2}$ de Hotelling, Wishart), así como su contextualización en familias de distribuciones más generales (e.g. distribuciones simétricas de contornos elípticos).
\end{itemize}

\subsection{Fundamentación probablilística (recordatorio)}

\subsubsection{Elementos}
\begin{itemize}
\item $(\Omega, \mathcal{F}, P)$ espacio de probabilidad.
\begin{itemize}
\item $\Omega$: Espacio muestral.
\item $\mathcal{F}$: $\sigma$-álgebra.
\item $P$: Medida de probabilidad.
\end{itemize}

\item $X = (X_{1}, \dots, X_{p})'$, vector aleatorio; i.e., aplicación medible
$$\begin{aligned}
X: & & (\Omega, \mathcal{F}) & \to & & (\mathbb{R}^{p}, \mathcal{B}^{p}) \\
& & w & \mapsto & & X(w) = (X_{1}(w), \dots, X_{p}(w))',
\end{aligned}$$
$$\forall B \in \mathcal{B}^{p}, X^{-1}(B) := \{w \in \Omega: X(w) \in B\} \in \mathcal{F}$$

\item $P_{X}$, distribución de probabilidad inducida por $X$ sobre ($\mathbb{R}^{p}, \mathcal{B}^{p}$):
$$P_{X}[B] := P[X^{-1}(B)], \text{ para cada } B \in \mathcal{B}^{p}$$

\item Nota: $X = (X_{1}, \dots, X_{p})' = (X_{1}, \dots, X_{p})^{t} = \begin{pmatrix} X_{1} \\ \vdots \\ X_{p} \end{pmatrix}$.
\end{itemize}

\subsubsection{Función de distribución}
\begin{itemize}
\item Se define la función de distribución asociada a $P_{X}$ como
$$F_{X}(x) = P_{X}[X_{1} \leq x_{1}, \dots, X_{p} \leq x_{p}], \forall (x_{1}, \dots, x_{p}) \in \mathbb{R}^{p}$$

\item Si existe $f_{X}$ integrable (en el sentido de Lebesgue) tal que
$$F_{X}(x) = \int_{-\infty}^{x_{p}} \dots \int_{-\infty}^{x_{1}} f_{X}(u_{1}, \dots, u_{p}) du_{1} \dots du_{p}, \forall x \in \mathbb{R}^{p}$$
$f_{X}$ se denomina función de distribución, estando definida salvo conjuntos de medida de Lebesgue nula. (Se dice entonces que la distribución de $X$ es ``continua'').

Si $f_{X}$ es una función continua, se puede escribir:
$$f_{X}(x) = \frac{\delta^{n}}{\delta x_{1} \dots \delta x_{p}} F_{X}(x), \forall x \in \mathbb{R}^{p}$$
(en otro caso, se podrá escribir la expresión únicamente para los puntos de continuidad de $f_{X}$) y $f_{X}$ estará definida de forma unívoca bajo esa condición.
\end{itemize}

\subsubsection{Independencia}
\begin{itemize}
\item En general, para un conjunto producto de la forma $B = B_{1} \times \dots \times B_{p}$, con $B_{j} \in \mathcal{B} (j = 1, \dots, p)$, se tendrá que
$$P_{X}[B] \neq P_{X_{1}}[B_{1}] \cdot ... \cdot P_{X_{p}}[B_{p}]$$

\item Cuando se da la igualdad para todo conjunto producto $B = B_{1} \times \dots \times B_{p} \in \mathcal{B}$ se dice que las variables $X_{1}, \dots, X_{p}$, componentes del vector aleatorio $X$, son independientes entre sí (o mutuamente independientes).

\item Equivalentemente, esto ocurre cuando
$$F_{X}(x) = F_{X_{1}}(x_{1}) \cdot ... \cdot F_{X_{p}}(x_{p}), \forall x = (x_{1}, \dots, x_{p})' \in \mathbb{R}^{p}$$
y en el caso continuo, si y sólo si
$$f_{X}(x) = f_{X_{1}}(x_{1}) \cdot ... \cdot f_{X_{p}}(x_{p}), \forall x = (x_{1}, \dots, x_{p})' \in \mathbb{R}^{p}$$

\item De forma más general, se habla de independencia entre cualquier número finito de vectores aleatorios (por ejemplo, entre subvectores de un vector aleatorio dado), bajo condiciones de factorización similares en términos de los factores correspondientes a éstos.

\item $X = \begin{pmatrix} X_{(1)} \\ X_{(2)} \end{pmatrix}$ \hspace{2cm} $X_{(1)} = \begin{pmatrix} X_{1} \\ \vdots \\ X_{k} \end{pmatrix}$ \hspace{1cm} $X_{(2)} = \begin{pmatrix} X_{k+1} \\ \vdots \\ X_{p} \end{pmatrix}$.

Si se cumple que
$$f_{X}(x) = f_{X_{(1)}}(x_{(1)}) \cdot f_{X_{(2)}}(x_{(2)}), \forall x \in \mathbb{R}^{p}$$
$$x = \begin{pmatrix} x_{(1)} \\ x_{(2)} \end{pmatrix} \iff X_{(1)} \text{ y } X_{(2)} \text{ son independientes}$$

\item Si $X, Y$ son independientes, entonces $g(X), h(Y)$ son independientes para aplicaciones medibles $g$ y $h$.
$$\begin{aligned}
g:(\mathbb{R}^{p}, \mathcal{B}^{p}) \to (\mathbb{R}^{p'}, \mathcal{B}^{p'}) & & h:(\mathbb{R}^{q}, \mathcal{B}^{q}) \to (\mathbb{R}^{q'}, \mathcal{B}^{q'}) \\
\underset{\underset{g(X)}{\longrightarrow}}{(\Omega, F) \overset{X}{\to} (\mathbb{R}^{p}, \mathcal{B}^{p}) \overset{g}{\to} (\mathbb{R}^{p'}, \mathcal{B}^{p'})} & & \underset{\underset{h(Y)}{\longrightarrow}}{(\Omega, F) \overset{Y}{\to} (\mathbb{R}^{q}, \mathcal{B}^{q}) \overset{h}{\to} (\mathbb{R}^{q'}, \mathcal{B}^{q'})}
\end{aligned}$$
\end{itemize}

\subsubsection{Función característica}
\begin{itemize}
\item Dado un vector aleatorio $X = (X_{1}, \dots, X_{p})'$, se define su función característica como
$$\phi_{X}(t) = E[e^{it'X}] = \int_{\Omega} e^{it'X(w)}P(dw) = \int_{\mathbb{R}^{p}} e^{it'x}P_{X}(dx), \forall t = (t_{1}, \dots, t_{p})' \in \mathbb{R}^{p}$$

\item Recordatorio: $e^{ia} = \cos a + i \sin a$. \hspace{2cm} $\mathbb{C} \equiv \mathbb{R}^{2}$.

\item Teorema (existencia y unicidad): La función característica de un vector aleatorio siempre existe, y determina de forma única su distribución.

\item La f.c. tiene gran utilidad como herramienta en la resolución de aspectos analíticos sobre distribuciones de vectores aleatorios.
\end{itemize}

\subsubsection{Función característica e independencia}
Dos resultados importantes:
\begin{itemize}
\item Resultado 1: Las componentes del vector aleatorio $X = (X_{1}, \dots, X_{p})'$ son independientes si y sólo si su función característica se factoriza de la forma
$$\phi_{X}(t) = E[e^{it'X}] = \prod_{k=1}^{p} E[e^{it_{k}X_{k}}] = \phi_{X_{1}}(t_{1}) \cdot ... \cdot \phi_{X_{p}}(t_{p})$$
$$\forall t = (t_{1}, \dots, t_{p})' \in \mathbb{R}^{p}$$
(i.e., como el producto de las funciones características de las distribuciones univariantes marginales).

\item Resultado 2: Si las componentes del vector aleatorio $X = (X_{1}, \dots, X_{p})'$ son independientes, entonces la función característica de la variable aleatoria
$$Y = \sum_{k=1}^{p} X_{k}$$
se factoriza como
$$\phi_{Y}(t) = E[e^{itY}] = \prod_{k=1}^{p} E[e^{itX_{k}}] = \phi_{X_{1}}(t) \cdot ... \cdot \phi_{X_{p}}(t), \forall t \in \mathbb{R}$$
\end{itemize}

\subsubsection{Función característica y transformaciones lineales}
\begin{itemize}
\item Sea $X = (X_{1}, \dots, X_{p})'$ un vector $p$-dimensional, y sea $Y = (Y_{1}, \dots, Y_{q})'$ un vector $q$-dimensional definido como
$$Y = BX + b$$
con $B$ matriz $q \times p$ (cte) y $b$ vector $q \times 1$ (cte).

Entonces, la f.c. de $Y$ se obtiene, para cada $t = (t_{1}, \dots, t_{q})' \in \mathbb{R}^{q}$, como
$$\phi_{Y}(t) = e^{it'b} \phi_{X}(B't) \hspace{1cm} (1)$$
En particular, si el vector aleatorio $X$ se particiona en dos subvectores como $X = \begin{pmatrix} X_{(1)} \\ X_{(2)} \end{pmatrix}$, con $X_{(1)}$ de dimensión $k \times 1$ y $X_{2}$ de dimensión $(p-k) \times 1$, se tiene que
$$\phi_{X_{(1)}}(t_{(1)}) = \phi_{X} \begin{pmatrix} t_{(1)} \\ 0_{(2)} \end{pmatrix} \hspace{1cm} (2)$$
con $t_{(1)} = (t_{1}, \dots, t_{k})' \in \mathbb{R}^{k}$ y $0_{(2)} = (0, \dots, 0)' \in \mathbb{R}^{p-k}$.

\item (1) $$\phi_{Y}(t) E[e^{it'Y}] = E[e^{it'(BX+b)}] =$$
$$E[e^{it'b} + e^{it'Bx}] = e^{it'b}E[e^{i'Bx}] = e^{it'b} \phi_{X}(Bt)$$

\item (2) $$B = (I_{k}, 0_{p-k}) \hspace{2cm} b = 0_{k \times 1}$$
$$X_{(1)} = (I_{k}, 0_{p-k})X + 0_{k \times 1}$$
$$X = \begin{pmatrix} X_{(1)} \\ X_{(2)} \end{pmatrix}$$
Aplicamos $(1)$.
$$\phi_{X_{(1)}} = e^{it_{(1)}'b} \phi_{X} (\begin{pmatrix} I_{k} \\ 0_{p-k} \end{pmatrix} t_{(1)}) \overset{t_{(1)}'b=0}{=} \phi_{X} \begin{pmatrix} t_{(1)} \\ 0_{p-k} \end{pmatrix} = \phi_{X} \begin{pmatrix} t_{(1)} \\ 0_{(2)} \end{pmatrix}$$
\end{itemize}

\subsubsection{Función característica y función de densidad (caso continuo)}
\begin{itemize}
\item La función de densidad de un vector aleatorio $X = (X_{1}, \dots, X_{p})'$ con distribución continua se puede obtener como la transformada de Fourier de la función característica:
$$f_{X}(x) = \frac{1}{(2\pi)^{p}} \int_{\mathbb{R}^{p}} e^{-it'x} \phi_{X}(t) dt$$
Recordemos que, en este caso, la función característica también se escribe como
$$\phi_{X}(t) = \int_{\mathbb{R}^{p}} e^{it'x}f_{X}(x) dx$$
(Es decir, $f_{X}(\cdot)$ y $\phi_{X}(\cdot)$ constituyen un ``par de transformadas de Fourier'')
\end{itemize}

\subsubsection{Observaciones finales}
\begin{itemize}
\item Desde el punto de vista puramente estadístico, especialmente en el contexto de la Inferencia Estadística, interesa acerca de un vector aleatorio, $X$, el conocimiento sobre su distribución, $P_{X}$, puesto que los sucesos de interés vendrán expresados en términos de condiciones relativas a los valores que toma $X$.

\item Son importantes, en particular, las familias de distribuciones multivariantes (modelos) que, de forma flexible, puedan dar una representación exacta o aproximada de la variabilidad o el comportamiento de poblaciones o fenómenos objeto de investigaciones aplicadas.

\item En este contexto, la distribución normal multivariante (DNM) constituye un modelo central en el desarrollo y aplicación de la Estadística Multivariante.
\end{itemize}

\subsection{Aspectos generales sobre vectores aleatorios}

\subsubsection{Esperanza y covarianza (``operadores'')}
\begin{itemize}
\item Sea $X = (X_{1}, \dots, X_{p})'$ un vector aleatorio.

\item Se define el vector de medias de $X$ como
$$\mu_{X} := E[X] := \begin{pmatrix}
E[X_{1}] \\
\vdots \\
E[X_{p}]
\end{pmatrix} = \begin{pmatrix}
\mu_{1} \\
\vdots \\
\mu_{p}
\end{pmatrix}$$
(siempre que existan las esperanzas unidimensionales)

\item Propiedad de linealidad: Sea $Y = (Y_{1}, \dots, Y_{1})'$ definido como
$$Y = BX + b$$
con $B$ matriz $q \times p$ (cte) y $b$ vector $q \times 1$ (cte).

Entonces,
$$\mu_{Y} = BE[X] + b = B \mu_{X} + b \hspace{1cm} (1)$$
(El resultado se extiende convenientemente al caso de transformaciones lineales de matrices aleatorias $(2)$).

\item (1) $$B = (b_{ij})_{\underset{1 \leq j \leq p}{1 \leq i \leq q}} \hspace{1cm} b = (b_{i})_{1 \leq i \leq p}$$
$$E(\sum_{j=1}^{p} b_{ij}X_{i} + b_{i}) = \sum_{j=1}^{p} b_{ij} E(X_{i}) + b_{i}$$
$$E[BX + b] = BE[X] + b$$

\item (2) $X$ matriz aleatoria $p \times q$. Sean $B$ una matriz $m \times p$ (cte.), $C$ una matriz $q \times n$ (cte.) y $D$ una matriz $m \times n$ (cte.). Construimos:
$$W = BXC + D$$
$W$ es una matriz aleatoria $m \times n$.

Definimos
$$E[X] = (E[X_{ij}])_{\underset{j = 1, \dots, q}{i = 1, \dots, p}}$$
$$E[W] = (E[W_{kl}])_{\underset{l = 1, \dots, n}{k = 1, \dots, m}} = BE[X]C + D$$

\item Se define la matriz de covarianzas de $X$ como
$$\Sigma_{X} = Cov(X) := E[(X - \mu_{X})(X - \mu_{X})'] = \begin{pmatrix}
\sigma_{11} & \dots & \sigma_{1p} \\
\vdots & \ddots & \vdots \\
\sigma_{p1} & \dots & \sigma_{pp}
\end{pmatrix}$$
con
$$\sigma_{ij} = E[(X_{i}-\mu_{i})(X_{j}-\mu_{j})] \hspace{1cm} (= \mu_{ji})$$
(siempre que existan las esperanzas unidimensionales).

En particular,
$$\sigma_{ii} = E[(X_{i} - \mu_{i})^{2}] = Var(X_{i}) \hspace{1cm} (= \sigma_{i}^{2}, \text{ notación})$$

\item Propiedades:
\begin{enumerate}
\item $\Sigma_{X}$ es simétrica.

\item Los elementos de la diagonal de $\Sigma_{X}$ son no negativos.

\item La clase de matrices de covarianzas (dim. $p \times p$) coincide con la clase de matrices simétricas definidas no negativas (dim. $p \times p$) $(3)$.
\end{enumerate}

\item (3) \begin{itemize}
\item[$\Rightarrow$)] $\forall \alpha \in \mathbb{R}^{p}$.
$$0 \leq Var(\alpha'X) = E[(\alpha'X - \underset{= \alpha'\mu}{E[\alpha'X]})^{2}] = E[(\alpha'(X - \mu))^{2}] = E[\alpha'(X - \mu)(X - \mu)'\alpha] =$$
$$= \alpha'E[(X - \mu)(X - \mu)']\alpha = \alpha'\Sigma_{X}\alpha$$

\item[$\Leftarrow$)] $r = rango(\Sigma_{p \times p}), r \leq p, \Sigma = C_{p \times r}C', rango(C) = r$. \\
Definimos $Y$ un v.a. $r \times 1$ con $\mu_{y} = 0$ y $Cov(Y) = I_{r \times r}$ tal que $X = CY$.
$$E[X] = C - E[Y] = 0$$
$$Cov(X) = E[XX'] = E[(CY)(CY)'] = E[CYY'C'] = CE[YY']C' = CC' = \Sigma$$
Hemos encontrado un v.a. que tiene a $\Sigma$ como matriz de covarianzas.
\end{itemize}

\item En relación con la propiedad 3, para cualquier matriz de covarianzas, $\Sigma$, se distinguen los casos siguientes:
\begin{enumerate}[label=(\Alph*)]
\item $\Sigma$ definida positiva (notación: $\Sigma > 0$).

En este caso, $\Sigma$ es no singular, con $|\Sigma| > 0$, y $\exists \Sigma^{-1}$.

``Normalización'' y distancia de Mahalanobis: Dado un vector aleatorio $p$-dimensional $X \sim (\mu, \Sigma)$, con $\Sigma > 0$, para cualquier elección de una matriz $C$ de dimensión $p \times p$ tal que $\Sigma = CC'$ se obtiene una ``normalización'' (en origen y escala multidimensionales) del vector mediante la transformación
$$Z = C^{-1}(X - \mu)$$
En efecto, se tiene que $Z \sim (0_{p}, I_{p \times p})$. \hspace{1cm} $(4)$

\begin{itemize}
\item (4) Veamos que $E[Z] = 0_{p}, Cov(Z) = \Sigma_{Z} = I_{p \times p}$
$$E[Z] = E[C^{-1}(X-\mu)] = C^{-1}E[X-\mu] = C^{-1}(\underset{= \mu}{E[X]} - \mu) = 0_{p}$$
$$Cov(Z) = \Sigma_{Z} = E[ZZ'] = E[C^{-1}(X-\mu)(X-\mu)'(C^{-1})'] =$$
$$= C^{-1}E[(X-\mu)(X-\mu)'](C^{-1})' = C^{-1}C C'(C-1)' = I_{p \times p}$$

\item Nota: $A$ matriz ortogonal.
$$AA' = A'A = I$$
$$\Sigma = CC' = CIC' = CAA'C' = CA(CA)'$$
Es decir, la transformación $\Sigma = CC'$ no es única.
\end{itemize}

Se define la distancia de Mahalanobis de $X \sim (\mu, \Sigma)$ (con $\Sigma > 0$) con respecto a su vector de medias $\mu$ como
$$\Delta(X, \mu) := \{(X - \mu)' \Sigma^{-1} (X - \mu)\}^{\frac{1}{2}}$$

Interpretación y observaciones:
\begin{itemize}
\item Se comprueba fácilmente que $\Delta(X, \mu) = ||Z||_{p}$, para cualquier normalización $Z$ de $X$ ($||\cdot||_{p}$ denota la norma euclídea en el espacio $\mathbb{R}^{p}$) $(5)$.

\item $\Delta(X, \mu)$ es una variable aleatoria, cumpliéndose que
$$E[\Delta^{2}(X, \mu)] = p \hspace{1cm} (6)$$

\item La ecuación
$$\Delta(x, \mu) = k$$
para $k \geq 0$ (cte.) y $x \in \mathbb{R}^{p}$, define un hiperelipsoide en $\mathbb{R}^{p}$, de tal modo que los puntos transformados por normalización se corresponden con la esfera euclídea $p$-dimensional de radio $k$ con centro en el origen 0 (7).

Nota: Esfera para $Z$, elipsoide para $X$.

Los ejes del elipsoide vienen determinados por los autovectores. La longitud de los semiejes son los inversos de las raíces cuadradas de los autovalores de la matriz $(k^{2}\Sigma)^{-1}$, es decir, $k\sqrt{\lambda_{j}}$.

\item (5) $$\Delta(X, \mu) = ((X - \mu)' \Sigma^{-1} (X - \mu))^{\frac{1}{2}} = ((X - \mu)' (CC')^{-1} (X - \mu))^{\frac{1}{2}} =$$
$$= ((X - \mu) (C')^{-1} C^{-1} (X-\mu))^{\frac{1}{2}} = ((X - \mu)' (C^{-1})' C^{-1} (X - \mu))^{\frac{1}{2}} =$$
$$= ((C^{-1}(X - \mu))' C^{-1} (X - \mu))^{\frac{1}{2}} = (Z'Z)^{\frac{1}{2}} = ||Z||_{2}$$

\item (6) $E[\Delta^{2}(X, \mu)] = p$.
$$Z \sim (0_{p}, I_{p \times p}) \hspace{1cm} Z = C^{-1}(X - \mu) \hspace{1cm} \Sigma = CC'$$
$$E[\Delta^{2}(X, \mu)] = E[(X - \mu)'(CC')^{-1}(X - \mu)] = E[Z'Z] =$$
$$= \sum_{i=1}^{p} E[Z_{i}^{2}] - \underset{= 0}{\underline{E[Z_{i}]E[Z_{i}]}} = \sum_{i=1}^{p} Var(Z_{i}) = \sum_{i=1}^{p} 1 = p$$

\item (7) $\Delta(x, \mu) = k, k \geq 0, x \in \mathbb{R}^{p}$.
$$\Delta(x, \mu) = ||z|| \hspace{1cm} z = C^{-1}(x - \mu) \hspace{1cm} \Sigma = CC'$$
$$||z|| = k$$
$$\Delta(x, \mu) = k \Rightarrow \{(x-\mu)' \Sigma^{-1} (x - \mu)\}^{\frac{1}{2}} = k \Rightarrow (x - \mu)' \Sigma^{-1} (x - \mu) = k^{2} \Rightarrow$$
$$\Rightarrow (x - \mu)' (k^{2}\Sigma)^{-1} (x - \mu) = 1$$
Como $\Sigma$ definida positiva, $k^{2} \Sigma$ es definida positiva y $(k^{2} \Sigma)^{-1}$ también lo es, por lo que la última ecuación representa un elipsoide.

Si los autovalores de $\Sigma$ son $\lambda_{j}$, los de $k^{2}\Sigma$ son $k^{2}\lambda_{j}$ y los de $(k^{2}\Sigma)^{-1}$ son $(k^{2}\lambda_{j})^{-1}$.
\end{itemize}

\item $\Sigma$ semidefinida positiva (notación $\Sigma \geq 0$ indicará, en general, ``definida no negativa'').

En este caso, $\Sigma$ es singular, es decir, $|\Sigma| = 0$ y $\nexists \Sigma^{-1}$.

Por tanto, no se puede realizar una normalización en todo el espacio $\mathbb{R}^{p}$, ni definir la distancia de Mahalanobis de $X \sim (\mu, \Sigma)$ con respecto a su vector de medias $\mu$ a nivel $p$-dimesional.

Observaciones:
\begin{itemize}
\item Se tendrá que, siendo $rango(\Sigma) = r < p$, se puede escribir
$$\Sigma = CC', \text{ con } C \text{ matriz } p \times r \text{ de rango } r$$

\item Como consecuencia: Con probabilidad 1, las componentes del vector aleatorio $X = (X_{1}, \dots, X_{p})'$ cumplirán (al menos) una relación de dependencia lineal del tipo
$$\alpha'X = k, \text{ con } \alpha \neq 0 \hspace{1cm} (8)$$
(es decir, la variabilidad de $X$ se sitúa ($P_{X}-$c.s.) en un hiperplano afín en $\mathbb{R}^{p}$)

\item (8) $\Sigma_{X}$ semidefinida positiva $\Rightarrow \exists \alpha$ autovector asociado autovalor 0.
$$Y = \alpha'X$$
$$\mu_{Y} = E[Y] = E[\alpha'X] = \alpha' E[X] = \alpha' \mu_{X} = k$$
$$Var[Y] = E[(Y - \mu_{y})^{2}] = E[(\alpha'X - \alpha'\mu_{X})^{2}] = E[(\alpha'[X - \mu_{X}])^{2}] =$$
$$= E[\alpha'(X - \mu_{X})(X - \mu_{X})(\alpha')'] = \alpha' E[(X - \mu_{X})'(X - \mu_{x})] \alpha  = \underset{0_{p}}{\alpha' \Sigma_{X} \alpha} = 0 \Rightarrow$$
$$\Rightarrow P[Y=k] = 1$$
\end{itemize}
\end{enumerate}

\item Transformaciones lineales: Sean $X \sim (\mu_{X}, \Sigma_{X})$ vec. a. $p$-dimensional, $Y = BX + b$ vec. a. $q$-dimensional, con $B$ matriz $q \times p$ (cte.) y $b$ vector $q \times 1$ (cte.).

Entonces, se tiene que
$$Y \sim (\mu_{Y}, \Sigma_{Y}) = (B \mu_{X} + b, B \Sigma_{X} B') \hspace{1cm} (9)$$

\item (9) $$E[Y] = E[BX + b] = BE[X] + b = B\mu_{X} + b$$
$$Var[Y] = E[(Y - \mu_{y})(Y - \mu_{Y})'] = E[(Bx+b - B\mu_{X}-b)(Bx+b - B\mu_{X}-b)'] =$$
$$=E[(B(X - \mu_{X}))(B(X - \mu_{X}))'] = E[B(X - \mu_{X})(X - \mu_{X})B'] =$$
$$= B E[(X - \mu_{X})(X - \mu_{X})']B' = B\Sigma_{X}B'$$

\item Algunas medias globales de variación:
\begin{itemize}
\item $|\Sigma| = \prod_{j=1}^{p} \lambda_{j}$ (determinante).

\item $tr(\Sigma) = \sum_{j=1}^{p} \sigma_{j}^{2} = \sum_{j=1}^{p} \lambda_{j}$ (traza).
\end{itemize}
($\lambda_{1}, \dots, \lambda_{p}$ autovalores de $\Sigma$)
\end{itemize}

\subsubsection{Función característica}
\begin{itemize}
\item Recordatorio: Dado un vector aleatorio $X = (X_{1}, \dots, X_{p})'$, se define su función característica como:
$$\phi_{X}(t) = E[e^{it'X}] = \int_{\Omega} e^{it'X(w)} P(dw) = \int_{\mathbb{R}^{d}} e^{it'x}P_{X}(dx), \forall t = (t_{1}, \dots, t_{p})' \in \mathbb{R}^{p}$$

\item Teorema: Sea $X = (X_{1}, \dots, X_{p})'$ un vector aleatorio. Se tiene que la distribución (multivariante) de $X$ queda unívocamente determinada por el conjunto de todas las distribuciones (univariantes) de variables aleatorias de la forma
$$\alpha' X, \forall \alpha \in \mathbb{R}^{p} \hspace{1cm} (10)$$

\item (10) Sabemos que $\phi_{X}(t) = E[e^{it'X}]$. Tomamos $Y_{n} = \alpha'X = \sum_{j=1}^{p} \alpha_{j}X_{j} \Rightarrow \phi_{Y_{\alpha}}(t) = E[e^{it(\alpha'X)}]$.

Tomando $t = 1 \Rightarrow \phi_{Y_{\alpha}} (1) = E[e^{i\alpha'X}] = \phi_{X}(\alpha)$. \\
Tomando $Y_{t} = t'X \Rightarrow \phi_{Y_{t}}(1) = \phi_{X}(t)$.

\item Conocer todas las distribuciones de $X_{1}, \dots, X_{p}$ no nos permiten conocer la de $X$ excepto en el caso de independencia, mientras que conocer todas las combinaciones de la forma $\alpha_{1}X_{1} + \dots + \alpha_{p}X_{p}$ si nos lo permite.
\end{itemize}

\subsubsection{Momentos y cumulantes}

Sea $X = (X_{1}, \dots, X_{p})'$ un vector aleatorio con función característica $\phi_{X}(t), t \in \mathbb{R}^{p}$.
\begin{itemize}
\item Momentos: Se define el momento (no centrado) $p$-dimensional de orden $(r_{1}, \dots, r_{p})$ de $X$ como
$$\mu_{r_{1} \dots r_{p}}^{1 \dots p} = E[X_{1}^{r_{1}} \dots X_{p}^{r_{p}}]$$

Los momentos pueden obtenerse a partir de la función característica, derivándose de su expansión de Taylor (respecto al origen):
$$\phi(t) = E[e^{it'x}] = E[\sum_{r=0}^{\infty} \frac{1}{r!} (it'X)^{r}] = \sum_{r=0}^{\infty} \sum_{r_{1} + \dots + r_{p} = r} \mu_{r_{1} \dots r_{p}}^{1 \dots p} \frac{(it_{1})^{r_{1}} \dots (it_{p})^{r_{p}}}{r_{1}! \dots r_{p}!}$$
\end{itemize}

En particular, se tiene el siguiente resultado:
\begin{itemize}
\item Teorema: Si $E[|X_{1}^{m_{1}}| \dots |X_{p}|^{m_{p}}] < \infty$, entonces la función característica de $X$ es $(m_{1}, \dots, m_{p})$ veces continuamente diferenciable, y
$$\frac{\delta^{m}}{\delta_{t_{1}}^{m_{1}} \dots \delta_{t_{p}}^{m_{p}}} \phi(t) \mid_{t=0} = i^{m}E[X_{1}^{m_{1}} \dots X_{p}^{m_{p}}] = i^{m} \mu_{m_{1} \dots m_{p}}^{1 \dots p}$$
($m = m_{1} + \dots + m_{p}$)
\end{itemize}

Cumulantes: Sea $X = (X_{1}, \dots, X_{p})'$ un vector aleatorio con función característica $\phi_{X}(t)$, $t \in \mathbb{R}^{p}$. Consideramos la función
$$\log \phi(t)$$
\begin{itemize}
\item Se definen los cumulantes $p$-dimensional de orden ($r_{1}, \dots, r_{p}$) de $X$ como los coeficientes de la correspondiente expansión (respecto al origen),
$$\log \phi(t) = \sum_{r=0}^{\infty} \sum_{r_{1} + \dots + r_{p} = r} \kappa_{r_{1} \dots r_{p}}^{1 \dots p} \frac{(it_{1})^{r_{1}} \dots (it_{p})^{r_{p}}}{r_{1}! \dots r_{p}!}$$
\end{itemize}

\subsubsection{Cambio de variables}
\begin{itemize}
\item Teorema: Sea $X = (X_{1}, \dots, X_{p})'$ un vector aleatorio con función de densidad $f_{X}(x), x \in \mathbb{R}^{p}$, positiva sobre $S \subseteq \mathbb{R}^{p}$ y continua-

Sea $Y = (Y_{1}, \dots, Y_{p})'$ un vector aleatorio con
$$Y = g(x) = (g_{1}(X), \dots, g_{p}(X))',$$
siendo la restricción
$$g = (g_{1}, \dots, g_{p})': S \to T \equiv g(S) \subseteq \mathbb{R}^{p}$$
una aplicación biyectiva, y sea $g^{-1} =: h = (h_{1}, \dots, h_{p})'$. Supongamos que existen las derivadas parciales
$$\frac{\delta h_{i} (y)}{\delta y_{j}}, (i, j = 1, \dots, p)$$
y son continuas sobre $T$.

Entonces, la función de densidad $f_{Y}(y)$, $y \in \mathbb{R}^{p}$, del vector aleatorio $Y = g(X)$ viene dada por
$$f_{Y}(y) = f_{X}(g^{-1}(y)) \cdot abs(J_{g^{-1}}(y))$$
siendo $J_{g^{-1}}(y) = [J_{g}(g^{-1}(y))]^{-1}$ el determinante de la matriz jacobiana de la transformación $g^{-1}$.

\item Caso lineal: Sean $X$ vec.a. $p$-dimensional, $Y$ vec.a. $p$-dimensional, definido por la transformación
$$Y = g(X) := BX + b$$
con $B$ matriz $p \times p$ (cte.) no singular, $b$ vector $p \times 1$ (cte.).

Entonces, en este caso, se tiene que
$$X = B^{-1}(Y - b)$$
$$J_{g^{-1}}(\cdot) \equiv |B^{-1}| = |B|^{-1}$$
$$f_{Y}(y) = f_{X}(B^{-1}(y-b)) \cdot abs(|B|^{-1})$$
\end{itemize}

\subsection{Descomposición Matriciales Diagonalización}

$A$ matriz cuadrada $p \times p$, simétrica, definida no negativa
$$A = QDQ' \iff Q'AQ = D$$
\begin{itemize}
\item $Q$ matriz ortogonal, $QQ' = Q'Q = I$.
\item $D$ matriz diagonal (autovalores).
\end{itemize}

Como es definida no negativa, podemos escribirlo como:
$$A = QD^{\frac{1}{2}}D^{\frac{1}{2}}Q' = (QD^{\frac{1}{2}})(QD^{\frac{1}{2}})'$$
(ya que $D^{\frac{1}{2}} = (D^{\frac{1}{2}})'$)

Llamando $C = D^{\frac{1}{2}}$, podemos tomar otra matriz $H$ otrogonal tal que $HH' = H'H = I$ y tenemos:
$$A = CC' = CIC' = CHH'C' = (CH)(CH)' = EE'$$

Tomando $H = Q'$ tenemos:
$$A = QD^{\frac{1}{2}}Q'QD^{\frac{1}{2}}Q' = (QD^{\frac{1}{2}}Q')(QD^{\frac{1}{2}}Q') = RR = R^{2}$$
Esta $R$ se interpreta como la raíz cuadrada de $A$ ($R = A^{\frac{1}{2}}$).

\subsubsection{Descomposición de Cholesky}

Es de la forma
$$A = TT'$$
donde $T$ es una matriz triangular superior (y por tanto $T'$ es triangular inferior), con todos los elementos de la diagonal no negativos.

Si $rango(A) = r$, entonces $T$ tiene $p - r$ filas nulas.

\subsection{Caso \texorpdfstring{$\Sigma > 0$}: definiciones, propiedades y caracterizaciones}

\subsubsection{Definición en términos de la densidad}
\begin{itemize}
\item Definición: Sea $X = (X_{1}, \dots, X_{p})'$ un vector aleatorio. Se dice que $X$ tiene una distribución normal $p$variante si su densidad es de la forma
$$f_{X}(x) = \frac{1}{(2\pi)^{\frac{p}{2}} |\Sigma|^{\frac{1}{2}}} exp\{-\frac{1}{2}(x-\mu)' \Sigma^{-1} (x-\mu)\}, \forall x \in \mathbb{R}^{p}$$
siendo $\mu = (\mu_{1}, \dots, \mu_{p})' \in \mathbb{R}^{p}$ y $\Sigma$ una matriz escalar $(p \times p)$-dimensional simétrica y definida positiva.

Se denota $X \sim N_{p}(\mu, \Sigma)$, representando $\mu$ y $\Sigma$ los ``parámetros'' de la distribución.

Se demuestra que $f_{X}$ está bien definida, es decir, que cumple las condiciones para ser una  función de densidad:
\begin{enumerate}
\item $f_{X}(x) \geq 0, \forall x \in \mathbb{R}^{p}$.
\item $\int_{\mathbb{R}^{p}} f_{X}(x) dx = 1 \hspace{1cm} (1)$.
\item[(1)] $\int_{\mathbb{R}^{p}} \frac{1}{(2\pi)^{\frac{p}{2}}|\Sigma|^{\frac{1}{2}}} exp\{-\frac{1}{2}(X - \mu)' \Sigma^{-1} (X - \mu)\} dX$. \\
Al ser $\Sigma > 0$, tomamos una descomposición $\Sigma = CC'$, con $C$ matriz cuadrada $p \times p$ no singular. Hacemos el siguiente ``cambio de variable'':
$$Z = C^{-1}(X - \mu)$$
$$X = CZ + \mu$$
Elegimos $C$ tal que $|C| > 0$. Supongamos que $|C| < 0$, tomamos $A$ tal que $\Sigma = (CA)(CA)' = CAA'C'$ con $AA' = A'A = I$ de forma que $|CA| > 0$, y tomamos $CA$ como C.

El determinante del Jacobiano de la transformación:
$$|C^{-1}| = |C|^{-1} \Rightarrow dX = |C|^{-1} dZ \Rightarrow dZ = |C| dX$$
$$\Sigma^{-1} = (CC')^{-1} = (C^{-1})'C^{-1}$$
$$|\Sigma| = |C||C'| = |C|^{2} \Rightarrow |\Sigma|^{\frac{1}{2}} = |C|$$

Con todo esto, nos queda:
$$\int_{\mathbb{R}^{p}} \frac{1}{(2\pi)^{\frac{p}{2}}|\Sigma|^{\frac{1}{2}}} exp\{-\frac{1}{2}(X - \mu)' \Sigma^{-1} (X - \mu)\} dX = \int_{\mathbb{R}^{p}} \frac{1}{(2\pi)^{\frac{p}{2}}|C|} exp\{-\frac{1}{2}Z'Z\} |C| dZ =$$
$$= \int_{\mathbb{R}^{p}} \frac{1}{(2\pi)^{\frac{p}{2}}} exp\{-\frac{1}{2} \sum_{j=1}^{p} z_{j}^{2}\} dZ = \prod_{j=1}^{p} \int_{\mathbb{R}} \frac{1}{(2\pi)^{\frac{1}{2}}} exp\{-\frac{1}{2} z_{j}^{2}\} dz_{j} = \prod_{j=1}^{p} 1 = 1$$
\end{enumerate}

\item Nota: Esto corresponde a
$$f_{X}(x) = \frac{1}{\sqrt{2\pi}\sigma}e^{-\frac{1}{2}(\frac{x-\mu}{\sigma})}$$
\end{itemize}

\subsubsection{Vector de medias y matriz de covarianzas}
\begin{itemize}
\item A partir de cualquier factorización de la forma
$$\Sigma = CC'$$
para alguna matriz $C$ de dimensión $p \times p$ no singular (por ser $\Sigma > 0$), se tiene que el vec.a. $Z = (Z_{1}, \dots, Z_{p})'$ definido por:
$$Z = C^{-1}(X - \mu) \hspace{2cm} (\text{es decir, con } X = CZ + \mu)$$
se distribuye con función de densidad
$$f_{Z}(z) = \prod_{j=1}^{p} \frac{1}{\sqrt{2\pi}} exp\{-\frac{1}{2} z^{\frac{1}{2}}\} = \prod_{j=1}^{p} f_{Z_{j}}(z_{j})$$
Es decir, para cada $j = 1, \dots, p$,
$$Z_{j} \sim N_{1}(0, 1)$$
siendo $Z_{1}, \dots, Z_{p}$ independientes (por tanto, incorreladas), y
$$Z \sim N_{p}(0, I_{p})$$

\item Se comprueba de forma inmediata que $\mu$ y $\Sigma$ representan, respectivamente, el vector de medias y la matriz de covarianzas del vector aleatorio $X$, es decir:
$$ \mu_{X} = \mu \hspace{1cm} (2)$$
$$ \Sigma_{X} = \Sigma \hspace{1cm} (3)$$

\item (2) $CC' = \Sigma$.
$$\mu_{X} = E[X] = E[CZ + \mu] = CE[Z] + \mu = \mu$$
\item (3) $$\Sigma_{X} = E[(X-\mu)(X-\mu)'] = E[(CZ+\mu-\mu)(CZ+\mu-\mu)'] = CE[ZZ']C' = CC'$$
Donde hemos utilizado que $Z_{1}, \dots, Z{p}$ son independientes:
$$E[Z_{1}Z_{2}] = E[Z_{1}]E[Z_{2}] = 0$$
$$E[ZZ'] = \begin{pmatrix}
z_{1}^{2} & z_{2}z_{1} & \dots \\
z_{1}z_{2} & z_{2}^{2} & \dots \\
\vdots & \vdots & \ddots
\end{pmatrix} = \begin{pmatrix}
z_{1}^{2} & 0 & \dots \\
0 & z_{2}^{2} & \dots \\
\vdots & \vdots & \ddots
\end{pmatrix} = I$$
\end{itemize}

\subsubsection{Algunas propiedades (caso \texorpdfstring{$\Sigma > 0$}))}
\begin{itemize}
\item[1] Cambio de variables lineal: Sean
$$X \sim N_{p}(\mu_{X}, \Sigma_{X}) \hspace{1cm} (\Sigma > 0)$$
$$Y = BX + b$$
con $B$ matriz $p \times p$ (cte.) no singular y $b$ vector $p \times 1$ (cte.).

Entonces, se tiene que:
$$Y \sim N_{p}(\mu_{Y}, \Sigma_{Y}) = N_{p}(B \mu_{X} + b, B\Sigma_{X}B') \hspace{1cm} (4)$$

\item (4) $X \sim N_{p}(\mu_{X}, \Sigma_{X}),  Y = BX + b$
$$f_{Y}(y) = F_{X}(B^{-1}(y-b)) abs(|B|^{-1}) =$$
$$= \frac{1}{(2\pi)^{\frac{p}{2}}|\Sigma|^{\frac{1}{2}}} exp\{-\frac{1}{2} (B^{-1}(y-b) - \mu)' \Sigma^{-1} (B^{-1}(y-\mu) - b)\} abs(|B|^{-1}) =$$
Utilizamos que $I = B^{-1}B$, y que $abs(|B^{-1}|) = \sqrt{|B|^{-2}}$.
$$= \frac{1}{(2\pi)^{\frac{p}{2}} |\Sigma|^{\frac{1}{2}} |B|^{\frac{1}{2}} |B|^{\frac{1}{2}}} exp\{-\frac{1}{2} (y-B\mu-b)' (B \Sigma B')^{-1} (y-B\mu-b)\}$$
($|\Sigma|^{\frac{1}{2}} |B|^{\frac{1}{2}} |B|^{\frac{1}{2}} = |B \Sigma B'|^{\frac{1}{2}}$)
$$Y \sim N_{p}(B\mu+b, B \Sigma B')$$

\item Caracterización 1: Un vector aleatorio $p$-dimensional $X$ tiene una DNM no singular, $X \sim N_{p}(\mu, \Sigma)$ ($\Sigma > 0$), si y sólo si
$$X \overset{d}{=} AZ + \mu$$
con $A$ matriz $p \times p$ (cte.) no singular, $AA' = \Sigma$ y $Z \sim N_{p}(0, I_{p})$ (5).

($\overset{d}{=}$ igual en distribución, tienen la misma distribución)

\item (5)
\begin{itemize}
\item[$\Leftarrow$] Supongamos $X \overset{d}{=} AZ + \mu \Rightarrow X \sim N_{p}(A \cdot 0 + \mu, A I_{p} A') = N_{p}(\mu, AA') = N_{p}(\mu, \Sigma)$.
\item[$\Rightarrow$] Supongamos $X \sim N_{p}(\mu, \Sigma), \Sigma > 0$. \\
Sea $A_{p \times p}$ no singular tal que $\Sigma = AA'$. Definimos $Z = A^{-1}(X - \mu) \Rightarrow X = AZ + \mu$.
$$Z = A^{-1}X + (A^{-1}\mu) \Rightarrow Z \sim N_{p} (A^{-1} \mu + (-A^{-1}\mu), A^{-1}\Sigma(A^{-1})') \equiv$$
$$\equiv N_{p} (0, A^{-1}AA'(A^{-1})') \equiv N_{p}(0, I_{p})$$
\end{itemize}

\item[2] Independencia y condicionamiento: Vemos a continuación algunos resultados fundamentales (dos sobre independencia y uno sobre condicionamiento) que, entre sobre posibilidades, pueden tratarse a partir de la definición de la DNM no singular en términos de su densidad.

\begin{itemize}
\item Resultado 1: Sea $X = (X_{1}, \dots, X_{p})'$ un vector aleatorio con DNM no singular,
$$X \sim N_{p}(\mu, \Sigma) \hspace{1cm} (\Sigma > 0)$$
Si la matriz $\Sigma$ es diagonal,
$$\Sigma = \begin{pmatrix}
\sigma_{1}^{2} & 0 & \dots & 0 \\
0 & \sigma_{2}^{2} & \dots & 0 \\
\vdots & \vdots & \ddots & \vdots \\
0 & 0 & \dots & \sigma_{p}^{2}
\end{pmatrix} = diag(\sigma_{1}^{2}, \dots, \sigma_{p}^{2})$$
entonces las variables aleatorios componentes del vector, $X_{i}, i = 1, \dots, p$, son independientes y tienen un DN univariante (no degenerada),
$$X_{i} \sim N(\mu_{i}, \sigma_{i}^{2}) \hspace{1cm} (\sigma_{i}^{2} > 0), \hspace{1cm} i = 1, \dots, p \hspace{1cm} (6)$$

\item (6) $$f_{X}(x) = \frac{1}{(2\pi)^{\frac{p}{2}}|\Sigma|^{\frac{1}{2}}} exp\{-\frac{1}{2}(x - \mu)' \Sigma^{-1}(x - \mu)\} =$$
$$= \frac{1}{(2\pi)^{\frac{p}{2}}|\prod_{i}^{p} \sigma_{i}^{2}|^{\frac{1}{2}}} exp\{-\frac{1}{2} (x-\mu)' \begin{pmatrix}
\sigma_{1}^{-2} & 0 & \dots & 0 \\
0 & \sigma_{2}^{-2} & \dots & 0 \\
\vdots & \vdots & \ddots & \vdots \\
0 & 0 & \dots & \sigma_{p}^{-2}
\end{pmatrix} (x - \mu)\} =$$
$$= \frac{1}{(2\pi)^{\frac{p}{2}}|\prod_{i}^{p} \sigma_{i}^{2}|^{\frac{1}{2}}} exp\{-\frac{1}{2} (\sigma_{1}^{-2} (x_{1} - \mu_{1})^{2} \cdot ... \cdot \sigma_{p}^{-2}(x_{p} - \mu_{p})^{2})\} =$$
$$= \prod_{i=1}^{p} \frac{1}{(2\pi)^{\frac{1}{2}}|\sigma_{i}^{2}|^{\frac{1}{2}}} exp\{-\frac{1}{2} \sigma_{i}^{-2} (x_{i} - \mu_{i})^{2}\} = \prod_{i=1}^{p} \frac{1}{(2\pi)^{\frac{1}{2}}|\sigma_{i}^{2}|^{\frac{1}{2}}} exp\{-\frac{1}{2} (\frac{x_{i} - \mu_{i}}{\sigma_{i}})^{2}\} =$$
$$= \prod_{i=1}^{p} f_{X_{i}}(x) \hspace{2cm} X_{i} \sim N(\mu_{i}, \sigma_{i}^{2})$$

\item El recíproco de este resultado es cierto en el siguiente sentido:

Si $X = (X_{1}, \dots, X_{p})'$ es un vector aleatorio con componentes (mutuamente) independientes, cada una de ellas con DN univariantes (no degenerada),
$$X_{i} \sim N(\mu_{i}, \sigma_{i}^{2}) \hspace{1cm} (\sigma_{i}^{2}), \hspace{1cm} i = 1, \dots, p$$
entonces $X$ tiene DNM no singular,
$$X \sim N_{p}(\mu, \Sigma) \hspace{1cm} (\Sigma > 0)$$
(Observación: El resultado no es válido si solo se supone la incorrelación, o incluso la ``independencia dos a dos'', entre las componentes del vector).

\item El Resultado 1 se puede generalizar al caso en que la matriz $\Sigma$ es una matriz diagonal por cajas, estableciéndose las conclusiones en términos de los correspondientes subvectores del vector $X$ (se enunciará, por simplicidad, en el caso de dos subvectores):

\item Resultado 2: Sea $X = (X_{1}, \dots, X_{p})'$ un vector aleatorio con DNM no singular,
$$X \sim N_{p}(\mu, \Sigma) \hspace{1cm} (\Sigma > 0)$$
Supongamos que las componentes de $X$ están ordenadas de tal modo que para la partición del vector
$$X = \begin{pmatrix} X_{(1)} \\ X_{(2)} \end{pmatrix}$$
con $X_{(1)} = (X_{1}, \dots, X_{q})', X_{(2)} = (X_{1+q}, \dots, X_{p})'$, se tiene
$$\mu = \begin{pmatrix} \mu_{(1)} \\ \mu_{(2)} \end{pmatrix} \hspace{1cm} \Sigma = \begin{pmatrix}
\Sigma_{(11)} & \Sigma_{(12)} \\
\Sigma_{(21)} & \Sigma_{(22)}
\end{pmatrix} = \begin{pmatrix}
\Sigma_{(11)} & 0 \\
0 & \Sigma_{(22)}
\end{pmatrix}$$
(i. e., $\Sigma = diag(\Sigma_{(11)}, \Sigma_{(22)})$). Entonces, los vectores aleatorios $X_{(1)}$ y $X_{(2)}$ son (mutuamente) independientes y tienen cada uno DNM, de dimensiones respectivas $q$ y $p-q$,
$$\begin{aligned} X_{(1)} \sim N_{q}(\mu_{(1)}, \Sigma_{(11)}) \\
X_{(2)} \sim N_{p-q}(\mu_{(2)}, \Sigma_{(22)}) \end{aligned} \hspace{1cm} (7)$$

\item (7) $$f_{X}(x) = \frac{1}{(2\pi)^{\frac{p}{2}}|\Sigma|^{\frac{1}{2}}} exp\{-\frac{1}{2} (x - \mu)'\Sigma(x - \mu)\} = (*) \hspace{1cm} X \sim N_{p}(\mu, \Sigma)$$
\begin{itemize}
\item $|\Sigma| = |\Sigma_{(11)}||\Sigma_{(22)}| \hspace{1cm} |\Sigma|^{\frac{1}{2}} = |\Sigma_{(11)}|^{\frac{1}{2}}|\Sigma_{(22)}|^{\frac{1}{2}} \hspace{1cm} \Sigma^{-1} = \begin{pmatrix}
(\Sigma_{(11)})^{-1} & 0 \\
0 & (\Sigma_{(22)})^{-1}
\end{pmatrix}$
\item $(x-\mu)'\Sigma(x-\mu) = (X_{(1)} - \mu_{(1)})(X_{(2)} - \mu_{(2)}) \begin{pmatrix}
(\Sigma_{(11)})^{-1} & 0 \\
0 & (\Sigma_{(22)})^{-1}
\end{pmatrix} \begin{pmatrix}
X_{(1)}-\mu_{(1)} \\
X_{(2)}-\mu_{(2)}
\end{pmatrix} =$ \\ 
$= (X_{(1)}-\mu_{(1)})'(\Sigma_{(11)})^{-1}(X_{(1)}-\mu_{(1)}) + (X_{(2)}-\mu_{(2)})'(\Sigma_{(22)})^{-1}(X_{(2)}-\mu_{(2)})$
\end{itemize}
\begin{tiny}
$$(*) = \frac{1}{(2\pi)^{\frac{p}{2}}|\Sigma_{(11)}|^{\frac{1}{2}}|\Sigma_{(22)}|^{\frac{1}{2}}} exp\{-\frac{1}{2}(X_{(1)}-\mu_{(1)})'(\Sigma_{(11)})^{-1}(X_{(1)}-\mu_{(1)})\} exp\{-\frac{1}{2}(X_{(2)}-\mu_{(2)})'(\Sigma_{(22)})^{-1}(X_{(2)}-\mu_{(2)})\} =$$
\end{tiny}
$$= f_{X_{(1)}}(x_{(1)}) f_{X_{(2)}}(x_{(2)}) \hspace{2cm} (\frac{p}{2} = \frac{q+p-q}{2})$$

\item Resultado 3: Sea (como en el Resultado 2)
$$X \sim N_{p}(\mu, \Sigma) \hspace{1cm} (\Sigma > 0)$$
con el particionamiento
$$X = \begin{pmatrix} X_{(1)} \\ X_{(2)} \end{pmatrix} \hspace{1cm} \mu = \begin{pmatrix} \mu_{(1)} \\ \mu_{(2)} \end{pmatrix} \hspace{1cm} \Sigma = \begin{pmatrix}
\Sigma_{(11)} & \Sigma_{(12)} \\
\Sigma_{(21)} & \Sigma_{(22)}
\end{pmatrix}$$
Entonces, se tiene que:
\begin{enumerate}
\item Los vectores $X_{(1)}$ y $X_{(2)} - \Sigma_{(21)}\Sigma_{(11)}^{-1}X_{(1)}$ son independientes.
\item Dichos vectores se distribuyen según
\begin{itemize}
\item $X_{(1)} \sim N_{q}(\mu_{(1)}, \Sigma_{(11)})$.
\item $X_{(2)} - \Sigma_{(21)}\Sigma_{(11)}^{-1}X_{(1)} \sim N_{p-q} (\mu_{(2)} - \Sigma_{(21)}\Sigma_{(11)}^{-1}\mu_{(1)}, \Sigma_{(22)} - \Sigma_{(21)}\Sigma_{(11)}^{-1}\Sigma_{(12)})$.
\end{itemize}
\item La distribución condicionada de $X_{(2)}$ dado $X_{(1)} = x_{(1)}$ es una DNM, de la forma
$$N_{p-q}(\mu_{(2)} + \Sigma_{(21)}\Sigma_{(11)}^{-1}(x_{(1)} - \mu_{(1)}), \Sigma_{(22)} - \Sigma_{(21)}\Sigma_{(11)}^{-1}\Sigma_{(12)}) \hspace{1cm} (8)$$
(En adelante, se denotará $\Sigma_{22 \cdot 1} = \Sigma_{(22)} - \Sigma_{(21)}\Sigma_{(11)}^{-1}\Sigma_{(12)}$)
\end{enumerate}

\item (8) Definimos la matriz
$$C = \begin{pmatrix}
I_{(1)} & 0 \\
-\Sigma_{(21)}\Sigma_{(11)}^{-1} & I_{(2)}
\end{pmatrix} \hspace{1cm} |C| = 1 \Rightarrow C \text{ no singular }$$
Consideramos el cambio de variables dado por la transformación:
$$Y = CX = C \begin{pmatrix} X_{(1)} \\ X_{(2)} \end{pmatrix} = \begin{pmatrix}
X_{(1)} \\
X_{(2)} - \Sigma_{(21)}\Sigma_{(11)}^{-1}X_{(1)}
\end{pmatrix} = \begin{pmatrix} Y_{(1)} \\ Y_{(2)} \end{pmatrix} = Y \hspace{1cm} X = C^{-1}Y$$
$$Y \sim N_{p}(C \mu, C \Sigma C') = N_{p}(C \begin{pmatrix} \mu_{(1)} \\ \mu_{(2)} \end{pmatrix}, C \begin{pmatrix} \Sigma_{(11)} & \Sigma_{(12)} \\ \Sigma_{(21)} & \Sigma_{(22)} \end{pmatrix} C^{-1}) =$$
$$= N_{p} (\begin{pmatrix} \mu_{(1)} \\ \mu_{2} - \Sigma_{(21)}\Sigma_{(11)}^{-1}\mu_{(1)} \end{pmatrix}, \begin{pmatrix}
\Sigma_{(11)} & 0 \\
0 & \Sigma_{(22)} - \Sigma_{(21)}\Sigma_{(11)}^{-1}\Sigma_{(12)} = \Sigma_{22 \cdot 1}
\end{pmatrix})$$
Por el Resultado 2, los vectores $X_{(1)}$ y $X_{(2)} - \Sigma_{(21)}\Sigma_{(11)}^{-1}X_{(1)}$ son independientes.

Nos queda ver el punto 3. Al tratarse de una distribución continua
$$f_{X_{(2)} \mid X_{(1)} = x_{(1)}} (x_{(2)}) = \frac{f_{X_{(1)}X_{(2)}}(x_{(1)}, x_{(2)})}{f_{X_{(1)}}(x_{(1)})} \overset{\text{c.v.}}{=} \frac{f_{Y_{(1)}Y_{(2)}}(x_{(1)}, x_{(2)} - \Sigma_{(21)}\Sigma_{(11)}^{-1}x_{(1)}) \cdot 1}{f_{X_{(1)}}(x_{(1)})} =$$
$$= \frac{\frac{1}{(2\pi)^{\frac{p}{2}}(|\Sigma_{11}| |\Sigma_{22 \cdot 1}|)^{\frac{1}{2}}} exp\{-\frac{1}{2}[A + B(x_{(2)})]\}}{\frac{1}{(2\pi)^{\frac{q}{2}} |\Sigma_{(11)}|^{\frac{1}{2}}} exp\{-\frac{1}{2}A\}} = \frac{1}{(2\pi)^{\frac{(p-1)}{2}} |\Sigma_{22 \cdot 1}|^{\frac{1}{2}}} exp\{-\frac{1}{2} B x_{(2)}\}$$
Nota: El cambio de variable lo vimos en la sección 1.3.4. Cambio de variables.
$$Cx = \begin{pmatrix}
I_{(1)} && 0 \\
-\Sigma_{(21)} \Sigma_{(11)}^{-1} & I_{(2)}
\end{pmatrix} \begin{pmatrix}
x_{(1)} \\ x_{(2)}
\end{pmatrix} = \begin{pmatrix}
x_{(1)} \\ x_{(2)} - \Sigma_{(21)}\Sigma_{(11)}^{-1}x_{(1)}
\end{pmatrix}$$
\begin{scriptsize}
$$\begin{pmatrix}
X_{(1)} \\ X_{(2)} - \Sigma_{(21)} \Sigma_{(11)}^{-1} x_{(1)}
\end{pmatrix} - \begin{pmatrix}
\mu_{(1)} \\ \mu_{(2)} - \Sigma_{(21)} \Sigma_{(21)} \Sigma_{(11)}^{-1} \mu_{(1)}
\end{pmatrix} = \begin{pmatrix}
X_{(1)} - \mu_{(1)} \\
X_{(2)} - (\mu_{(2)} + \Sigma_{(21)} \Sigma_{(11)}^{-1}(x_{1} - \mu_{(1)}))
\end{pmatrix} \Rightarrow$$
$$\Rightarrow \begin{pmatrix}
X_{(1)} - \mu_{(1)} \\
X_{(2)} - (\mu_{(2)} + \Sigma_{(21)} \Sigma_{(11)}^{-1}(x_{1} - \mu_{(1)}))
\end{pmatrix}' \begin{pmatrix}
\Sigma_{(11)}^{-1} & 0 \\
0 & \Sigma_{22 \cdot 1}^{-1}
\end{pmatrix}^{-1} \begin{pmatrix}
X_{(1)} - \mu_{(1)} \\
X_{(2)} - (\mu_{(2)} + \Sigma_{(21)} \Sigma_{(11)}^{-1}(x_{1} - \mu_{(1)}))
\end{pmatrix} =$$
\end{scriptsize}
\begin{tiny}
$$= (X_{(1)} - \mu_{(1)})' \Sigma_{(11)}^{-1}(X_{(1)} - \mu_{(1)}) + (x_{(2)} - (\mu_{(2)} + \Sigma_{(21)}\Sigma_{(11)}^{-1}(X_{(1)} - \mu_{(1)}))' \Sigma_{22 \cdot 1}^{-1} (x_{(2)} - (\mu_{(2)} + \Sigma_{(21)}\Sigma_{(11)}^{-1}(x_{(1)} - \mu_{(1)}))) =$$
\end{tiny}
$$ = A + B(x_{(2)})$$

Por lo que tenemos:
$$N_{p-q} (\mu_{(2)} + \Sigma_{(21)} \Sigma_{(11)}^{-1}(x_{(1)} - \mu_{(1)}), \Sigma_{(22)} - \Sigma_{(21)} \Sigma_{(11)}^{-1} \Sigma_{(12)})$$
\end{itemize}
\end{itemize}

\subsubsection{Función característica de la DNM (caso \texorpdfstring{$\Sigma > 0$}))}
La función característica de la DNM, en el caso no singular ($\Sigma > 0$), puede obtenerse de forma directa a partir de la función de densidad.

\begin{itemize}
\item Caracterización 2: Un vector aleatorio $p$-dimensional $X$ con vector de medias $\mu$ y matriz de covarianzas $\Sigma > 0$ tiene una DNM no singular, $X \sim N_{p}(\mu, \Sigma)$, si y sólo si su función característica tiene la forma
$$\phi_{X}(t) = exp\{it'\mu - \frac{1}{2}t'\Sigma t\} \hspace{1cm} t \in \mathbb{R}^{p} \hspace{1cm} (9)$$
(Observación: Se verá más adelante que en el casos singular, con $\Sigma$ semidefinida positiva, la función característica tiene la misma expresión formal).

\item (9) $$\phi_{X}(t) = E[e^{it'X}] \hspace{1cm} \forall t \in \mathbb{R}^{p}$$
$$\phi_{X}(t) = \int_{\mathbb{R}^{p}} e^{it'x} f_{X}(x)dx = \int_{\mathbb{R}^{p}} \frac{1}{(2\pi)^{\frac{p}{2}} |\Sigma|^{\frac{1}{2}}} exp\{it'x - \frac{1}{2}(x - \mu)'\Sigma^{-1}(x - \mu)\} dx =$$
Hacemos un cambio de variable: $\Sigma = CC', Y = C^{-1}(X - \mu) \Rightarrow X = CY + \mu$:
$$= \int_{\mathbb{R}^{p}} \frac{1}{(2\pi)^{\frac{p}{2}}(|C||C'|)^{\frac{1}{2}}} exp\{it'(CY + \mu) - \frac{1}{2}Y'Y\} |C| dy =$$
$$= e^{it'\mu} \int_{\mathbb{R}^{p}} \frac{1}{(2\pi)^{\frac{p}{2}}} exp\{i(t'C)Y - \frac{1}{2}Y'Y\} dy =$$
Llamamos $t'C = \alpha = (\alpha_{1}, \dots, \alpha_{p})$.
$$= e^{it'\mu} \int_{\mathbb{R}^{p}} \frac{1}{(2\pi)^{\frac{1}{2}}} exp\{\sum_{j=1}^{p}[i \alpha_{j} y_{j} - \frac{1}{2}y_{j}^{2}]\} dy =$$
$$= \prod_{j=1}^{p} e^{it_{j}\mu_{j}} \phi_{Y_{j}}(\alpha_{j}) =$$
donde $\phi_{N(\mu, \sigma^{2})}(t) = e^{it\mu - \frac{1}{2}\sigma^{2}t^{2}}$, por lo que $\phi_{N(0, 1)}(t) = e^{-\frac{1}{2}t^{2}}$, y como $Y \sim N(0, I_{p \times p}) \Rightarrow Y_{j} \sim N(0,1)$.
$$= \prod_{j=1}^{p} e^{it_{j}\mu_{j} - \frac{1}{2} \alpha_{j}^{2}} =$$
$$= exp\{\sum_{j=1}^{p} (it_{j}\mu_{j} - \frac{1}{2} \alpha_{j}^{2})\} = e^{it'\mu - \frac{1}{2}t't\Sigma t}$$
donde hemos usado $\sum_{j=1}^{p} t_{j}\mu_{j} = t'\mu$, $\sum_{j=1}^{p} \alpha_{j}^{2} = \alpha'\alpha = t'CC't = t' \Sigma t$.

\item Recordatorio: $X$ v.a. (multidimensional) $\sim N(\mu, \sigma^{2})$, $\mu \in \mathbb{R}, \sigma^{2} > 0$. Función característica:
$$\phi_{X}(t) = e^{it\mu - \frac{1}{2}\sigma^{2}t^{2}} \hspace{1cm} t \in \mathbb{R}$$
En el caso $Z \sim N(0, 1)$, tenemos:
$$\phi_{Z}(t) = e^{\frac{-1}{2}t^{2}} \hspace{1cm} t \in \mathbb{R}$$

\item Frente al uso directo de la función de densidad (caso no singular), la función característica es muy útil en la derivación de diversos resultados, algunos de los cuales se ven a continuación.

\item Resultado 4: Sean
$$X \sim N_{p}(\mu_{X}, \Sigma_{X}) \hspace{1cm} (\Sigma_{X} > 0)$$
$$Y = BX + b$$
con $B$ matriz $q \times p$ (cte.) de rango $q$ ($\Rightarrow q \leq p$) y $b$ vector $q \times 1$ (cte.).

Entonces, se tiene que
$$Y \sim N_{q}(\mu_{Y}, \Sigma_{Y}) = N_{q}(B\mu_{X}+b, B\Sigma_{X}B')$$
(siendo $rango(B\Sigma_{X}B') = q$, es decir, $B\Sigma_{X}B' > 0$) \hspace{1cm} (10)

\item (10) $$\phi_{Y}(t) = E[e^{it'(BX + b)}] = E[e^{it'BX} e^{it'b}] = e^{it'b} E[e^{it'BX}] =$$
$$= e^{it'b} \phi_{X}(B't) = e^{it'b} exp\{i(B't)'\mu_{X} - \frac{1}{2}(B't)'\Sigma_{X}(B't)\} =$$
$$= exp\{it'(B\mu_{X}+b) - \frac{1}{2}t'B\Sigma_{X}B't\} = exp\{it'\mu_{y} - \frac{1}{2}t'\Sigma_{Y}t\} \Rightarrow$$
$$\Rightarrow Y \sim N_{q}(\mu_{Y}, \Sigma_{Y}) = N_{q}(B\mu_{X}+b, B\Sigma_{X}B')$$ 

Sea $A$ una matriz cuadrada no singular ($p \times p$).
\begin{itemize}
\item Si $B$ matriz ($m \times p$), entonces $rango(BA) = rango(B)$.
\item Si $C$ matriz ($p \times n$), entonces $rango(AC) = rango(C)$.
\end{itemize}
Si $D$ es una matriz ($m \times n$), entonces $rango(DD') = rango(D'D) = rango(D) = rango(D')$.

En este caso, tenemos $\Sigma_{X} = CC'$, por lo que:
$$B\Sigma_{X}B' = BCC'B' = (BC)(BC)'$$
con $B$ no singular, y $rango(C) = q$, entonces $rango(BC) = rango(B) = q$, y por lo tanto, $rango((BC)(BC)') = rango(BC) = q$.
\end{itemize}

\subsubsection{Marginalización (caso \texorpdfstring{$\Sigma > 0$}))}
En particular, del resultado anterior, se obtiene el siguiente, sobre normalidad de las distribuciones marginales en una DNM
\begin{itemize}
\item Resultado 5: Sea
$$X = (X_{1}, \dots, X_{p})' \sim N_{p}(\mu, \Sigma) \hspace{1cm} (\Sigma > 0)$$
Entonces, para todo subvector
$$X_{r} = (X_{r_{1}}, \dots, X_{r_{q}})', \hspace{1cm} r = (r_{1}, \dots, r_{1})', \hspace{1cm} q \leq p$$
(con $r_{1}, \dots, r_{1}$ distintos entre sí) se tiene que
$$X_{r} \sim N_{q}(\mu_{r}, \Sigma_{r}) \hspace{1cm} (\Sigma_{r} > 0)$$
siendo
$$\mu_{r} \text{ el subvector de } \mu \text{ correspondiente a } r$$
$$\Sigma_{r} \text{ la submatriz principal de } \Sigma \text{ correspondiente a } r$$
$$(11)$$

\item (11) Tomamos $B_{q \times p} = (b_{ij})_{\underset{j = 1, \dots, p}{i = 1, \dots, q}}$ con $b_{ij} = \begin{cases}
1 & \text{si } j = r_{i}\\
0 & \text{si } j \neq r_{i} 
\end{cases}$.
$$X_{r} = BX + b \sim N_{q}(B\mu + b, B\Sigma_{X}B')$$
con $B\mu + b = B \mu = \mu_{r}$ y $B \Sigma B' = \Sigma_{r}$, por lo que $X_{r} \sim N_{q}(\mu_{r}, \Sigma_{r})$.
\end{itemize}

\subsubsection{Caracterización de la DNM en términos de normalidad de combinaciones lineales de las componentes (caso \texorpdfstring{$\Sigma > 0$}))}
\begin{itemize}
\item Caracterización 3: Un vector aleatorio $p$-dimensional $X = (X_{1}, \dots, X_{p})'$ tiene una DNM no singular si y solo si toda combinación de la forma
$$\alpha'X, \hspace{1cm} \text{con } \alpha \in \mathbb{R}^{p} \setminus \{0\}$$
tiene una DN univariante no degenerada (i. e., con varianza no nula) \hspace{1cm} (12).

\item (12)
\begin{itemize}
\item[$\Rightarrow$] $X \sim N_{p}(\mu, \Sigma), \Sigma > 0$, $\alpha = (\alpha_{1}, \dots, \alpha_{p}) \in \mathbb{R}^{p} \setminus \{0\}$. \\
$Y_{\alpha} = a'X \Rightarrow Y_{\alpha} \sim N(\alpha'\mu, \alpha'\Sigma\alpha)$
\item[$\Leftarrow$] $\alpha \in \mathbb{R}^{p} \setminus \{0\}$, $Y_{\alpha} = \alpha'X \sim N(\mu_{\alpha}, \sigma_{\alpha}^{2}), \sigma_{\alpha}^{2} > 0$. \\
$\mu_{\alpha} = \alpha'\mu_{X}, \sigma_{\alpha}^{2} = \alpha'\Sigma_{X}\alpha > 0$. \\
$\phi_{X}(t) = \phi_{Y_{t}}(1), t \in \mathbb{R}^{p}$. \\
$\phi_{Y_{t}}(s) = exp\{is\mu_{t} - \frac{1}{2}\sigma_{t}^{2}s^{2}\}$. \\
$\phi_{Y_{t}}(1) = exp\{it'\mu_{t} - \frac{1}{2}t'\Sigma_{X}t\} \Rightarrow X \sim N_{p}\{\mu_{X}, \Sigma_{X}'\}, \Sigma_{X} > 0$.
\end{itemize}
\end{itemize}

\subsubsection{Caracterización de la DNM en términos de la densidad esférica estándar (caso \texorpdfstring{$\Sigma > 0$}))}
\begin{itemize}
\item Sea $X = (X_{1}, \dots, X_{p})'$ un vector aleatorio con DNM no singular,
$$X \sim N_{p}(\mu, \Sigma) \hspace{1cm} (\Sigma > 0)$$
Para alguna elección de una matriz $C$ no singular con $\Sigma = CC'$, consideramos la normalización de $X$ dada por el vector aleatorio
$$Z := C^{-1}(X - \mu) \sim N_{p}(0, I_{p})$$
Definimos ahora el vector aleatorio
$$U := \frac{Z}{||Z||} = \frac{Z}{(Z'Z)^{\frac{1}{2}}}$$
Puesto que $||U|| = ||\frac{Z}{||Z||}|| = \frac{||Z||}{||Z||} = 1$, se tiene que $U$ se distribuye sobre la esfera unidad en $\mathbb{R}^{p}, \mathcal{S}_{p}$. Además, se cumple que
$$U \overset{d}{=} HU$$
para toda matriz ortogonal $H$ de dimensión $p$ (i.e., $HH' = H'H = I_{p}$); es decir, $U$ ha de tener la distribución uniforme sobre $\mathcal{S}_{p}$ (distribución esférica sobre la esfera unidad) (13).

\item[(13)] $$HU = H \frac{Z}{||Z||} = H \frac{Z}{(Z'Z)^{\frac{1}{2}}} = H \frac{Z}{(Z'H'HZ)^{\frac{1}{2}}} = \frac{HZ}{((HZ)(HZ)')^{\frac{1}{2}}} =$$
$$= \frac{HZ}{||HZ||} \overset{*}{=} \frac{HZ}{||Z||} \overset{d}{=} \frac{Z}{||Z||} \Rightarrow HU \overset{d}{=} U$$
$*$ se debe a:
$$Z \sim N_{p}(0, I_{p})$$
$$HZ \sim N_{p}(H \cdot 0, H I_{p} H') = N_{p}(0, I_{p})$$

\item Un resultado importante: El vector aleatorio
$$U = \frac{Z}{(Z'Z)^{\frac{1}{2}}}$$
y la variable aleatoria
$$R = (Z'Z)^{\frac{1}{2}}$$
son independientes.

(Este resultado se demuestra, en general, para vectores aleatorios con distribución esférica)

\item Caracterización 4: Un vector aleatorio $p$-dimensional $X = (X_{1}, \dots, X_{p})'$ tiene una DNM no singular,
$$X \sim N_{p}(\mu, \Sigma) \hspace{1cm} (\Sigma > 0)$$
si y sólo si las realizaciones de $X$ pueden generarse a partir del siguiente procedimiento (experimento aleatorio):
\begin{enumerate}
\item Observar el valor de una variable aleatoria $R^{2} := V \sim \mathcal{X}_{p}^{2}$ (Se identifica $V=v$).
\item Elegir aleatoriamente (i.e., según la distribución uniforme), y de forma independiente, un punto en la esfera $p$-dimensional de radio $r = \sqrt{v}$ (Se identifica como $Z = (Z_{1}, \dots, Z_{p})' = (z_{1}, \dots, z_{p})' = z$).
\item Dados un vector de medias $\mu$ y una matriz de covarianzas $\Sigma > 0$, eligiendo $C$ no singular tal que $\Sigma = CC'$, obtener el valor de $X$ mediante la expresión $X := CZ + \mu$ (Se identifica $X = x = Cz + \mu$).
\end{enumerate}
\end{itemize}

\subsection{Extensión al caso general, \texorpdfstring{$\Sigma \geq 0$})}

\subsubsection{Introducción}
\begin{itemize}
\item En el apartado anterior se ha introducido la DNM en el caso $\Sigma > 0$, dándose varias formulaciones equivalentes, una de las cuales se ha adoptado como definición y las otras como caracterizaciones:
\begin{itemize}
\item[D] En términos de la densidad asociada (se requiere formalmente la condición $|\Sigma| \neq 0$ y, por tanto, $\exists \Sigma^{-1}$).

\item[C-I] Como transformación lineal (no singular) de un vector de componentes independientes con DN univariante estándar.

\item[C-II] En términos de la forma de la función característica.

\item[C-III] Bajo la condición de normalidad (no degenerada) de combinaciones lineales de las componentes.

\item[C-IV] Mediante transformaciones lineales (no singulares) a partir de la densidad esférica estándar y radio aleatorio $\sqrt{\chi_{p}^{2}}$ (donde también se requiere explícitamente la existencia de $\Sigma^{-1}$)
\end{itemize}

\item Con el fin de dar una formulación de la DNM extendida al caso general en que $\Sigma \geq 0$ (pudiendose ser singular, es decir, semidefinida positiva), se observa que [C-I], [C-II] y [C-III] ofrecen una viabilidad más directa, especialmente en el caso de [C-II].

\item En este apartado, se trata de introducir dicha formulación extendida partiendo de la identificación de la DNM por su función característica (enfoque [C-II]) y discutiendo las implicaciones con respecto a las otras formas de interpretación.
\end{itemize}

\subsubsection{Sobre la descomposición espectral de la matriz de covarianzas, \texorpdfstring{$\Sigma \geq 0$})}
\begin{itemize}
\item Sea $\Sigma$ la matriz de covarianzas de algún vector aleatorio $p-$dimensional. Por ser $\Sigma \geq 0$ y simétrica, puede factorizarse de la forma:
$$\Sigma = H \Lambda H'$$
con $H$ una matriz $p \times p$ ortogonal ($HH' = H'H = I_{p}$), $\Lambda$ una matriz $p \times p$ diagonal, cuyos elementos diagonales distintos son los autovalores de $\Sigma$.

Sea $r = rango(\Sigma), r \leq p$. Supongamos, por conveniencia, que las matrices se eligen de modo que
$$\Lambda = \begin{pmatrix}
D & 0 \\
0 & 0
\end{pmatrix}$$
con $D$ una matriz $r \times r$ no singular, escribiendo
$$H = \begin{pmatrix} H_{1} & H_{2} \end{pmatrix}$$
con $H_{1}$ una matriz $p \times r$ y $H_{2}$ una matriz $p \times (p - r)$.

Se tiene entonces que
$$\Sigma = \begin{pmatrix} H_{1} & H_{2} \end{pmatrix} \begin{pmatrix}
D & 0 \\
0 & 0
\end{pmatrix} \begin{pmatrix} H_{1}' \\ H_{2}' \end{pmatrix} = H_{1} D H_{1}'$$
\end{itemize}

\subsubsection{Función característica de la DNM (caso \texorpdfstring{$\Sigma \geq 0$}))}
Introducimos formalmente la DNM en el caso general ($\Sigma \geq 0$) mediante la identificación de su función característica:

\begin{itemize}
\item Definición (extensión de [C-II]): Un vector aleatorio $p-$dimensional $X$ con vector de medias $\mu$ y matriz de covarianzas $\Sigma \geq 0$ tiene una DNM, $X \sim N_{p}(\mu, \Sigma)$, su y sólo si su función característica tiene la forma:
$$\phi_{X}(t) = exp\{it'\mu - \frac{1}{2}t' \Sigma t\}$$

\item Antes de tratar las extensiones de [C-I] y [C-II] a partir de esta definición, veremos un resultado general sobre transformaciones lineales.
\end{itemize}

\subsubsection{Transformaciones lineales de rango no necesariamente máximo (caso \texorpdfstring{$\Sigma \geq 0$}))}
\begin{itemize}
\item Resultado: Sean
$$X \sim N_{p}(\mu_{X}, \Sigma_{X}) \hspace{1cm} (\Sigma_{X} \geq 0)$$
$$Y = BX + b$$
con $B$ matriz $q \times p$ (cte.) ($\Rightarrow rango(B) \leq \min\{q, p\}$), $b$ vector $q \times 1$ (cte.).

Entonces, se tiene que
$$Y \sim N_{q}(\mu_{Y}, \Sigma_{Y}) = N_{q}(B\mu_{X} + b, B\Sigma_{X}B')$$
(siendo $rango(B\Sigma_{X}B') \leq \min\{rango(B), rango(\Sigma_{X})\}$) (1).

\item[(1)] $t \in \mathbb{R}^{q}$,
$$\phi_{Y}(t) = E[e^{it'Y}] = E[e^{it'(BX+b)}] = e^{it'b}E[e^{it'BX}] = e^{it'b}E[e^{i(B't)'X}] = e^{it'b}\phi_{X}(B't) =$$
$$= e^{it'b}[e^{i(B't)\mu_{X}-\frac{1}{2}(B't)'\Sigma_{X}(B't)}] = e^{it'(B\mu_{X}+b)-\frac{1}{2}t'(B\Sigma_{X}B')t} = e^{it'\mu_{Y}-\frac{1}{2}t'\Sigma_{y}t} \Rightarrow$$
$$\Rightarrow Y \sim N_{q}(B\mu_{X}+b, B\Sigma_{X}B')$$
Para ver el rango:
$$rango(B(\Sigma_{X}B')) \leq \min\{rango(B), rango(\Sigma_{X}B')\}$$
Como:
$$rango(\Sigma_{X}B') \leq \min\{rango(\Sigma_{X}), rango(B)\}$$
tenemos:
$$rango(B\Sigma_{X}B') \leq \min\{rango(B), rango(\Sigma_{X})\}$$
\end{itemize}

\subsubsection{Extensión de la caracterización [C-I] (caso \texorpdfstring{$\Sigma \geq 0$}))}
\begin{itemize}
\item Intuitivamente, en el caso en que la matriz $\Sigma$ sea singular, la variabilidad del vector aleatorio $X$, con DNM (versión extendida) podrá ser explicada (con probabilidad 1) en términos de un número de variables aleatorias, con DN univariante e independientes, igual al rango de $\Sigma$. A continuación, vemos la argumentación formal para llegar al enunciado de la correspondiente caracterización, como extensión de [C-I].

\item A partir de la descomposición espectral de $\Sigma$, vista al principio, de la forma:
$$\Sigma = H_{1}DH_{1}'$$
y teniendo en cuenta que
$$HH' = \begin{pmatrix} H_{1} & H_{2} \end{pmatrix} \begin{pmatrix} H_{1}' \\ H_{2}' \end{pmatrix} = H_{1}H_{1}' + H_{2}H_{2}'$$
podemos escribir la función característica de $X$ como:
$$\phi_{X}(t) = e^{it'\mu - \frac{1}{2}t'\Sigma t} = e^{it'HH'\mu - \frac{1}{2}t'H_{1}DH_{1}'t} = e^{it'H_{1}H_{1}'\mu - \frac{1}{2}t'H_{1}DH_{1}t}e^{it'H_{2}H_{2}'\mu}$$

\item Esta expresión sugiere considerar el cambio de variables lineal
$$Y := H'X = \begin{pmatrix} H_{1}' \\ H_{2}' \end{pmatrix} X = \begin{pmatrix} H_{1}'X \\ H_{2}'X \end{pmatrix} =: \begin{pmatrix} Y_{1} \\ Y_{2} \end{pmatrix}$$
de modo que, por ser $H$ ortogonal, se tiene también que $X = HY$.

Los momentos de primer y segundo orden de $Y$ vienen dados por
$$\mu_{y} = H'\mu = \begin{pmatrix} H_{1}'\mu \\ H_{2}'\mu \end{pmatrix} = \begin{pmatrix} \mu_{Y_{1}} \\ \mu_{Y_{2}} \end{pmatrix}$$
$$\Sigma_{Y} = H'\Sigma H = H'H \Lambda H'H = \Lambda = \begin{pmatrix}
D & 0 \\
0 & 0
\end{pmatrix} =: \begin{pmatrix}
\Sigma_{Y_{1}} & 0 \\
0 & \Sigma_{Y_{2}}
\end{pmatrix}$$
Es decir,
$$Y_{1} \sim N_{r}(H_{1}'\mu, D)$$
$$Y_{2} \sim N_{p-r}(H_{2}'\mu, 0)$$
(es decir, la distribución degenerada en $H_{2}'\mu: P[Y_{2} = H_{2}'\mu] = 1$)

\item Por tanto, tenemos que la función característica de $X$ se puede interpretar como
$$\phi_{X}(t) = E[e^{it'X}] = E[E^{it'HY}] = \phi_{Y}(H't)$$
Con el cambio de variables
$$v := H't = \begin{pmatrix} H_{1}' \\ H_{2}' \end{pmatrix} t = \begin{pmatrix} H_{1}'t \\ H_{2}'t \end{pmatrix} =: \begin{pmatrix} v_{1} \\ v_{2} \end{pmatrix}$$
podemos escribir
$$\phi_{Y}(v) = \phi_{X}(t) = e^{it'H_{1}H_{1}'\mu - \frac{1}{2}t'H_{1}DH_{1}'t} e^{it'H_{2}H_{2}'\mu} = e^{iv_{1}'\mu_{Y_{1}} - \frac{1}{2}v_{1}'\Sigma_{Y_{1}}v_{1}}e^{iv_{2}'\mu_{Y_{2}}} = \phi_{Y_{1}}(v_{1})\phi_{Y_{2}}(v_{2})$$
Es decir, $Y_{1}$ e $Y_{2}$ son independientes.

\item Ahora, puesto que $Y_{1} \sim N_{r} (H_{1}'\mu, D)$, con $D > 0$, se tiene por [C-I] que $Y_{1}$ puede representarse en distribución como
$$Y_{1} \overset{d}{=} D^{\frac{1}{2}}Z + H_{1}'\mu$$
con $Z \sim N_{R}(0, I_{r})$.

Para $Y_{2}$ se puede escribir
$$Y_{2} \overset{P-c.s.}{=} H_{2}'\mu \equiv 0Z + H_{2}'\mu$$
siendo aquí 0 una matriz $(p-r) \times r$.

Conjuntamente (teniendo en cuenta que $\overset{P-c.s.}{=} \Rightarrow \overset{d}{=}$), podemos escribir
$$Y = \begin{pmatrix} Y_{1} \\ Y_{2} \end{pmatrix} \overset{d}{=} \begin{pmatrix}
D^{\frac{1}{2}}Z + H_{1}'\mu \\
0Z + H_{2}'
\end{pmatrix} \begin{pmatrix} D^{\frac{1}{2}} \\ 0 \end{pmatrix}Z + H'\mu$$
Se deshace el cambio de variables multiplicando por $H$,
$$X = NY \overset{d}{=} H \begin{pmatrix} D^{\frac{1}{2}} \\ 0 \end{pmatrix}Z + HH'\mu = H_{1}D^{\frac{1}{2}}Z + \mu$$
Finalmente, denotando $A = H_{1}D^{\frac{1}{2}}$, se tiene que
$$X \overset{d}{=} AZ + \mu$$
con $A$ matriz $p \times r$ (cte.) de rango $r$, $Z \sim N_{r}(0, I_{r})$.

\item Recíprocamente, es inmediato que si se cumple que
$$X \overset{d}{=} AZ + \mu$$
con $A$ matriz $p \times r$ (cte.) de rango r, $Z \sim N_{r}(0, I_{r})$, teniendo en cuenta que la correspondencia biunívoca entre distribuciones y funciones características (teorema de inversión de Paul Lévy), ha de ser
$$\phi_{X} \equiv \phi_{AZ + \mu}$$
Es decir, para todo $t \in \mathbb{R}^{p}$
$$\phi_{X}(t) = \phi_{AZ + \mu}(t) = E[e^{it'(AZ + \mu)}] = e^{it'\mu} E[e^{i(t'A)Z}] = e^{it'\mu}\phi_{Z}(A't) =$$
$$= e^{it'\mu}e^{-\frac{1}{2}(t'A)(A't)} = e^{it'\mu - \frac{1}{2}t'(AA')t}$$
Por tanto, según la definición general de la DNM dada a partir de la extensión de [C-II],
$$X \sim N_{p}(\mu, \Sigma)$$
con $\Sigma_{X} = AA' \geq 0$.

\item Se puede enunciar, como conclusión, la caracterización siguiente:

Caracterización 1 (caso general): Un vector aleatorio $p$-dimensional $X$ tiene una DNM, $X \sim N_{p}(\mu, \Sigma)$ ($\Sigma \geq 0$), si y solo si
$$X \overset{d}{=} AZ + \mu$$
con $A$ matriz $p \times r$ (cte.), $rango(A) = r$, $AA' = \Sigma$, $Z \sim N_{r}(0, I_{r})$.
\end{itemize}

\subsubsection{Extensión de la caracterización [C-III] (caso \texorpdfstring{$\Sigma \geq 0$}))}
En relación con la caracterización [C-III], el resultado extendido al caso general puede enunciarse de la forma siguiente:

\begin{itemize}
\item Caracterización 3 (caso general): Un vector aleatorio $p$-dimensional $X = (X_{1}, \dots, X_{p})'$ tiene una DNM (versión extendida) si y solo si toda combinación de la forma
$$\alpha'X \hspace{1cm} \text{con } \alpha \in \mathbb{R}^{p}$$
tiene una DN univariante (posiblemente degenerada, i.e. con varianza nula) (2).

(La demostración es similar a la correspondiente al caso no singular, con mínimas modificaciones)

\item[(2)] \begin{itemize}
\item[$\Rightarrow$] Supongamos $X \sim N_{p}(\mu, \Sigma)$ $(\Sigma \geq 0)$. \\
Sea $\alpha \in \mathbb{R}^{p}$ y sea $Y = \alpha'X$. \\
Tomamos $B_{1 \times p} = \alpha'$, $b_{1 \times 1} = 0$. Aplicando el resultado de transformaciones lineales de rango no necesariamente máximo, tenemos:
$$Y_{\alpha} \sim N_{1}(\alpha'\mu, \alpha'\Sigma\alpha) \hspace{1cm} \text{con } \alpha'\Sigma\alpha \geq 0$$
en el caso en el que $\alpha'\Sigma\alpha = 0 \Rightarrow$ degenerada.

\item[$\Leftarrow$] Supongamos $Y_{\alpha} = \alpha'X \sim N_{1}(\mu_{Y_{\alpha}}, \sigma_{Y_{\alpha}}^{2})$, con $\alpha \in \mathbb{R}^{p}$. \\
Necesariamente se tiene $\mu_{Y_{\alpha}} = \alpha'\mu$ y $\sigma_{Y_{\alpha}} = \alpha'\Sigma\alpha$, por lo que $X \sim (\mu, \Sigma)$. \\
Utilizamos el Teorema de la página 8 de vectores aleatorios: La distribución de $X$ queda unívocamente definida por el conjunto de todas las distribuciones univariantes de la forma $\alpha'X$ del siguiente modo:
$$\phi_{Y_{t}}(1) = \phi_{X}(t) \hspace{1cm} \text{con } Y_{t} = t'X \text{ (hemos cambiado } \alpha \text{ por } t)$$
$$\phi_{Y_{t}}(s) = exp\{is't'X-\frac{1}{2}s't'\Sigma ts\}$$
para $s=1$:
$$\phi_{Y_{t}}(1) = exp\{it'X - \frac{1}{2}t'\Sigma t\} = \phi_{X}(t)$$
$\Rightarrow X \sim N_{p}(\mu, \Sigma)$.
\end{itemize}
\end{itemize}

\subsubsection{Discusión en relación con la posible no existencia de densidad en todo \texorpdfstring{$\mathbb{R}^{p}$}) (caso \texorpdfstring{$\Sigma \geq 0$}))}
\begin{itemize}
\item En referencia a la definición basada en la función de densidad, dada para el caso no singular ($\Sigma > 0$), en el caso general que nos ocupa ($\Sigma \geq 0$) sólo existirá propiamente la densidad para un subvector, con dimensión igual al rango de la matriz $\Sigma$, de un vector obtenido a partir de $X$ mediante un cambio de variables conveniente.

Dicho cambio estará asociado a la descomposición espectral de $\Sigma$, y el subvector corresponderá a las componentes generadas por los autovalores estrictamente positivos de $\Sigma$.

Es decir, se tiene que la distribución es no singular en un subespacio afín de dimensión $r = rango(\Sigma)$. Para el resto de dimensiones, la distribución es degenerada.

\item Análogamente, se podrá dar una representación como distribución simétrica de contornos elípticos restringida al subespacio afín de dimensión $r = rango(\Sigma)$ mencionado.
\end{itemize}

\subsubsection{Normalidad de combinaciones lineales de vectores aleatorios con DNM (versión general) independientes}
\begin{itemize}
\item Resultado: Sean $X_{k}$, $k = 1, \dots, m$, vectores aleatorios $p$-dimensionales con DNM
$$X_{k} \sim N_{p}(\mu_{k}, \Sigma_{k}) \hspace{1cm} \text{respectivamente)}$$
e independientes. Entonces, para cualquier conjunto de matrices $A_{k}$, $k = 1, \dots, m$ (ctes.), de dimensión $q \times p$, se verifica que
$$Y := \sum_{k=1}^{m} A_{k}X_{k} \sim N_{q}(\sum_{k=1}^{m}A_{k}\mu_{k}, \sum_{k=1}^{m}A_{k}\Sigma_{k}A_{k}') \hspace{1cm} (3)$$

\item (3) $X_{k} \sim N_{p}(\mu_{k}, \Sigma_{k})$ indep., $A_{k} \in \mathcal{M}_{q \times p}$ (ctes.) $\Rightarrow$ \\ $\Rightarrow Y = \sum_{k=1}^{m} A_{k}X_{k} \sim N_{q} (\sum_{i=1}^{m} A_{k}\mu_{k}, \sum_{k=1}^{m}A_{k}\Sigma_{k}A_{k}')$.

Demostración:
$$\phi_{Y}(t) = E[e^{it'Y}] = E[e^{it'(\sum_{k=1}^{m}A_{k}X_{k})}] = E[e^{\sum_{k=1}^{m} it'A_{k}X_{k}}] = E[e^{\sum_{i=1}^{m} i(A_{k}'t)'X_{k}}] =$$
$$= \prod_{k=1}^{m} E[e^{i(A_{k}'t)'X_{k}}] = \prod_{k=1}^{m} \phi_{X_{k}}(A_{k}tX_{k}) = \prod_{k=1}^{m} e^{it'A_{x}\mu_{X} - \frac{1}{2}t'A_{k}\Sigma_{k}A_{k}'t} = e^{\sum_{k=1}^{m}(it'A_{k}\mu_{k} - \frac{1}{2}t'A_{k}\Sigma_{k}A_{k}'t)} =$$
$$= e^{it'\sum_{k=1}^{m}A_{k}\mu_{k} - \frac{1}{2}t'\sum_{k=1}^{m}(A_{k}\Sigma_{k}A_{k})'t}$$
\end{itemize}

\subsubsection{[Complemento] El ``problema de Lévy-Cramér'' (teorema de Cramér)}
\begin{itemize}
\item Como consecuencia del resultado anterior, se tiene, en particular, el siguiente caso simple:

Sean $X_{1}$ y $X_{2}$ vectores aleatorios $p$-dimensionales con
$$X_{1} \sim N_{p}(\mu_{1}, \Sigma_{1}) \hspace{1cm} X_{2} \sim N_{p}(\mu_{2}, \Sigma_{2})$$
e independientes. Entonces
$$X_{1} + X_{2} \sim N_{p}(\mu_{1} + \mu_{2}, \Sigma_{1} + \Sigma_{2})$$

\item En este caso (i.e. bajo normalidad), se tiene que el recíproco también es cierto:

Sean $X_{1}$ y $X_{2}$ vectores aleatorios $p$-dimensionales independientes tales que $X_{1} + X_{2}$ tiene DNM. Entonces, $X_{1}$ y $X_{2}$ tiene también, cada uno, DNM.

\item Históricamente, el resultado fue conjeturado por Paul Lévy, aunque no fue demostrado formalmente, por Harald Cramér, hasta 1936, mediante herramientas de la teoría de funciones analíticas, para el caso $p=1$ (4).

\item (4) $X = X_{1} + X_{2}$ tiene una DNM $\overset{C-III}{\Rightarrow} \forall \alpha \in \mathbb{R}^{n}, \alpha'X$ tiene una DN u. \\
$\alpha'X = \alpha'X_{1} + \alpha'X_{2}$ tiene DN u. \\
$X_{1}$ y $X_{2}$ son independientes $\Rightarrow \alpha X_{1}$ y $\alpha X_{2}$ son independientes. \\
$\Rightarrow \alpha'X_{1}$ y $\alpha'X_{2}$ tienen DN u. $\overset{C-III}{\Rightarrow} X_{1}$ y $X_{2}$ tienen DNM.

\item No veremos aquí la demostración. No obstante, es fácil probar (por ejemplo, mediante la caracterización [C-III]) que, partiendo de que el resultado es cierto para $p=1$, lo es inmediatamente para cualquier $p$.
\end{itemize}

\subsection{Complemento: Distribuciones esféricas y elípticas}
\subsubsection{Distribuciones esféricas}
\begin{itemize}
\item Definición: Se dice que un vector aleatorio $X = (X_{1}, \dots, X_{p})'$ tiene una distribución esférica si $X$ y $HX$ tienen la misma distribución, para toda matriz ortogonal $H$ de dimensión $p \times p$.

\item Caracterización 1 (en términos de la densidad, si existe): Un vector aleatorio $X = (X_{1}, \dots, X_{p})'$ con distribución de tipo continuo tiene una distribución esférica si y sólo si su función de densidad puede expresarse como
$$f_{X}(x) = g(x'x)$$
para alguna función escalar $g: \mathbb{R}_{0}^{+} \to \mathbb{R}_{0}^{+}$.

\item Caracterización 2 (en términos de la función característica): Un vector aleatorio \\ $X = (X_{1}, \dots, X_{p})'$ tiene una distribución esférica si y sólo si su función característica es de la forma
$$\phi_{X}(t) = \xi(t't)$$
para alguna función escalar $\xi: \mathbb{R}^{+}_{0} \to \mathbb{R}$.

\item Ejemplo 1: $Z \sim N_{p}(0, \sigma^{2}I_{p})$ tiene una distribución esférica.

(Observación: Bajo la caracterización 1 se requerirá que la distribución sea no degenerada, i.e. $\sigma^{2} > 0$) (1).

\item (1) $Z \sim (0, \sigma^{2}I_{p})$ \\
Opción 1:
$$f_{Z}(z) = \frac{1}{(2\pi)^{\frac{p}{2}} |\sigma|^{p}} exp\{-\frac{1}{2}(z-\mu)'\Sigma^{-1}(z-\mu)\} = \frac{1}{(2\pi)^{\frac{p}{2}} |\sigma|^{p}} exp\{-\frac{1}{2\sigma^{2}}Z'Z\}$$
$$Z'Z = Y$$
$$g(Y) = \frac{1}{(2\pi)^{\frac{p}{2}} |\sigma|^{p}} exp\{-\frac{1}{2\sigma^{2}}Y\}, g: \mathbb{R}^{+}_{0} \to \mathbb{R}^{+}_{0}$$
Opción 2:
$$\phi_{Z}(t) = \exp\{it'\mu_{Z} - \frac{1}{2}t'\Sigma_{Z}t\} = exp\{it' \cdot 0 - \frac{1}{2}t'\sigma^{2}I_{p}t\} = exp\{-\frac{1}{2} \sigma^{2}t't\}$$
$$\xi(t't) = \xi(s) = e^{-\frac{1}{2}\sigma^{2}s}, \xi: \mathbb{R}^{+}_{0} \to \mathbb{R}$$

\item Ejemplo 2: Sea $X$ un vector aleatorio con distribución esférica y tal $P[X=0]=0$. Se define $U := \frac{X}{||X||}$, para $X \neq 0$, y con cualquier asignación cuando $X = 0$ (por ejemplo, en la propia esfera unidad, $\mathcal{S}_{p} \subset \mathbb{R}^{p}$). Entonces $U$ también tiene una distribución esférica (2).

\item Resultado auxiliar: Sean $X$ e $Y$ vectores aleatorios $p$-dimensionales. Sea $g: \mathbb{R}^{p} \to \mathbb{R}^{q}$ una aplicación medible (Borel). Entonces:
$$X \overset{d}{=} Y \Rightarrow g(X) \overset{d}{=} g(Y)$$

\item Ejemplo 3: Más generalmente, si $X = (X_{1}, \dots, X_{p})'$ es un vector aleatorio $p$-dimensional con distribución esférica, se tiene entonces que, dada cualquier transformación radial ``isotrópica'' Borel-medible $h: \mathbb{R}^{p} \to \mathbb{R}^{p}$, de la forma
$$h(x) = g(||x||)x, \hspace{1cm} \text{con } g: \mathbb{R}_{0}^{+} \to \mathbb{R}_{0}^{+} \text{ Borel-medible}$$
el vector aleatorio transformado $h(X) = (h_{1}(X), \dots, h_{p}(X))'$ también tendrá una distribución esférica.

\item Resultado (extensión de un resultado visto anteriormente):

Si $X$ tiene una distribución esférica con $P[X=0]=0$, y se definen
$$R = ||X|| = (X'X)^{\frac{1}{2}}$$
$$U = \frac{X}{(X'X)^{\frac{1}{2}}} = R^{-1}X, \hspace{1cm} \text{para } X \neq 0$$
(con cualquier asignación para $U$ cuando $X = 0$; por ejemplo, en la propia esfera unicidad $\mathcal{S}_{p} \subset \mathbb{R}^{p}$), entonces $U$ tiene también una distribución esférica, y $R$ y $U$ son independientes.

\item Este resultado significa que la distribución de $X$, en el caso de una distribución esférica, viene en realidad determinada por la distribución de la variable aleatorio $R$.
\end{itemize}

\subsubsection{Definiciones elípticas}
\begin{itemize}
\item Definición: Se dice que un vector aleatorio $X = (X_{1}, \dots, X_{p})'$ tiene una distribución elíptica, con parámetros dados por un vecor $p$-dimensional $\mu$ y una matriz ($p \times p$)-dimensional $V$ simétrica y definida positiva, si se puede expresar de la forma
$$X = AZ + \mu$$
con $V = AA'$, $A$ matriz $p \times p$ no singular, $Z$ vector aleatorio con distribución esférica.

Se denotará, en este caso, $X \sim E_{p}(\mu, V)$.

\item Caracterización 1 (en términos de la densidad, si existe): Un vector aleatorio $X = (X_{1}, \dots, X_{p})'$ con distribución de tipo continuo tiene una distribución elíptica si y solo si su función de densidad puede expresarse como
$$f_{X}(x) = |V|^{-\frac{1}{2}}g((x-\mu)'V^{-1}(x-\mu))$$
para alguna función escalar $g: \mathbb{R}_{0}^{+} \to \mathbb{R}_{0}^{+}$.

\item Caracterización 2 (en términos de la función característica): Un vector aleatorio \\ $X = (X_{1}, \dots, X_{p})'$ tiene una distribución elíptica su y sólo si su función característica es de la forma
$$\phi_{X}(t) = e^{it'\mu}\xi(t'Vt)$$
para alguna función escalar $\xi: \mathbb{R}_{0}^{+} \to \mathbb{R}$.

\item Ejemplo: $X \sim N_{p}(\mu, \Sigma)$, ($\Sigma > 0$) tiene una distribución elíptica, $X \sim E_{p}(\mu, \Sigma)$. En este caso, por tanto, se tiene que $V = \Sigma$.

\item Observaciones:
\begin{itemize}
\item Si $X \sim E_{p}(0, \sigma^{2}I_{p})$, con $\sigma^{2} > 0$, entonces $X$ tiene una distribución esférica (y recíprocamente).

\item Si para $X \sim E_{p}(\mu, V)$ existen los momentos $E[X]$ y $Cov(X)$, entonces se tiene que
$$E[X] = \mu$$
$$Cov(X) = \alpha V, \hspace{1cm} \text{con } \alpha = -2\xi'(0)$$

\item Las distribuciones marginales de distribuciones elípticas son elípticas.

\item Las distribuciones condicionadas de distribuciones elípticas son elípticas.

\item Independencia y normalidad: Sea $X \sim E_{p}(\mu, V)$, con $V$ una matriz diagonal. Si $X_{1}, \dots, X_{p}$ son (mutuamente) independientes, entonces $X \sim N_{p}(\mu, V)$.
\end{itemize}
\end{itemize}

\subsection{Formas cuadráticas basadas en vectores aleatorios con DNM}

\subsubsection{Motivación}
\begin{itemize}
\item Sea $X = (X_{1}, \dots, X_{p})'$ un vector aleatorio y sea $A$ una matriz (cte.) ($p \times p$)-dimensional. Se considera la variable aleatoria dada por la forma cuadrática
$$X'AX$$
Se plantea, en general, el problema de estudiar la distribución de $X'AX$ a partir del conocimiento de la distribución de $X$.

(Como extensión, si $X$ se reemplaza por una matriz aleatoria de dimensión $p \times q$, la forma cuadrática resultante constituirá una matriz aleatoria de dimensión $q \times q$).

\item En particular, se considera el caso en que $X \sim N_{p}(\mu, \Sigma)$, que conduce a la distribución $\chi^{2}$ no centrada.

La distribución $\chi^{2}$ no centrada y, definida a partir de ésta, la distribución $F$ no centrada, son fundamentales, como distribuciones de diversos estadísticos de interés, en relación con la inferencia basada en la DNM.
\end{itemize}

\subsubsection{Distribuciones \texorpdfstring{$\chi^{2}$}) y \texorpdfstring{$F$}) no centradas}
DISTRIBUCIÓN $\chi^{2}$ CENTRADA.
\begin{itemize}
\item Recordemos que la distribución $\chi^{2}$ centrada con $n$ grados de libertad se define como la distribución de la suma de cuadrados de $n$ variables aleatorias independientes con distribución normal estándar:
$$Z = (Z_{1}, \dots, Z_{n})' \sim N_{n}(0, I_{n}) \rightarrow Y = Z'Z = \sum_{i=1}^{n} Z_{i}^{2} \sim \chi_{n}^{2}$$

Función de densidad: $f_{Y}(y) = \frac{1}{2^{\frac{n}{2}}\Gamma(\frac{n}{2})}y^{\frac{n}{2}-1}e^{-\frac{y}{2}}, y > 0$.

Función de distribución: $F_{Y}(y) = \frac{\gamma(\frac{n}{2}, \frac{y}{2})}{\Gamma(\frac{n}{2})}, y > 0$.

\item Nota: Se definen, $\forall z \in \mathbb{C}, \mathcal{R}(z) > 0$,
\begin{itemize}
\item Función gamma: $\Gamma(z) = \int_{0}^{\infty} t^{z-1}e^{-t} dt$.
\item Funciones gamma incompletas: para $v > 0$,
$$\Gamma(z, v) = \int_{v}^{\infty} t^{z-1}e^{-t}dt, \hspace{1cm} \gamma(z, v) = \int_{0}^{v} t^{z-1}e^{-t}dt$$
\end{itemize}

Función característica: $\phi_{Y}(t) = (1 - 2it)^{-\frac{n}{2}}, \forall t \in \mathbb{R}$, de donde se obtienen, en particular, los momentos
$$E[Y] = n$$
$$Var(Y) = 2n$$
\end{itemize}

DISTRIBUCIÓN $\chi^{2}$ NO CENTRADA.
\begin{itemize}
\item Definición (resultado): Sea $X = (X_{1}, \dots, X_{n})' \sim N_{n}(\mu, I_{n})$. Entonces, la variable aleatoria $Y = X'X$ tiene función de densidad
$$f_{Y}(y) = e^{-\frac{\delta}{2}}{}_{0}F_{1}(;\frac{1}{2}n; \frac{1}{4}\delta y) \frac{1}{2^{\frac{n}{2}}\Gamma(\frac{n}{2})}e^{-\frac{y}{2}}y^{\frac{n}{2}-1}, \text{ para } y > 0$$
siendo $\delta = \mu'\mu$. Se dice que la variable $Y$ tiene distribución $\chi^{2}$ no centrada con $n$ grados de libertad y parámetro de no centralidad $\delta$, denotándose $\chi_{n}^{2}(\delta)$.

\item Nota: En la expresión anterior, ${}_{0}F_{1}$ representa la `función hipergeométrica generalizada' de órdenes 0 y 1, también llamada función hipergeométrica confluente límite.

(Cuando $\mu=0$, se tiene la distribución $\chi_{n}^{2}$ centrada)

Función característica: $\phi_{Y}(t) = (1 - 2it)^{-\frac{n}{2}}e^{\frac{it\delta}{1-2it}}, \forall t \in \mathbb{R}$, de donde se obtienen, en particular, los momentos
$$\begin{cases}
E[Y] = n + \delta \\
Var(Y) = 2n + 4\delta
\end{cases} (1)$$

\item (1) $$\Phi_{Y}(t) = (1 - 2it)^{\frac{n}{2}} e^{\frac{it\delta}{1-2it}}$$
$$iE[Y] = \frac{\delta}{\delta t} \Phi_{Y}(t) \mid_{t=0} =$$
$$=-\frac{n}{2}(1-2it)^{-\frac{n}{2}-1}(-2i)e^{\frac{it\delta}{1-2it}} + (1-2i)^{-\frac{n}{2}} \frac{i\delta(1-2it)+2i(it\delta)}{(1-2it)^{2}}e^{\frac{it\delta}{1-2it}} \mid_{t=0} = i(n+\delta) \Rightarrow$$
$$\Rightarrow E[Y] = n + \delta$$

$$i^{2}E[Y^{2}] = \frac{\delta^{2}}{\delta t^{2}} \Phi_{Y}(t) \mid_{t=0} = \frac{\delta}{\delta t}[ni(1 - 2it)^{-\frac{n}{2}-1} e^{\frac{it\delta}{1-2it}} + (1-2it)^{-\frac{n}{2}-2} i\delta e^{\frac{it\delta}{1-2it}}] \mid_{t=0} =$$
$$= \frac{\delta}{\delta t}[(ni(1-2it)^{-1} + \delta i(1-2it)^{-2})\Phi_{y}(t)] \mit_{t=0} = [....] = -2n -4\delta + i^{2}(n+\delta)^{2} =$$
$$= - (n^{2}+2n+2n\delta+4\delta+\delta^{2}) \Rightarrow E[Y^{2}] = n^{2}+2n+2n\delta$$
$$Var[Y] = E[Y^{2}] - E[Y]^{2} = n^{2}+2n+2n\delta + 4\delta+\delta^{2}-(n+\delta)^{2} = 2n+4\delta$$

\item Nota: $X \sim N(\mu, \sigma^{2}) \Rightarrow \phi_{X^{2}}(t) = \frac{1}{(1-2i\sigma^{2}t)} exp\{\frac{it\mu}{1-2i\sigma^{2}t}\}$.

\item Un resultado de interés: Si $Y_{1}$ y $Y_{2}$ son variables aleatorias independientes con
$$Y_{1} \sim \chi_{n_{1}}^{2}(\delta_{1}), \hspace{1cm} Y_{2} \sim \chi_{n_{2}}^{2}(\delta_{2})$$
entonces se tiene que
$$Y_{1} + Y_{2} \sim \chi_{n_{1}+n_{2}}^{2}(\delta_{1} + \delta_{2}) \hspace{1cm} (2)$$

\item (2) $Y = Y_{1} + Y_{2}$.
$$\Phi_{Y}(t) = E[e^{it(Y_{1}+Y_{2})}] = E[E^{itY_{1} + itY_{2}}] \overset{Y_{1}, Y_{2} \text{ ind}}{=} E[e^{itY_{1}}]E[e^{itY_{2}}] = \Phi_{Y_{1}}(t) \Phi_{Y_{2}}(t)$$
$$Y_{1} \sim \chi^{2}_{n_{1}}(\sigma_{1}), Y_{2} \sim \chi^{2}_{n_{2}}(\sigma_{2})$$
$$Y_{1} + Y_{2} \sim \chi_{n_{1}+n_{2}}^{2}(\sigma{1} + \sigma{2})$$
\end{itemize}

DISTRIBUCIÓN $F$ CENTRADA
\begin{itemize}
\item Recordemos que la distribución $F$ centrada con $(n_{1}, n_{2})$ grados de libertad se define como la distribución del cociente:
$$F = \frac{\frac{Y_{1}}{n_{1}}}{\frac{Y_{2}}{n_{2}}}$$
con $Y_{1} \sim \chi_{n_{1}}^{2}$ e $Y_{2} \sim \chi_{n_{2}}^{2}$ (ambas centradas), independientes. Se denota $F_{n_{1}, n_{2}}$.

Puede verse como la distribución del cociente
$$\frac{\frac{1}{n_{1}}Z_{(1)}'Z_{(1)}}{\frac{1}{n_{2}}Z_{(2)}'Z_{(2)}} = \frac{\frac{1}{n_{1}} \sum_{k=1}^{n_{1}} Z_{1k}^{2}}{\frac{1}{n_{2}} \sum_{l=1}^{n_{2}} Z_{2k}^{2}}$$
con: $Z_{(1)} = (Z_{11}, \dots, Z_{1n_{1}})' \sim N_{n_{1}}(0, I_{n_{1}})$, $Z_{(2)} = (Z_{21}, \dots, Z_{2n_{1}})' \sim N_{n_{2}}(0, I_{n_{2}})$, $Z_{(1)}$ y $Z_{(2)}$ independientes; equivalentemente,
$$Z = \begin{pmatrix} Z_{(1)} \\ Z_{(2)} \end{pmatrix} \sim N_{n_{1} + n_{2}}(0, U_{n_{1} + n_{2}})$$

Función de densidad: con $n_{1}, n_{2} \in \mathbb{N} \setminus \{0\}$,
$$g_{F}(f) = \frac{1}{fB(\frac{n_{1}}{2}, \frac{n_{2}}{2})} (\frac{n_{1}f}{n_{1}f + n_{2}})^{\frac{n_{1}}{2}} (1 - \frac{n_{1}f}{n_{1}f+n_{2}})^{\frac{n_{2}}{2}}$$

Función de distribución: $G_{F}(f) = I_{\frac{n_{1}f}{n_{1}f+n_{2}}} (\frac{n_{1}}{2}, \frac{n_{2}}{2}), f \geq 0$.

\item Nota: Se definen, $\forall a, b \in \mathbb{C}$, con $\mathcal{R}(a), \mathbb{R}(b) > 0$,
\begin{itemize}
\item Función beta: $B(a,b) = \frac{\Gamma(a)\Gamma(b)}{\Gamma(a+b)} = \int_{0}^{1} t^{a-1} (1-t)^{b-1} dt$.

\item Función beta incompleta: para $x \in [0,1]$, $B(x;a,b) = \int_{0}^{x} t^{a-1} (1-t)^{b-1} dt$, (para $x = 1$ se tiene la función beta completa).

\item Función beta incompleta regularizada: $I_{x}(a,b) = \frac{B(x;a,b)}{B(a,b)}$.
\end{itemize}

Función característica (versión corregida de Philips (1982)):
$$\phi_{F}(t) = \frac{\Gamma(\frac{n_{1} + n_{2}}{2})}{\Gamma(\frac{n_{2}}{2})} U(\frac{n_{1}}{2}, 1 - \frac{n_{2}}{2}; -\frac{n_{2}}{n_{1}}it)$$

\item Nota: En la expresión anterior, $U$ representa la función hipergeométrica confluente de segunda especie,
$$U(a,b;z) = \frac{\Gamma(1-b)}{\Gamma(a+1-b)}{}_{1}F_{1}(a;b;z) + \frac{\Gamma(b-1)}{\Gamma(a)}z^{1-b} {}_{1}F_{1}(a+1-b; 2-b, z)$$
con ${}_{1}F_{1}$ la `función hipergeométrica generalizada de órdenes 1 y 1', también llamada función de Kimar de primera especie).

Se obtienen, en particular, los momentos
$$E[F] = \frac{n_{2}}{n_{2}-2}, \text{ para } n_{2} > 2 (= \infty \text{ si } n_{2} \in (0, 2])$$
$$Var(F) = \frac{2n_{2}^{2}(n_{1}+n_{2}-2)}{n_{1}(n_{2}-2)^{2}(n_{2}-4)}, \text{ para } n_{2} > 4 (=\infty \text{ si } n_{2} \in (2, 4])$$
\end{itemize}

DISTRIBUCIÓN $F$ NO CENTRADA
\begin{itemize}
\item Definición (resultado): Sean $Y_{1} \sim \chi_{n_{1}}^{2}(\delta)$ e $Y_{2} \sim \chi_{n_{2}}^{2}$, independientes. Entonces, la variable
$$F = \frac{\frac{Y_{1}}{n_{1}}}{\frac{Y_{2}}{n_{2}}}$$
tiene función de densidad
$$\begin{aligned}
g_{F}(f) =  & e^{-\frac{\delta}{2}}{}_{1}F_{1}(\frac{1}{2}(n_{1}+n_{2}); \frac{1}{2}n_{1}; \frac{-\frac{1}{2} \frac{n_{1}}{n_{2}} \delta f}{1 + \frac{n_{1}}{n_{2}}f}) & \\
& \times \frac{\Gamma(\frac{1}{2}(n_{1} + n_{2}))}{\Gamma(\frac{1}{2}n_{1}) \Gamma(\frac{1}{2}n_{2})} \frac{f^{\frac{n_{1}}{2}-1}(\frac{n_{1}}{n_{2}})^{\frac{n_{1}}{2}}}{(1 + \frac{n_{1}}{n_{2}}f) \frac{(n_{1}+n_{2})}{2}}, & \text{ para } f > 0
\end{aligned}$$

Se dice que $F$ tiene distribución $F$ no centrada con $n_{1}$ y $n_{2}$ grados de libertad y parámetro de no centralidad $\sigma$, denotándose $F_{n_{1}, n_{2}}(\delta)$.

Función característica (versión corregida de Philips (1982)):
$$\phi_{F}(t) = \frac{e^{-\frac{1}{2}\delta}}{\Gamma(\frac{1}{2}n_{2})} \sum_{j=0}^{\infty} \frac{(\frac{\delta}{2})^{j}}{j!} \Gamma(\frac{1}{2}n_{1} + \frac{1}{2}n_{2} + j) U(\frac{n_{1}}{2} + j, 1 - \frac{n_{2}}{2}; -\frac{n_{1}}{n_{2}}it)$$

Se obtienen, en particular, los momentos
$$E[F] = \frac{n_{2}(n_{1}+\delta)}{n_{1}(n_{2}-2)}, \text{ para } n_{2} > 2 (=\infty \text{ si } n_{2} \in (0,2])$$
$$Var(F) = 2 (\frac{n_{2}}{n_{1}})^{2} \frac{(n_{1}+\delta)^{2} + (n_{1}+2\delta) (n_{2}-2)}{(n_{2}-2)^{2} (n_{2}-4)}, \text{ para } n_{2} > 4 (=\infty \text{ si } n_{2} \in (2,4])$$
\end{itemize}

\subsubsection{Formas cuadráticas \texorpdfstring{$X'AX$}), con \texorpdfstring{$X \sim N_{p}(\mu, \Sigma)$}) (\texorpdfstring{$\Sigma > 0$}))}
A continuación veremos algunos resultados relativos a la distribución de formas cuadráticas del tipo
$$X'AX$$
con $X \sim N_{p}(\mu, \Sigma)$ ($\Sigma > 0$) y $A$ una matriz (cte.) ($p \times p$)-dimensional simétrica.

\begin{itemize}
\item De forma preliminar, se puede hacer un estudio directo de los momentos de primer y segundo orden:

Sea $X \sim N_{p}(\mu,\Sigma)$ ($\Sigma > 0$) y $A$ una matriz (cte.) ($p \times p$)-dimensional simétrica. Entonces,
\begin{itemize}
\item[(a)] $E[X'AX] = tr(A\Sigma) + \mu'A\mu \hspace{1cm} (3)$.
\item[(b)] $Var(X'AX) = 2tr((A\Sigma)^{2}) + 4 \sigma'A\Sigma A\mu$.
\end{itemize}
(Observación: El resultado (a) no requiere la hipótesis de normalidad, solo la existencia de la esperanza)

\item (3) $(X \sim N_{p}(\mu, \Sigma) \Rightarrow E[X'AX] = tr(A\Sigma) + \mu'A\mu$.
$$E[X'AX] = E[tr(X'AX)] = E[tr(AXX')] = tr(E[AXX']) = tr(A E[XX'])$$
donde hemos usado que $X'AX$ es de dim 1 y que $tr(CB) = tr(BC)$.

Usaremos ahora que:
$$E[XX'] = \Sigma + \mu\mu'$$
ya que:
$$\Sigma = E[(X-\mu)(X-\mu)'] = E[XX'] - E[X\mu'] - E[\mu X'] + E[\mu\mu'] =$$
$$= E[XX'] - \mu'E[X] - \mu E[X'] + \mu\mu' = E[XX'] - \mu'\mu - \mu\mu' + \mu\mu' = E[XX'] - \mu\mu'$$

Entonces:
$$E[X'AX] = tr(A E[XX']) = tr(A(\Sigma+\mu\mu')) = tr(A\Sigma + A\mu\mu') =$$
$$= tr(A\Sigma) + tr(A\mu\mu') = tr(A\Sigma) + tr(\mu'A\mu) = tr(A\Sigma) + \mu'A\mu$$

\item Resultado 1: \\
Sea $X \sim N_{p}(\mu, \Sigma)$ ($\Sigma > 0$). Entonces:
\begin{itemize}
\item[1.] $(X - \mu)' \Sigma^{-1} (X - \mu) \sim \chi_{p}^{2} \hspace{1cm} (4)$.
\item[2.] $X'\Sigma^{-1}X \sim \chi_{p}^{2}(\delta)$, con $\delta = \mu'\Sigma^{-1}\mu \hspace{1cm} (5)$.
\end{itemize}

\item (4) $X \sim N_{p}(\mu, \Sigma), \Sigma > 0, \Sigma = CC'$. \\
Hacemos el cambio $Y = C^{-1}(X - \mu) \sim N_{p}(0, I_{p})$.
$$(X-\mu)'\Sigma^{-1}(X-\mu) = (X-\mu)(CC')^{-1}(X-\mu) = (X-\mu)(C^{-1})'C^{-1}(X-\mu) = Y'Y$$
Como $Y \sim N_{p}(0, I_{p}) \Rightarrow YY' \sim \chi^{2}_{p}$.

\item (5) $V = C^{-1}X \sim N_{p}(C^{-1}\mu, C^{-1}\Sigma(C^{-1})') \equiv N_{p}(C^{-1}\mu, C^{-1}CC'(C^{-1})') \equiv N_{p}(C^{-1}\mu, I_{p})$.
$$X'\Sigma^{-1}X = X'(CC')^{-1}X = [...] = V'V$$
Como $V \sim N_{p}(C^{-1}\mu, I_{p}) \Rightarrow V'V \sim \chi_{p}^{2}((C^{-1}\mu)'(C^{-1}\mu)) \equiv \chi_{p}^{2}(\mu\Sigma^{-1}\mu)$, donde hemos usado:
$$(C^{-1}\mu)'(C^{-1}\mu) = \mu'(C^{-1})'C^{-1}\mu = \mu'(CC')^{-1}\mu = \mu'\Sigma^{-1}\mu$$

\item Resultado 2: \\
Sea $X \sim N_{p}(\mu, I_{p})$ y sea $B$ una matriz (cte.) ($p \times p$)-dimensional simétrica. Entonces, $X'BX$ tiene una distribución $\chi^{2}$ no centrada si y sólo si $B$ es idempotente (i.e. $B^{2}=B$), en cuyo caso los grados de libertad y el parámetro de no centralidad son, respectivamente, $k = rango(B) = tr(B)$ y $\delta = \mu'B\mu$ \hspace{1cm} (6).

\item (6) \begin{itemize}
\item[$\Rightarrow$] Supongamos $X'BX \sim \chi^{2}_{k}(\delta)$. Sea $r = rango(B)$. \\
Por ser $B$ simétrica podemos escribir la diagonalización
$$H'BH = D \hspace{1cm} (\iff B = HDH')$$
siendo $H$ matriz $p \times p$ ortogonal y $D$ una matriz $p \times p$ diagonal.
$$D = \begin{pmatrix}
\lambda_{1} & \dots & 0 & 0 & \dots & 0 \\
\vdots & \ddots & \vdots & \vdots & \dots & \vdots \\
0 & \dots & \lambda_{r} & 0 & \dots & 0 \\
0 & \dots & 0 & 0 & \dots & 0 \\
\vdots & \dots & \vdots & \vdots & \dots & \vdots \\
0 & \dots & 0 & 0 & \dots & 0
\end{pmatrix}$$
siendo $\lambda_{1}, \dots, \lambda_{r} \neq 0$. \\
Transformamos $X$ mediante $H$, $V = H'X$, siendo entonces
$$V \sim N_{p} (H'\mu, H' \Sigma H) \equiv N_{p}(H'\mu, I_{p}) \equiv N_{p}(\mathcal{V}, I_{p})$$
Podemos escribir:
$$X'BX = X'HDH'X = V'DV = \sum_{j=1}^{r} \lambda_{j} V_{j}^{2}$$
Cada $V_{j}^{2}$ tiene distribución $V_{j}^{2} \sim \chi_{1}^{2}$ ($\mathcal{V}_{j}^{2}$), siendo $V_{1}^{2}, \dots, V_{r}^{2}$ independientes.

La función característica de $X'BX$ se obtienen como:
$$\Phi_{X'BX}(t) = \Phi_{\sum_{j=1}^{r}\lambda_{j}V_{j}^{2}}(t) = \prod_{j=1}^{r} \Phi_{\lambda_{j}V_{j}^{2}}(t) = \prod_{j=1}^{r} \Phi_{V_{j}^{2}}(\lambda_{j}t)$$
Cada función característica individual $\Phi_{V_{j}^{2}}(\cdot)$ es de la forma:
$$\Phi_{\chi_{1}^{2}(\delta)}(s) = (1 - 2is)^{-\frac{1}{2}} exp\{\frac{is\delta}{1-2is}\}$$
con $\delta = \mathcal{V}_{j}^{2}$. \\
Luego, conjuntamente, se tiene:
$$\Phi_{X'BX}(t) = \prod_{j=1}^{r} (1-2i\lambda_{j}t)^{-\frac{1}{2}} exp\{\frac{i\lambda_{j}t\mathcal{V}_{j}^{2}}{1-2i\lambda_{j}t}\} =$$
$$= (\prod_{j=1}^{r}(1-2i\lambda_{j}t)^{-\frac{1}{2}}) exp\{it \sum_{j=1}^{r} \frac{\lambda_{j}\mathcal{V}_{j}^{2}}{1-2i\lambda_{j}t}\}$$
Por otra parte, al ser $X'BX \sim \chi_{k}^{2}(\delta)$, se tiene que:
$$\Phi_{X'BX}(t) = (1-2it)^{-\frac{k}{2}} exp\{\frac{it\delta}{1-2it}\}$$ 

Igualando las dos expresiones (para todo $t \in \mathbb{R}$), han de ser:
\begin{itemize}
\item $\lambda_{j} = 1 \hspace{1cm} j=1, \dots, r$.
\item $\delta = \sum_{j=1}^{r} \lambda_{j}\mathcal{V}_{j}^{2} = \sum_{j=1}^{r} \mathcal{V}_{j}$.
\item $r = k$.
\end{itemize}
En consecuencia
$$H'BH = D = \begin{pmatrix}
1 & \dots & 0 & 0 & \dots & 0 \\
\vdots & \ddots & \vdots & \vdots & \dots & \vdots \\
0 & \dots & 1 & 0 & \dots & 0 \\
0 & \dots & 0 & 0 & \dots & 0 \\
\vdots & \dots & \vdots & \vdots & \dots & \vdots \\
0 & \dots & 0 & 0 & \dots & 0
\end{pmatrix} = \begin{pmatrix}
I_{k} & 0 \\
0 & 0
\end{pmatrix}$$
que es una matriz idempotente, es decir, $(H'BH)(H'BH) = H'BH$. Entonces:
$$(H'BH)(H'BH) = H'B^{2}H = H'BH \Rightarrow B^{2} = B$$
por lo que $B$ es idempotente. Además:
$$\delta = \sum_{j=1}^{r} \mathcal{V}_{j}^{2} = \mathcal{V}'D\mathcal{V} = (H'\mu)'D(H'\mu) = \mu'HDH'\mu = \mu'B\mu$$

\item[$\Leftarrow$] Supongamos $B$ (simétrica) es idempotente y de rango $k$. Podemos escribir:
$$H'BH = \begin{pmatrix}
I_{k} & 0 \\
0 & 0
\end{pmatrix}$$
para alguna matriz ortogonal $H$.

Definimos el vector:
$$V = H'X \sim N_{p}(H'\mu, H'I_{p}H) \equiv N_{p}(H'\mu, I_{p})$$
Por tanto, dado que podemos escribir:
$$X'BX = X'H\begin{pmatrix}
I_{k} & 0 \\
0 & 0
\end{pmatrix}H'X = V'\begin{pmatrix}
I_{k} & 0 \\
0 & 0
\end{pmatrix} V = \sum_{i=1}^{k} V_{j}^{2}$$
y, puesto que las variables $V_{1}, \dots, V_{k}$ son independientes, se tiene que:
$$X'BX \sim \chi_{k}^{2}(\delta)$$
con parámetro de no centralidad:
$$\delta = E[V'] \begin{pmatrix}
I_{k} & 0 \\
0 & 0
\end{pmatrix} E[V]$$
donde $E[V] = H'\mu$, así que tenemos:
$$\delta = (H'\mu)' \begin{pmatrix}
I_{k} & 0 \\
0 & 0
\end{pmatrix} (H'\mu) = \mu'H \begin{pmatrix}
I_{k} & 0 \\
0 & 0
\end{pmatrix} H'\mu = \mu' B \mu$$
\end{itemize}

\item Resultados auxiliares:
\begin{itemize}
\item Una matriz simétrica es idempotente si y sólo si sus autovalores son 0 o 1.

A parir de esto, si $B$ simétrica idempotente, entonces:
$$rango(B) = tr(B)$$
\item (Lema 10.1.2 Harville, p. 134) Sea $A$ una matriz cuadrada. Entonces:
\begin{enumerate}
\item $A'$ es idempotente $\iff A$ es idempotente.
\item $I - A$ es idempotente $\iff A$ es idempotente.
\end{enumerate}
\end{itemize}

\item Resultado 3: \\
Sea $X \sim N_{p}(\mu, \Sigma)$ ($\Sigma > 0$). Supongamos el particionamiento
$$X = \begin{pmatrix} X_{(1)} \\ X_{(2)} \end{pmatrix}, \hspace{1cm} \mu = \begin{pmatrix} \mu_{(1)} \\ \mu_{(2)} \end{pmatrix}, \hspace{1cm} \Sigma = \begin{pmatrix}
\Sigma_{(11)} & \Sigma_{(12)} \\
\Sigma_{(21)} & \Sigma_{(22)}
\end{pmatrix}$$
con $X_{(1)}$ y $\mu_{(1)}$ subvectores $q$-dimensionales y $\Sigma_{(11)}$ submatriz ($q \times q$)-dimensional. Entonces,
$$Q := (X-\mu)' \Sigma^{-1} (X-\mu) - (X_{(1)}-\mu_{(1)})' \Sigma_{(11)}^{-1} (X_{(1)}-\mu_{(1)}) \sim \chi_{p-1}^{2} \hspace{1cm} (7)$$

\item (7) Como $X \sim N_{p}(\mu, \Sigma)$ con $\Sigma > 0$, se puede descomponer $\Sigma = CC'$. Ponemos $C = \begin{pmatrix} C_{(1)} \\ C_{(2)} \end{pmatrix}$, tal que $\Sigma = CC' = \begin{pmatrix}
C_{(1)}C_{(1)}' & C_{(1)}C_{(2)}' \\
C_{(2)}C_{(1)}' & C_{(2)}C_{(2)}'
\end{pmatrix}$.

Normalizamos, definiendo $Y = C^{-1}(X-\mu) \Rightarrow X = CY + \mu$, $Y \sim N_{p}(0, I_{p})$, luego $X_{(1)} = C_{(1)}Y + \mu_{(1)}$.
$$Q = (CY')\Sigma^{-1}(CY) - Y'C_{(1)}'(C_{(1)}C_{(1)}')^{-1}C_{(1)}Y \Rightarrow$$
$$\Rightarrow Q = Y'C'(CC')^{-1}CY - [...] = Y'[I_{p}-C_{(1)}'(C_{(1)}C_{(1)}')^{-1}C_{(1)}]Y$$

$$[C_{(1)}'(C_{(1)}C_{(1)}')^{-1}C_{(1)}]^{2} = C_{(1)}'(C_{(1)}C_{(1)}')^{-1}C_{(1)}C_{(1)}'(C_{(1)}C_{(1)}')^{-1}C_{(1)} = C_{(1)}' I_{q} (C_{(1)}C_{(1)}')^{-1}C_{(1)}$$

$\exists B$ tal que $B^{2} = B$ y $Q = YBY \sim \chi_{k}^{2}(\delta)$, $\delta = \mu'B\mu = 0'B0 = 0$, $k = rango(B) = tr(B) = tr(I_{p}) - tr(C_{(1)}'(C_{(1)}C_{(1)}')^{-1}C_{(1)}) = p - tr(\underset{=I_{q}}{C_{(1)}C_{(1)}'(C_{(1)}C_{(1)})^{-1}}) = p - q$.

\item Resultado 4: \\
Sea $X \sim N_{p}(\mu, \Sigma)$ ($\Sigma > 0$) y sea $B$ una matriz (cte.) ($p \times p$)-dimensional simétrica. Entonces, $X'BX$ tiene una distribución $\chi_{k}^{2}(\delta)$, con $k = rango(B)$ y $\delta = \mu'B\mu$, si y sólo si $B\Sigma$ es idempotente (i.e. $(B\Sigma)^{2} = B\Sigma$; equivalentemente, en este caso, $B\Sigma B = B$) \hspace{1cm} (8).

\item (8) Sea $\Sigma = CC'$, con $C_{p \times p}$ no singular. Sea $Y = C^{-1}X$. \\
Entonces $Y \sim N_{p}(C^{-1}\mu, C^{-1}\Sigma(C^{-1})') \equiv N_{p}(C^{-1}\mu, I_{p})$.

Entonces $X'BX = (CY)'B(CY) = Y'C'BCY$. Por el Resultado 2 tenemos:
$$X'BX \sim \chi_{k}^{2}(\delta) \iff C'B \text{ idempotente}$$

Hay que probar entonces que $C'BC$ es idempotente $\iff B\Sigma$ es idempotente.
$$B\Sigma \text{ idempotente} \iff (B\Sigma)(B\Sigma) = B\Sigma \iff (BCC')(BCC') = BCC' \iff$$
$$\iff BCC'BCC'(C')^{-1} = BCC'(C')^{-1} \iff BCC'BC = BC \iff$$
$$\iff (C')BCC'BC = (C')BC \iff (C'BC)'(C'BC) = C'BC \iff C'BC \text{ idempotente}$$

Además, por el resultado 2: 
$$k = rango (C'BC) = tr(C'BC) = tr(BCC') = tr(B\Sigma) = rango(B\Sigma) = rango(B)$$
$$\delta = (C^{-1}\mu)'(C'BC)(C^{-1}\mu) = \mu'(C')^{-1}C'BCC^{-1}\mu = \mu'B\mu$$
\end{itemize}

\subsubsection{APÉNDICE: Funciones hipergeométricas generalizadas}
\begin{itemize}
\item Definición: Se denomina función hipergeométrica generalizada de órdenes $p$ y $a$ a
$${}_{p}F_{q}(a_{1}, \dots, a_{p}; b_{1}, \dots, b_{q}; z) := \sum_{k=0}^{\infty} \frac{(a_{1})_{k} \dots (a_{p})_{k}}{(b_{1})_{k} \dots (b_{p})_{k}} \frac{z^{k}}{k!}$$
donde $(a)_{k} = a(a+1) \dots (a+k-1)$, siendo $a_{1}, \dots, a_{p}, b_{1}, \dots, b_{q}$ parámetros (posiblemente complejos) y $z \in \mathbb{C}$ el argumento de la función.

\item Algunas observaciones:
\begin{itemize}
\item Ningún parámetro $b_{j}$ puede ser 0 o un entero negativo (en este caso, uno de los denominadores de la serie sería 0 a partir de un cierto $k$).

\item Si algún parámetro en el numerador es 0 o un entero negativo, los términos de la serie se anulan a partir de un cierto $k$ y queda un polinomio en $z$.

\item La serie
\begin{itemize}
\item converge para todo $z$ finito si $p \leq q$.
\item converge para $|z| < 1$ y diverge para $|z|>1$ si $p=q+1$.
\item diverge para todo $z \neq 0$ si $p > q + 1$.
\end{itemize}

\item El término `generalizada' se refiere a que ${}_{p}F_{q}$ es una generalización de la función hipergeométrica cñásica (o gaussiana), $_{2}F_{1}$.

\item $_{1}F_{1}$ se denomina función hipergeométrica confluente.

\item $_{0}F_{1}$, definida como
$${}_{0}F_{1}(;b;z) := \lim\limits_{a \to \infty} {}_{1}F_{1}(a;b;\frac{z}{a})$$
se denomina función hipergeométrica confluente límite.
\end{itemize}
\end{itemize}

\subsection{Temario examen}

\begin{itemize}
\item Fundamentación probabilística $\to$ No entra.

\item Aspectos generales sobre vectores aleatorios:
\begin{itemize}
\item Propiedad 3 de Esperanza y covarianzas si entra.
\item Normalización: $Z \sim (0_{p}, I_{p \times p})$ si entra.
\item Propiedades $[...] = k$ si entra.
\item Semidefinida positiva: Observación 2 si entra.
\item Teorema $\alpha'X$ de función característica si entra.
\item Caso lineal del teorema de cambio de variable si entra.
\end{itemize}

\item Caso $\Sigma > 0$ entra casi entero:
\begin{itemize}
\item Algunas propiedades: Caso (3) sobre el condicionamiento no entra.
\item Caracterización en términos de la densidad esférica, el resultado sobre $U$ y $Z$ independientes no entra.
\end{itemize}

\item Caso $\Sigma \geq 0$ entra casi entero:
\begin{itemize}
\item Extensión de la caracterización C-I es importante que sepamos la diferencia con el caso no general.
\item Discusión en relación con la posible no existencia de densidad en todo $\mathbb{R}^{p}$ si entra.
\item Resultado sobre normalidad de combinaciones lineales de vectores aleatorios con DNM no entra.
\item El Complemento del problema de Lévy-Crámer, la generalización asumiendo que es cierto para 1 si entra.
\end{itemize}

\item Distribuciones esféricas y elípticas.
\begin{itemize}
\item Los 3 ejemplos entran.
\item El resultado de independencia no entra.
\item Distribuciones elípticas entran sin demostraciones.
\item El ejemplo de demostraciones elípticas de $V = \Sigma$ si entra.
\end{itemize}

\item Formas cuadráticas basadas en vectores aleatorios con DNM.
\begin{itemize}
\item No hay que aprenderse las funciones de densidad ni características.
\item Saber calcular los momentos a partir de la característica si entra.
\item Resultado sobre $\chi^{2}$ si entra.
\item Si hay que saber como se definen las distribuciones.
\item Formas cuadráticas $X'AX$ entra el resultado a).
\item La prueba del resultado 2 no entra.
\item El apéndice no entra.
\end{itemize}

\item Ejercicios: no va a haber ejercicios de tipo numérico, como los 7, 8 y 9 de la relación.\item Los ejercicios extra son interesantes.

2 preguntas de teoría y 2 problemas. Entre las preguntas de teoría podemos tener varias opciones entre las cuales elegimos 2.
\end{itemize}

\newpage

\section{Inferencia en la Distribución Normal Multivariante}
\subsection{Introducción}
\subsubsection{``Muestra aleatoria simple'' vectorial}
\begin{itemize}
\item Sea $X = (X_{1}, \dots, X_{p})'$ un vector aleatorio con DNM no singular,
$$X \sim N_{p} (\mu, \Sigma) \hspace{1cm} (\Sigma > 0)$$
(Por ejemplo, consideramos que las variables representan, en una situación práctica, $p$ magnitudes relativas a ``características'' de los ``individuos'' de una ``población'', con esa distribución de probabilidad conjunta).

\item Se trata de realizar la inferencia (bajo distintas formas) sobre la distribución, supuestos desconocidos total o parcialmente a sus parámetros, a partir de una \emph{muestra aleatoria simple} obtenida de ésta (i.e., un conjunto de ``observaciones'' dadas en términos de \emph{variables aleatorias independientes e indénticamente distribuidas} con igual distribución de $X$), que denotaremos
$$\{X_{\alpha}: \alpha = 1, \dots, N\} \hspace{1cm} (N = \text{ ``tamaño muestral''})$$
siendo $X_{\alpha i}$ la componente $i-$ésima (correspondiente a la variable $X_{i}$) de la observación (vectorial) $\alpha-$ésima, $X_{\alpha}$.

\item Se utilizará la misma notación pero en minúsculas cuando las variables de una muestra aleatoria simple tomen valores numéricos concretos, es decir, se tenga una realización de ésta:
$$\{x_{\alpha}: \alpha = 1, \dots, N\} \hspace{1cm} (N = \text{ ``tamaño muestral''})$$

\item A efectos algebraicos y computacionales, es frecuente disponer las variables componentes de la muestra en la forma de ``matriz de datos'' (aleatoria),
$$X = \begin{pmatrix}
X_{1}' \\ \vdots \\ X_{\alpha}' \\ \vdots \\ X_{p}'
\end{pmatrix} = \begin{pmatrix}
X_{11} & \dots & X_{1i} & \dots & X_{1p} \\
\vdots & \ddots & \vdots & \ddots & \vdots \\
X_{\alpha 1} & \dots & X_{\alpha i} & \dots & X_{\alpha p} \\
\vdots & \ddots & \vdots & \ddots & \vdots \\
X_{N1} & \dots & X_{Ni} & \dots & X_{Np}
\end{pmatrix}$$
(y análogamente, usando minúsculas, en el caso de las correspondientes realizaciones de la muestra).
\end{itemize}

\subsubsection{Medidas muestrales de primer y segundo orden}
\begin{itemize}
\item En primer lugar, como paso previo a otros aspectos de la inferencia, se abordará en el apartado siguiente la \emph{estimación puntual} de los parámetros $\mu$ y $\Sigma$. Se adoptará el criterio basado en la \emph{maximización de la función de verosimilitud}.

Como se verá, los \emph{estimadores máximo-verosímiles} de $\mu$ y $\Sigma$ se expresarán en términos de medidas descriptivas de la muestra dadas por el \emph{vector de medias muestral} y la \emph{matriz de dispersiones muestral} (respecto del vector de medias muestral).

\item VECTOR DE MEDIAS MUESTRALES (COMPLETAR).

\item MATRIZ DE DISPERSIONES MUESTRAL ($p \times p$)
$$A := \sum_{\alpha=1}^{N} (X_{\alpha} - \bar{X})(X_{\alpha} - \bar{X})' =$$
\begin{small}
$$= \begin{pmatrix}
\sum_{\alpha=1}^{N} (X_{\alpha 1} - X_{1})^{2} & \sum_{\alpha = 1}^{N} (X_{\alpha 1} - \bar{X})(X_{\alpha 2} - \bar{X}_{2}) & \dots & \sum_{\alpha = 1}^{N} (X_{\alpha 1} - \bar{X}_{1})(X_{\alpha p} - \bar{X}_{p}) \\
\sum_{\alpha=1}^{N} (X_{\alpha 2} - \bar{X}_{2})(X_{\alpha 1} - \bar{X}_{1}) & \sum_{\alpha=1}^{N}(X_{\alpha 2} - \bar{X}_{2})^{2} & \dots & \sum_{\alpha = 1}^{N} (X_{\alpha 2} - \bar{X}_{2})(X_{\alpha p} - \bar{X}_{p}) \\
\vdots & \vdots & \ddots & \vdots \\
\sum_{\alpha = 1}^{N}(X_{\alpha p} - \bar{X}_{p})(X_{\alpha 1} - \bar{X}_{1}) & \sum_{\alpha = 1}^{N} (X_{\alpha p} - \bar{X}_{p})(X_{\alpha 2} - \bar{X}_{2}) & \dots & \sum_{\alpha=1}^{N} (X_{\alpha p} - \bar{X}_{p})^{2}
\end{pmatrix}$$
\end{small}
A partir de la matriz de dispersiones muestral, se definen:
\begin{itemize}
\item MATRIZ DE COVARIANZAS MUESTRAL: $S_{N} = \frac{A}{N} = (s_{ij})_{i,j = 1,\dots,p}$.
\item MATRIZ DE CUASI-COVARIANZAS MUESTRAL: $S_{N-1} = \frac{A}{N-1}$.
\item MATRIZ DE CORRELACIONES MUESTRAL: $R = (\frac{s_{ij}}{s_{ii}^{\frac{1}{2}}s_{jj}^{\frac{1}{2}}})_{i,j = 1,\dots,p}$ \\
(puede escribirse como $R = D^{-\frac{1}{2}}S_{N}D^{-\frac{1}{2}}$ con $D = \operatorname{dia}(s_{11}, \dots, s_{pp})$).
\end{itemize}
\end{itemize}

\subsubsection{Fórmula de momentos multivariante muestral}
\begin{itemize}
\item LEMA: Sea $\{X_{\alpha}: \alpha=1,\dots,N\}$ una muestra aleatoria de una distribución $p$-dimensional, y sea $\bar{X}$ el vector de medias muestral. Entonces, para cualquier vector $b \in \mathbb{R}^{p}$, se verifica que
$$\sum_{\alpha=1}^{N}(X_{\alpha}-b)(X_{\alpha}-b)' = \sum_{\alpha=1}^{N}(X_{\alpha}-\bar{X})(X_{\alpha}-\bar{X})'+N(\bar{X}-b)(\bar{X}-b)'=$$
$$=A+N(\bar{X}-b)(\bar{X}-b)' \hspace{1cm} (1)$$
En particular, se tiene:
\begin{itemize}
\item Con $b = \mu$ (vector de medias poblacional, si existe)
$$\sum_{\alpha=1}^{N}(X_{\alpha}-\mu)(X_{\alpha}-\mu)'=A+N(\bar{X}-\mu)(\bar{X}-\mu)'$$
\item Con $b = 0$
$$\sum_{\alpha=1}^{N} X_{\alpha}X_{\alpha}' = A + N\bar{X}\bar{X}'$$
(`fórmula de momentos multivariante muestral')
\end{itemize}

\item (1) 
$$\sum_{\alpha=1}^{N} (X_{\alpha} - b) (X_{\alpha} - b)' = \sum_{\alpha=1}^{N} [(X_{\alpha} - \bar{X}) + (\bar{X} - b)][(X_{\alpha} - \bar{X}) + (\bar{X} - b)]' =$$
$$= \sum_{\alpha=1}^{n} [(X_{\alpha} - \bar{X}) + (\bar{X} - b)][(X_{\alpha} - \bar{X})' + (\bar{X} - b)'] =$$
\begin{small}
$$= \underset{A}{\sum_{\alpha=1}^{N} (X_{\alpha} - \bar{X})(X_{\alpha} - \bar{X})'} + \underset{B}{\sum_{\alpha=1}^{N}(\bar{X} - b)(\bar{X} - b)'} + \underset{C}{\sum_{\alpha=1}^{N} (X_{\alpha} - \bar{X})(\bar{X} - b)'} + \underset{D}{\sum_{\alpha=1}^{N} (\bar{X} - b)(X_{\alpha} - \bar{X})'} = \star$$
\end{small}

$$B \Rightarrow \sum_{\alpha=1}^{N}(\bar{X} - b)(\bar{X} - b)' = N (\bar{X} - b)(\bar{X} - b)'$$
$$C \Rightarrow \sum_{\alpha=1}^{N} (X_{\alpha} - \bar{X})(\bar{X} - b)' = \sum_{\alpha=1}^{N}X_{\alpha}\bar{X}' - \sum_{\alpha=1}^{N}X_{\alpha}b' - \sum_{\alpha=1}^{N}\bar{X}\bar{X}' + \sum_{\alpha=1}^{N}\bar{X}b' =$$
$$= N\bar{X}\bar{X}' - N\bar{X}b' - N\bar{X}\bar{X}' + N\bar{X}b' = 0$$
$$D \Rightarrow \sum_{\alpha=1}^{N} (\bar{X} - b)(X_{\alpha} - \bar{X})' = \sum_{\alpha=1}^{N} (X_{\alpha} - \bar{X})(\bar{X} - b)' = 0$$

$$\star = A + N(\bar{X}-b)(\bar{X}-b)'$$
\end{itemize}

\subsection{Inferencia en la DNM}
\subsubsection{Función de verosimilitud}
\begin{itemize}
\item Sea $X \sim N_{p}(\mu,\Sigma)$ ($\Sigma>0$), y sea $\{X_{\alpha}: \alpha=1,\dots,N\}$ una muestra aleatoria simple de dicha distribución.

Para una realización dada de la muestra $\{x_{\alpha}: \alpha=1,\dots,N\}$, la \emph{función de verosimilitud} se expresa como la función (de argumentos $\mu$ y $\Sigma$).
$$L(\mu,\Sigma;x_{1},\dots,x_{N}) := f_{\mu,\Sigma}(x_{1},\dots,x_{N}) = \prod_{\alpha=1}^{N}f_{\mu,\Sigma}(x_{\alpha}) =$$
$$= \prod_{\alpha=1}^{N} \frac{1}{(2\pi)^{\frac{p}{2}}|\Sigma|^{\frac{1}{2}}} \operatorname{exp}\{-\frac{1}{2}(x_{\alpha}-\mu)'\Sigma^{-1}(x_{\alpha}-\mu)\} =$$
$$= \frac{1}{(2\pi)^{\frac{pN}{2}}|\Sigma|^{\frac{N}{2}}} \operatorname{exp}\{-\frac{1}{2}\sum_{\alpha=1}^{N}(x_{\alpha}-\mu)'\Sigma^{-1}(x_{\alpha}-\mu)\}$$
De forma abreviada, se denotará simplemente $L(\mu, \Sigma)$, sobreentendiéndose implícita la realización muestral de referencia.

\item Mediante operaciones algebraicas, y usando la fórmula de momentos multivariante generalizada (caso $b=\mu$), se comprueba que la función de verosimilitud puede expresarse de la forma
$$L(\mu,\Sigma) = \frac{1}{(2\pi)^{\frac{pN}{2}}|\Sigma|^{\frac{N}{2}}} \operatorname{exp}\{-\frac{1}{2} \operatorname{tr}(\Sigma^{-1}\bar{A}) - \frac{N}{2} (\bar{x}-\mu)'\Sigma^{-1}(\bar{x}-\mu)\}$$
con $\bar{A} = \sum_{\alpha=1}^{N}(x_{\alpha}-\bar{x})(x_{\alpha}-\bar{x})'$ (es decir, la matriz de dispersiones muestral evaluada en la realización dada de la muestra, $\{x_{\alpha}: \alpha=1, \dots, N\}$) (1).

Más convenientemente, a efectos de maximización, se suele usar la transformación
$$\ln(L(\mu,\Sigma)) = -\frac{pN}{2}\ln(2\pi) -\frac{N}{2}\ln(|\Sigma|) -\frac{1}{2}\operatorname{tr}(\Sigma^{-1}\bar{A}) -\frac{N}{2}(\bar{x}-\mu)'\Sigma^{-1}(\bar{x}-\mu)$$
(Es habitual proceder, equivalentemente, a la minimización de la función $-\ln(L(\mu,\Sigma))$, no negativa).

\item (1) $$L(\mu, \Sigma; x_{1}, \dots, x_{n}) = \frac{1}{(2\pi)^{\frac{pN}{2}}|\Sigma|^{\frac{N}{2}}} \operatorname{exp}\{-\frac{1}{2} \sum_{\alpha=1}^{N}(x_{\alpha}-\mu)'\Sigma^{-1}(x_{\alpha}-\mu)\}$$

$$\sum_{\alpha=1}^{N} (x_{\alpha}-\mu)' \Sigma^{-1} (x_{\alpha}-\mu) = \sum_{\alpha=1}^{N}[(x_{\alpha}-\bar{x}) + (\bar{x}-\mu)]' \Sigma^{-1} [(x_{\alpha}-\bar{x}) + (\bar{x} - \mu)] =$$
$$= \sum_{\alpha=1}^{N} [(x_{\alpha} - \bar{x})' + (\bar{x}-\mu)'] \Sigma^{-1} [(x_{\alpha} - \bar{x}) + (\bar{x} - \mu)] =$$
$$= \sum_{\alpha=1}^{N} (x_{\alpha} - \bar{x})' \Sigma^{-1} (x_{\alpha} - \mu) + \sum_{\alpha=1}^{N}(x_{\alpha}-\bar{x})' \Sigma^{-1}(\bar{x}-\mu) +$$
$$+ \sum_{\alpha=1}^{N} (\bar{x}-\mu) \Sigma^{-1} (x_{\alpha} - \bar{x}) + \sum_{\alpha=1}^{N} (\bar{x}-\mu)' \Sigma^{-1} (\bar{x}-\mu) =$$
$$= \sum_{\alpha=1}^{N} \operatorname{traza}(x_{\alpha} - \bar{x})' \Sigma^{-1} (x_{\alpha} - \mu) + N(\bar{x}-\mu)' \Sigma^{-1} (\bar{x}-\mu) =$$
$$= \sum_{\alpha=1}^{N} \operatorname{traza} [\Sigma^{-1}(x_{\alpha} - \mu)(x_{\alpha} - \bar{x})'] + N(\bar{x} - \mu)' \Sigma^{-1} (\bar{x}-\mu) =$$
$$= \operatorname{traza}[\sum_{\alpha=1}^{N} \Sigma^{-1}(x_{\alpha} - \mu) (x_{\alpha} - \bar{x}'] + N(\bar{x} - \mu)' \Sigma^{-1} (\bar{x}-\mu) =$$
$$= \operatorname{traza}[\Sigma^{-1} \sum_{\alpha=1}^{N} (x_{\alpha} - \mu)(x_{\alpha} - \bar{x})'] + N(\bar{x}-\mu)' \Sigma^{-1} (\bar{x}-\mu) =$$
$$= \operatorname{traza}(\Sigma^{-1}\bar{A}) + N(\bar{x}-\mu)' \Sigma^{-1} (\bar{x}-\mu)$$

$$L(\mu, \Sigma; x_{1}, \dots, x_{n}) = \frac{1}{(2\pi)^{\frac{pN}{2}}|\Sigma|^{\frac{N}{2}}} \operatorname{exp}\{-\frac{1}{2} \operatorname{traza}(\Sigma^{-1}\bar{A}) - \frac{N}{2}(\bar{x}-\mu)' \Sigma^{-1} (\bar{x}-\mu)\}$$
 
\end{itemize}

\subsubsection{Estimadores máximo-verosímiles de \texorpdfstring{$\mu$}) y \texorpdfstring{$\Sigma$})}
\begin{itemize}
\item RESULTADO: Sea $X \sim N_{p}(\mu, \Sigma)$ ($\Sigma > 0$), y sea $\{X_{\alpha}: \alpha=1,\dots,N\}$ una muestra aleatoria simple de dicha distribución. Entonces, los \emph{estimadores máximo-verosímiles} (EMV) de $\mu$ y $\Sigma$ son, respectivamente,
$$\bar{X} \hspace{1cm} \text{y} \hspace{1cm} \frac{A}{N} = S_{N}$$
este último bajo la condición de ser $A$ definida positiva $(*)$.

[$(*)$ Este aspecto se discutirá más adelante, tras la demostración, en relación con el `Teorema de Dykstra']

\item El problema de optimización se podría abordar de forma directa mediante derivación matricial. No obstante, es interesante probar el resultado por el procedimiento que se expondrá a continuación.

\item Para la demostración, se utilizará el siguiente resultado auxiliar en la parte relativa al estimador del parámetro $\Sigma$: \\
LEMA (Watson): Sea
$$f(G) = -N\ln(|G|) - \operatorname{tr}(G^{-1}D)$$
con argumento de $G$ una matriz simétrica definida positiva, y siendo $D$ una matriz simétrica definida positiva dada, ambas $p \times p$. Entonces, existe (y es único) el máximo de $f$ respecto a $G$, y se alcanza en $G = \frac{1}{N}D$, siendo el valor máximo alcanzado
$$f(\frac{1}{N}D) = pN\ln(N) - N\ln(|D|) - pN \hspace{1cm} (2)$$

\item (2) (No examen) Por ser $D$ simétrica y definida positiva, podemos representar $D = EE'$ con $E$ matriz no singular. Sea $H = E'G^{-1}E \Rightarrow G^{-1} = (E')^{-1}HE^{-1} \Rightarrow G = EH^{-1}E'$. Se tiene que
$$|G| = |EH^{-1}E'| = |E||H^{1}||E'| = |H^{-1}||EE'| = \frac{|D|}{|H|}$$
$$\operatorname{traza}(G^{-1}D) = \operatorname{traza}(G^{-1}EE') = \operatorname{traza}(E'G^{-1}E) = \operatorname{traza}(H)$$
Podemos escribir:
$$f(G) = g(H) = -N\ln(|D|) + N\ln(|H|) - \operatorname{traza}(H)$$
Con $H$ en el espacio de matrices simétricas y definidas positivas ($p \times p$).

Consideramos la descomposición de Cholesky de $H$:
$$H = T'T$$
con $T$ matriz triangular superior con elementos diagonales estrictamente positivos. \\
Escribimos
$$f(G) = g(H) = h(T) = -N\ln(|D|) + N\ln(|T|^{2}) - \operatorname{traza}(T'T)$$
Con $T$ en el espacio de matrices triangulares superiores con elementos diagonales estrictamente positivos.
$$|T|^{2} = \sum_{i=1}^{p} t_{ii}^{2} \Rightarrow \ln(|T|^{2}) = \sum_{i=1}^{p} \ln(t_{ii}^{2}) \Rightarrow N\ln(|T|^{2}) = N\sum_{i=1}^{p}\ln(t_{ii}^{2}) = \sum_{i=1}^{p}N\ln(t_{ii}^{2})$$
$$\operatorname{traza}(T'T) = \sum_{i=1}^{p} t_{ii}^{2} + \sum_{\underset{i<j}{i,j=1}}^{p}t_{ij}^{2}$$
\begin{small}
$$T'T = \begin{pmatrix}
t_{11} & 0 & \dots & 0 \\
t_{12} & t_{22} & \dots & 0 \\
\vdots & \vdots & \ddots & \vdots \\
t_{1p} & t_{2p} & \dots & t_{pp}
\end{pmatrix} \begin{pmatrix}
t_{11} & t_{12} & \dots & t_{1p} \\
0 & t_{22} & \dots & t_{2p} \\
\vdots & \vdots & \ddots & \vdots \\
0 & 0 &\dots & t_{pp}
\end{pmatrix} = \begin{pmatrix}
t_{11}^{2} & * & \dots & * \\
* & t_{12}^{2} + t_{22}^{2} & \dots & * \\
\vdots & \vdots & \ddots & \vdots \\
* & * & \dots & t_{1p}^{2} + t_{2p}^{2} + \dots + t_{pp}^{2}
\end{pmatrix}$$
\end{small}
Entonces:
$$h(T) = -N\ln(|D|) + \sum_{i=1}^{p}[N\ln(t_{ii}^{2}) - t_{ii}^{2}] - \sum_{\underset{i<j}{i, j=1}}^{p} t_{ij}^{2}$$
Alcanza el máximo para $t_{ij} = 0, \forall i < j \Rightarrow T$ diagonal.
$$y = N\ln(x) - x \Rightarrow y' = \frac{N}{x} - 1 \Rightarrow \frac{N}{x} - 1 = 0 \Rightarrow x = N \Rightarrow y'' = \frac{-N}{x^{2}} < 0 \Rightarrow \text{máximo}$$
Para que $h(T)$ alcance el máximo, elegimos $t_{ii}^{2} = N, \forall i \Rightarrow t_{ii} = \sqrt{N}$.

Entonces tenemos:
$$T = \sqrt{N}I_{p} \Rightarrow H = T'T = NI_{p}$$
El máximo de $f(G)$ se obtiene para:
$$G = E'H^{1}E' = E(NI_{p})^{-1}E' = \frac{1}{N}EI_{p}E' = \frac{1}{N}EE' = \frac{1}{N}D$$

\item Demostración:
\begin{itemize}
\item Maximización de $\ln(L(\mu,\Sigma))$ en $\mu$: (Se puede hacer, en este caso, independientemente de $\Sigma$). \\
De la forma obtenida para $\ln(L(\mu,\Sigma))$, se desprende que, independientemente del valor de $\Sigma$, el máximo se alcanzará donde se minimice la forma cuadrática
$$(\bar{x} - \mu)' \Sigma^{-1} (\bar{x} - \mu)$$
Ahora bien, dado que $\Sigma^{-1}$ es definida positiva (por serlo $\Sigma$), la forma cuadrática alcanza el valor mínimo 0 para (y sólo para) $\bar{x} - \mu = 0$, es decir, $\mu = \bar{x}$. Por tanto, el estadístico definido por:
$$\hat{\mu} := \bar{X} \hspace{1cm} \text{(el vector de medias muestral)}$$
es el EMV (único, c.s.) de $\mu$.

\item Maximización de $\ln(L(\bar{x}, \Sigma))$ en $\Sigma$: \\
Dado que tratamos de maximizar la función
$$\ln(L(\bar{x},\Sigma)) = -\frac{pN}{2} \ln(2\pi) - \frac{N}{2}\ln(|\Sigma|) - \frac{1}{2} \operatorname{tr}(\Sigma^{-1}\bar{A})$$
(donde ya ha desaparecido el término correspondiente a la forma cuadrática), identificamos, en la notación del `lema de Watson',
$$G := \Sigma \hspace{1cm} D := \bar{A}$$
$$f(\Sigma) := 2[\ln L(\bar{x}, \Sigma) + \frac{pN}{2}\ln(2\pi)] = pN\ln(2\pi) + 2\ln(L(\bar{x}, \Sigma)) = -N\ln(|\Sigma|) - \operatorname{tr}(\Sigma^{-1}\bar{A})$$
por lo que el máximo de $f$ en $\Sigma$ (igualmente, entonces, el máximo de $L(\bar{x}, \Sigma)$ en $\Sigma$) se alcanza para (y sólo para) $\Sigma = \frac{\bar{A}}{N} =: \bar{S}_{N}$. Por tanto, el estadístico definido por
$$\hat{\Sigma} := \frac{A}{N} = S_{N} \hspace{1cm} (\text{la matriz de covarianzas muestral})$$
es el EMV (único, c.s.) de $\Sigma$.
\end{itemize}

\item Se comprueba que el valor máximo alcanzado por la función de verosimilitud en el punto $(\mu, \Sigma) = (\bar{x}, \bar{S}_{N})$ del espacio paramétrico es:
$$L(\mu, \Sigma) = \frac{1}{(2\pi e)^{\frac{pN}{2}}|\bar{S}_{N}|^{\frac{N}{2}}} = [\frac{N}{(2\pi e)|\bar{A}|^{\frac{1}{p}}}]^{\frac{pN}{2}} \hspace{1cm} (3)$$
Como puede verse [...] % COMPLETAR
\end{itemize}

\subsubsection{Teorema de Dykstra}
\begin{itemize}
\item En relación con el resultado anterior sobre el EMV del par paramétrico ($\mu, \Sigma$), se plantea la cuestión siguiente:

Formalmente, puede ocurrir que para una realización concreta de la muestra aleatoria la matriz $\bar{A}$ resulte ser singular (es decir, semidefinada positiva). Ahora bien, ¿con qué probabilidad puede darse este caso?

\item La respuesta viene dada por el `teorema de Dykstra' (1970): \\
TEOREMA (Dykstra): Sea $X \sim N_{p}(\mu, \Sigma)$ ($\Sigma > 0$). Sea $\{X_{\alpha}: \alpha=1,\dots,N\}$ una muestra aleatoria simple de dicha distribución y sea $A = \sum_{\alpha=1}^{N}(X_{\alpha} - \bar{X})(X_{\alpha}-\bar{X})'$ la correspondiente matriz de dispersiones muestral. Entonces, $A$ es definida positiva, con probabilidad 1, si y sólo si $N > p$.

\item Para la demostración del teorema, se establecen algunos resultados previos: \\
LEMA 1: Sean $\{X_{\alpha}: \alpha=1,\dots,N\}$ vectores (aleatorios o no) de dimensión $p$. Sea $C = (c_{\alpha\beta})_{\alpha,\beta=1,\dots,N}$ una matriz ortogonal de dimensión $N \times N$. Entonces, definiendo
$$Y_{\alpha} = \sum_{\beta=1}^{N} c_{\alpha\beta}X_{\beta}, \hspace{1cm} \alpha=1,\dots,N$$
se tiene que
$$\sum_{\alpha=1}^{n}X_{\alpha}X_{\alpha}' = \sum_{\alpha=1}^{N}Y_{\alpha}Y_{\alpha}' \hspace{1cm} (4)$$

\item (4) $$\sum_{\alpha=1}^{N} Y_{\alpha}Y_{\alpha}' = \sum_{\alpha=1}^{N} (\sum_{\beta=1}^{N} c_{\alpha\beta}X_{\beta})(\sum_{\gamma=1}^{N} c_{\alpha\gamma}X_{\gamma})' = \sum_{\alpha=1}^{N} \sum_{\beta=1}^{N} \sum_{\gamma=1}^{N} c_{\alpha\beta} c_{\alpha\gamma} X_{\beta}X_{\gamma}' =$$
$$= \sum_{\beta=1}^{N} \sum_{\gamma=1}^{N} X_{\beta} X_{\gamma}' (\sum_{\alpha=1}^{N} c_{\alpha\beta} c_{\alpha\gamma}) =$$
Como tenemos $(\sum_{\alpha=1}^{N} c_{\alpha\beta} c_{\alpha\gamma}) = <c_{\beta}, c_{\gamma}> = \delta_{\beta\gamma} = \begin{cases}
0 & \text{si } \beta \neq \gamma \\
1 & \text{si } \beta = \gamma
\end{cases}$, entonces:
$$= \sum_{\beta=1}^{N} X_{\beta}X_{\beta}'$$

\item LEMA 2: Sean $\{X_{\alpha}: \alpha=1,\dots,N\}$ vectores aleatorios de dimensión $p$, con $X_{\alpha} \sim N_{p}(\mu_{\alpha}, \Sigma), \alpha = 1,\dots,N$ (es decir, tienen la misma matriz de covarianzas, aunque posiblemente distinto vector de medias), independientes. Sea $C = (c_{\alpha\beta})_{\alpha, \beta = 1,\dots,N}$ una matriz ortogonal de dimensión $N \times N$. Entonces, definiendo
$$Y_{\alpha} = \sum_{\beta=1}^{N} c_{\alpha\beta}X_{\beta}, \hspace{1cm} \alpha=1,\dots,N$$
se tiene
\begin{itemize}
\item $Y_{\alpha} \sim N_{p}(\mathcal{V}_{\alpha}, \Sigma)$, con $\mathcal{V}_{\alpha} = \sum_{\beta=1}^{N}c_{\alpha\beta}\mu_{\beta}, \alpha = 1, \dots, N \quad$ (5).
\item Los vectores $\{Y_{\alpha}: \alpha=1,\dots,N\}$ son independientes (6).
\end{itemize}

\item (5) $$c_{\alpha\beta} X_{\beta} = c_{\alpha\beta} I_{p} X_{\beta} = \begin{pmatrix}
c_{\alpha\beta} & 0 & \dots & 0 \\
0 & c_{\alpha\beta} & \dots & 0 \\
\vdots & \vdots & \ddots & \vdots \\
0 & 0 & \dots & c_{\alpha\beta}
\end{pmatrix} X_{\beta} = A_{\alpha\beta} X_{\beta}$$
$$Y_{\alpha} = \sum_{\beta=1}^{N} A_{\alpha\beta} X_{\beta} \sim N_{p}(\sum_{\beta=1}^{N}(A_{\alpha\beta}\mu_{\beta}), \sum_{\beta=1}^{N} A_{\alpha\beta} \Sigma A_{\alpha\beta}') \equiv N_{p}(\sum_{\alpha=1}^{N}c_{\alpha\beta} \mu_{\beta}, \sum_{\beta=1}^{N}c_{\alpha\beta}^{2}\Sigma) \equiv N_{p}(\mathcal{V}_{\alpha}, \Sigma)$$
donde hemos usado que $\sum_{\beta=1}^{N} c_{\alpha\beta}^{2}=1$.

Entonces
$$\Phi_{Y_{\alpha}}(t) = e^{it'\mathcal{V}_{\alpha} - \frac{1}{2}t' \Sigma t}$$

\item (6) Consideramos el vector $(pN) \times 1$:
$$Y = \begin{pmatrix} Y_{1} \\ Y_{2} \\ \vdots \\ Y_{N} \end{pmatrix}$$
Para cada $t \in \mathbb{R}^{pN}$, tenemos:
$$t = \begin{pmatrix} t_{1} \\ t_{2} \\ \vdots \\ t_{N} \end{pmatrix}$$
con $t_{\alpha} \in \mathbb{R}, \alpha = 1, \dots, N$. \\
La función característica de $Y$ se calcula como:
$$\Phi_{Y_{1}, \dots, Y_{N}}(t) = E[e^{it'Y}] = E[e^{i\sum_{\alpha=1}^{N}t_{\alpha}'Y_{\alpha}}] = E[e^{i\sum_{i=1}^{N}t_{\alpha}'(\sum_{\beta=1}^{N} c_{\alpha\beta}X_{\beta})}] =$$
$$= E[e^{i\sum_{\beta=1}^{N} (\sum_{\alpha=1}^{N} t_{\alpha}'c_{\alpha\beta})X_{\beta})}] = E[\prod_{\beta=1}^{N} e^{i(\sum_{\alpha=1}^{N}t_{\alpha}'c_{\alpha\beta}X_{\beta}}] =$$
Como $X_{1}, \dots, X_{N}$ son independientes, tenemos que la esperanza del producto es el producto de las esperanzas:
$$= \prod_{\beta=1}^{N} E[e^{i(\sum_{\alpha=1}^{N} t_{\alpha}'c_{\alpha\beta})X_{\beta}}] = \prod_{\beta=1}^{N} \Phi_{X_{\beta}}(\sum_{\alpha=1}^{N}t_{\alpha}c_{\alpha\beta}) =$$
Como $X_{\beta} \sim N(\mu_{\beta}, \Sigma), \beta = 1, \dots, N$, tenemos:
$$= \prod_{\beta=1}^{N} e^{i(\sum_{\alpha=1}^{N} t_{\alpha}c_{\alpha\beta})'\mu_{\beta} - \frac{1}{2}(\sum_{\alpha=1}^{N} t_{\alpha}c_{\alpha\beta})'\Sigma(\sum_{\gamma=1}^{N} t_{\gamma}c_{\gamma\beta})} =$$
$$= e^{i\sum_{\beta=1}^{N}(\sum_{\alpha=1}^{N} t_{\alpha}c_{\alpha\beta})'\mu_{\beta} - \frac{1}{2} \sum_{\beta=1}^{N}[(\sum_{\alpha=1}^{N} t_{\alpha}c_{\alpha\beta})' \Sigma (\sum_{\gamma=1}^{N} t_{\gamma}c_{\gamma\beta})]} =$$
$$= e^{i\sum_{\alpha=1}^{N} t_{\alpha}' (\sum_{\beta=1}^{N} c_{\alpha\beta}\mu_{\beta}) - \frac{1}{2} \sum_{\alpha=1}^{N} \sum_{\gamma=1}^{N}[t_{\alpha}' \Sigma t_{\gamma} (\sum_{\beta=1}^{N}c_{\alpha\beta}c_{\gamma\beta})]} =$$
donde tenemos que $\sum_{\beta=1}^{N}c_{\alpha\beta}c_{\gamma\beta} = \delta_{\alpha\gamma}$.
$$= e^{i\sum_{\alpha=1}^{N} t_{\alpha}'\mathcal{V}_{\alpha} - \frac{1}{2} \sum_{\alpha=1}^{N} t_{\alpha}' \Sigma t_{\alpha}} = \prod_{\alpha=1}^{N}e^{it_{\alpha}'\mathcal{V}_{\alpha} - \frac{1}{2}t_{\alpha}' \Sigma t_{\alpha}} = \prod_{\alpha=1}^{N} \Phi_{Y_{\alpha}}(t_{\alpha})$$
$\Rightarrow Y_{1}, \dots, Y_{N}$ independientes.

\item Demostración (esquema): Consideremos, en nuestro caso, una matriz $B = (b_{\alpha\beta})_{\alpha, \beta = 1, \dots, N}$ de dimensión $N \times N$, ortogonal, cuya última fila sea $(\frac{1}{\sqrt{N}}, \dots, \frac{1}{\sqrt{N}})$. Es decir, el resto de filas han de ser vectores ortogonales con éste y entre sí, con norma igual a 1. Definimos el vector:
$$Z_{\alpha} = \sum_{\beta=1}^{N} b_{\alpha\beta} X_{\beta}, \hspace{1cm} \alpha = 1, \dots, N$$
Por los resultados anteriores (lemas 1 y 2), tenemos que:
\begin{itemize}
\item $Z_{\alpha} \sim N_{p}(\mathcal{V}_{\alpha}, \Sigma)$, con:
$$\mathcal{V}_{\alpha} = \sum_{\beta=1}^{N}b_{\alpha\beta}\mu = (\sum_{\beta=1}^{N} b_{\alpha\beta}) \mu = \begin{cases}
0, & \text{ si } \alpha = 1, \dots, N-1, \\
\frac{N}{\sqrt{N}}\mu = \sqrt{N}\mu, & \text{ si } \alpha = N
\end{cases}$$

\item Los vectores $\{Z_{\alpha}: \alpha = 1, \dots, N\}$ son independientes.

\item $Z_{m} = \sum_{\beta=1}^{N} \frac{1}{\sqrt{N}}X_{\beta} = \sqrt{N}(\frac{1}{N} \sum_{\beta=1}^{N} X_{\beta}) = \sqrt{N}\bar{X}$. \\
(es decir, $\bar{X} = \frac{1}{\sqrt{N}}Z_{N})$.

\item $A = \sum_{\alpha=1}^{N}(X_{\alpha} - \bar{X})(X_{\alpha} - \bar{X})' = \sum_{\alpha=1}^{N} X_{\alpha}X_{\alpha}' - N\bar{X}\bar{X}' =$ \\ $= \sum_{\alpha=1}^{N} Z_{\alpha}Z_{\alpha}' - Z_{N}Z_{N}' = \sum_{\alpha=1}^{N-1}Z_{\alpha}Z_{\alpha}'$.

\item Como consecuencia, $\bar{X}$ y $A$ son independientes.
\end{itemize}
Por último, observamos que podemos escribir
$$A = Z'Z$$
con $Z = \begin{pmatrix} Z_{1}' \\ \vdots \\ Z_{N-1}' \end{pmatrix}$, matriz $(N-1) \times p$, $Z' = \begin{pmatrix} Z_{1} & \dots & Z_{N-1} \end{pmatrix}$, matriz $p \times (N-1)$.

Esto equivale a que la matriz $A$ es definida no negativa.

Por otra parte, se cumple que
$$\operatorname{rango}(Z) = \operatorname{rango}(Z') = \operatorname{rango}(Z'Z) = \operatorname{rango}(A)$$
Por tanto, la demostración quedará concluida probando que $\operatorname{rango}(Z) = p$ con probabilidad 1 si y sólo si $N > p \quad$ (6).

\item (6) \begin{itemize}
\item[$\Rightarrow$] Si fuese $N \leq p$, sería $N-1 < p$. Es decir, el rango de $Z$ sería como máximo igual a $N - 1 < p$.
\item[$\Leftarrow$] Supongamos que $N > p$. Vamos a probar que, entonces, $\operatorname{rango}(Z) = \operatorname{rango}(A) = p$ con probabilidad 1.

Probamos primero para $N = p+1$, y después se argumentará por inducción para cualquier $N > p$.

$N = p + 1 \Rightarrow N - 1 = p \Rightarrow Z' = (Z_{1}, \dots, Z_{p})_{p \times p}$ y $Z_{\alpha} \sim N_{p}(0, \Sigma), \alpha = 1, \dots, N-1$ independientes.

\begin{itemize}
\item NOTA: Para cada $\alpha = 1, \dots, N-1 = p$, se tiene que
$$P[Z_{\alpha} \in S] = 0, \forall S \in \mathcal{B}^{p} \text{ tal que } \underset{\text{medida de Lebesgue en } (\mathbb{R}^{p}, \mathcal{B}^{p})}{\lambda_{p}}(S) = 0$$
$$P_{Z_{\alpha}} \underset{\text{``absolutamente continua''}}{<<} \lambda_{p}$$
En particular, es cierto para cualquier hiperplano ($p-1$)-dimensional
$$S(a_{1}, \dots, a_{p-1})$$
engendrado por los vectores $a_{1}, \dots, a_{p-1} \in \mathbb{R}^{p}$.
\end{itemize}

Vamos a probar que la matriz $Z' = \begin{pmatrix} Z_{1}, \dots, Z_{p=N-1} \end{pmatrix}$ tiene rango $p$.
$$P[\operatorname{rango}(Z) < p] = P[Z_{1}, \dots, Z_{p} \text{ son linealmente dependientes}]=$$
$$= P[\{\omega \in \Omega: Z_{1}(\omega), \dots, Z_{p}(\omega) \text{ son linealmente dependientes}\}] \leq$$
$$\leq \sum_{i=1}^{p}P[Z_{i} \in S(Z_{1}, \dots, Z_{i-1}, Z_{i+1}, \dots, Z_{p})] = pP[Z_{1} \in S(Z_{2}, \dots, Z_{p})] =$$
$$= p \cdot 0 = 0$$
$S(Z_{1}, \dots, Z_{i-1}, Z_{i+1}, \dots, Z_{p})$ hiperplano generado por $Z_{1}, \dots, Z_{i-1}, Z_{i+1}, \dots, Z_{p}$. \\
Falta probar que $P[Z_{1} \in S(Z_{2}, \dots, Z_{p})] = 0$.

\begin{itemize}
\item NOTA: En general,
$$P[A] = E_{p}[I_{A}(\cdot)] = \int_{\Omega} I_{A}(\omega) P(dw) = \int_{A} P(dw)$$
\end{itemize}

Para ello:
$$P[Z_{1} \in S(Z_{2}, \dots, Z_{p})] = E_{p}[I_{[z_{1} \in S(z_{2}, \dots, z_{p})]}(\cdot)] =$$
$$= \int_{\mathbb{R}^{p} \times \dots \times \mathbb{R}^{p}} I_{[z_{1} \in S(z_{2}, \dots, z_{p})]}(z_{1}, \dots, z_{p}) P_{Z_{1} \dots Z_{p}}(dz_{1}, \dots, dz_{p}) =$$

\begin{itemize}
\item Como son independientes
$$P_{Z_{1}, \dots, Z_{p}} (dz_{1} \dots dz_{p}) = P_{Z_{1}}(dz_{1})P_{Z_{2}, \dots, Z_{p}}(dz_{2} \dots dz_{p})$$
\end{itemize}

$$= \int_{\mathbb{R}^{p} \times \dots \times \mathbb{R}^{p}} (\int_{\mathbb{R}^{p}} I_{[z_{1} \in S(z_{2}, \dots, z_{p})}(z_{1}, \dots, z_{p}) P_{Z_{1}}(dz_{1})) P_{z_{2}, \dots, z_{p}} (dz_{2}, \dots, dz_{p}) =$$
$$= \int_{\mathbb{R}^{p} \times \dots \times \mathbb{R}^{p}} (P_{Z_{1}}[Z_{1} \in S(z_{2}, \dots, z_{p}]) P_{z_{2}, \dots, z_{p}} (dz_{2} \dots dz_{p}) =$$
$$= \int_{\mathbb{R}^{p} \times \dots \times \mathbb{R}^{p}} 0 P_{z_{2}, \dots, z_{p}} (dz_{2} \dots dz_{p}) = 0$$

Finalmente, veamos la inducción. Supongamos que la implicación se cumple para cierto $N > p$ y vamos a ver que entonces se cumple para $N' = N+1$.

Definimos $Z^{[N]}$ como la matriz $Z$ definida previamente asociada al conjunto $X_{1}, \dots, X_{N}$, y $Z^{[N']}$ como la matriz asociada al conjunto $X_{1}, \dots, X_{N}, X_{N'}$.
$$B^{[N]} = \begin{pmatrix}
b_{11} & b_{12} & \dots & b_{1N} \\
\vdots & \vdots & \ddots & \vdots \\
b_{N-1 \hspace{0.1cm} 1} & b_{N-1 \hspace{0.1cm} 2} & \dots & b_{N-1 \hspace{0.1cm} N} \\
\frac{1}{\sqrt{N}} & \frac{1}{\sqrt{N}} & \dots & \frac{1}{\sqrt{N}}
\end{pmatrix}$$
$$B^{[N']} = \begin{pmatrix}
b_{11} & b_{12} & \dots & b_{1N} & 0 \\
\vdots & \vdots & \ddots & \vdots & \vdots \\
b_{N-1 \hspace{0.1cm} 1} & b_{N-1 \hspace{0.1cm} 2} & \dots & b_{N-1 \hspace{0.1cm} N} & 0 \\
b_{N1} & b_{N2} & \dots & b_{NN} & b_{NN+1} \\
\frac{1}{\sqrt{N}} & \frac{1}{\sqrt{N}} & \dots & \frac{1}{\sqrt{N}} & \frac{1}{\sqrt{N}}
\end{pmatrix}$$

$$Z^{[N]} = \begin{pmatrix}
Z_{1}^{[N]'} \\ \vdots \\ Z_{N-1}^{[N]'}
\end{pmatrix}$$
$$Z^{[N']} = \begin{pmatrix}
Z_{1}^{[N']'} = Z_{1}^{[N]'} \\ \vdots \\ Z_{N-1}^{[N']'} = Z_{N-1}^{[N]'} \\ Z_{N}^{[N']'}
\end{pmatrix}$$
\end{itemize}
\end{itemize}

\subsubsection{Teorema de Fisher multivariante}
De la demostración del teorema de Dykstra, se tiene también la prueba del siguiente resultado, conocido como `teorema de Fisher multivariante':

TEOREMA (Fihser): Dada una muestra aleatoria simple $X_{1}, \dots, X_{N}$ de distribución $N_{p}(\mu, \Sigma)$, el vector de medias muestral $\bar{X}$ se distribuye según una $N_{p}(\mu, \frac{\Sigma}{N})$, y la matriz de dispersiones muestral de igual modo que $\sum_{\alpha=1}^{N-1}Z_{\alpha}Z_{\alpha}'$, con $Z_{1}, \dots, Z_{N-1}$ vectores aleatorios independientes e idénticamente distribuidos según una $N_{p}(0, \Sigma)$, siendo el vector $\bar{X}$ y la matriz $A$ independientes.

\subsubsection{Propiedades de los EMV \texorpdfstring{$\hat{\mu}$}) y \texorpdfstring{$\hat{\Sigma}$})}
En este apartado se trata, a continuación, sobre el posible cumplimiento de algunas propiedades importantes por los EMV $\hat{\mu} = \hat{X}$ y $\hat{\Sigma} = S_{N}$. Concretamente:
\begin{itemize}
\item Insesgadez.
\item Consistencia (fuerte, débil).
\item Eficiencia.
\end{itemize}

\subsubsection{Propiedades de los EMV \texorpdfstring{$\hat{\mu}$}) y \texorpdfstring{$\hat{\Sigma}$}) Sobre INSESGADEZ}
DEFINICIÓN: Sea $\{X_{\alpha}: \alpha=1, \ldots, N\}$ una muestra aleatoria simple de una distribución dependiente de un parámetro (en general, multidimensional), $\theta = (\theta_{1}, \ldots, \theta_{k})' \in \Theta \subseteq \mathbb{R}^{k}$. Un estimador $\hat{\theta}$ (i. e., función medible de la muestra, $\hat{\theta} = T(X)$) del parámetro $\theta$ se dice insesgado si

$$E_{\theta}[\hat{\theta}-\theta]=0, \quad \forall \theta \in \Theta \quad \text { (es decir, } E_{\theta}[\hat{\theta}]=\theta, \quad \forall \theta \in \Theta)$$

Sea $\mathbf{X} \sim N_{p}(\mu, \Sigma) \quad(\Sigma>0)$

\begin{itemize}
\item VECTOR DE MEDIAS MUESTRAL: \\
$\hat{\mu}:=\bar{X}$ es un estimador insesgado de $\mu$ (7)

(En realidad, en este caso no hace falta suponer normalidad, solo la existencia de $\mu$ )

\item (7) $\hat{\mu} = \bar{X}$ es estimador insesgado de $\mu$. \\
$(X_{1}, X_{2}, \dots, X_{N})$ v.a.i.i.d., $X_{i} \sim N(\mu, \Sigma)$.
$$\bar{X} = \frac{1}{N} \sum_{\alpha=1}^{N} X_{\alpha} = E[\bar{X}] = E[\frac{1}{N} \sum_{\alpha=1}^{N} X_{\alpha}] \overset{\text{linealidad}}{=} |\frac{1}{N} \sum_{\alpha=1}^{N} E[X_{\alpha}] = \frac{1}{N} N\mu = \mu$$

\item MATRIZ DE COVARIANZAS MUESTRAL: \\
$\hat{\Sigma}:=S_{N}$ no es un estimador insesgado de $\Sigma$ (8)

(Tener en cuenta, por el teorema de Fisher, que $A \stackrel{d}{=} \sum_{\alpha=1}^{N-1} Z_{\alpha} Z_{\alpha}'$, con $Z_{\alpha} \sim N_{p}(0, \Sigma), \alpha=1, \ldots, N-1$, independientes)

Se tiene, en este caso, que: \\
$S_{N-1}=\frac{N}{N-1} S_{N}$ sí es un esimador insesgado de $\Sigma$

\item (8) $\hat{\Sigma} = S_{N} = \frac{A}{N}$
$$E[S_{N}] = E[\frac{A}{N}] = \frac{1}{N}E[\sum_{\alpha=1}^{N-1} Z_{\alpha}Z_{\alpha}'] = \frac{1}{N} \sim_{\alpha=1}^{N-1} E[Z_{\alpha}Z_{\alpha}']$$
$$E[S_{N-1}] = E[\frac{N}{N-1}S_{N}] = \frac{N}{N-1} E[\frac{1}{N}] = \frac{1}{N-1}E[A] = \frac{1}{N-1} E[\sum_{\alpha=1}^{N-1} Z_{\alpha}Z_{\alpha}'] =$$ $$= \frac{1}{N-1}(N-1)E[Z_{\alpha}Z_{\alpha}'] = \Sigma$$
\end{itemize}

\subsubsection{Propiedades de los EMV \texorpdfstring{$\hat{\mu}$}) y \texorpdfstring{$\hat{\Sigma}$}): Sobre CONSISTENCIA}
DEFINICIÓN: Sea $\{X_{\alpha}: \alpha=1, \ldots, N\}$ una muestra aleatoria simple (de tamaño $N$, considerado variable) de una distribución dependiente de un parámetro (en general, multidimensional), $\theta(\theta_{1}, \ldots, \theta_{k})' \in \Theta \subseteq \mathbb{R}^{k}$. Un estimador $\hat{\theta}_{N}$ i. e., función medible de la muestra, $\hat{\theta}_{N}=T_{N}(X))$ del parámetro $\theta$ se dice
\begin{enumerate}[label=(\alph*)]
\item débilmente consistente si $\hat{\theta}_{N}$ converge en probabilidad a $\theta$, es decir,
$$\forall \epsilon>0, \quad \lim\limits_{N \to \infty} P_{\theta} [\|\hat{\theta}_{N}-\theta\|<\epsilon]=1$$

\item fuertemente consistente si $\hat{\theta}_{N}$ converge casi seguramente a $\theta$, es decir,
$$P_{\theta} [\lim\limits_{N \to \infty} \hat{\theta}_{N}=\theta]=1$$
\end{enumerate}

Sea $X \sim N_{p}(\mu, \Sigma) \quad(\Sigma>0)$. Se demuestra que: \\
$\hat{\mu}:=\bar{X}$ $\hat{\Sigma}:=S_{N}$ son estimadores fuertemente consistentes de $\mu$ y $\Sigma$, respectivamente.

(En el caso de $\hat{\mu}:=\overline{X}$, no se requiere normalidad. $S_{N-1}$ también es un estimador fuertemente consistente de $\Sigma$) [Se omite la demostración]

\subsubsection{Propiedades de los EMV \texorpdfstring{$\hat{\mu}$}) y \texorpdfstring{$\hat{\Sigma}$}): Sobre EFICIENCIA}
(Para estimadores insesgados)

DEFINICIÓN: Sea $T=(T_{1}, \ldots, T_{k})'$ un estimador insesgado de un parámetro (en general, multidimensional) $\theta=(\theta_{1}, \ldots, \theta_{k})' \in \bar{\Theta}$, siendo $\Theta=\mathbb{R}^{k}$ o un 'rectángulo' en $\mathbb{R}^{k}$. Se dice que $T$ es eficiente para $\theta=(\theta_{1}, \ldots, \theta_{k})'$ si, para cualquier otro estimador insesgado $U$ de $\theta$, se verifica que la diferencia de matrices de covarianzas $\operatorname{Cov}_{\theta}(U)-\operatorname{Cov}_{\theta}(T)$, para todo $\theta \in \Theta$, es una matriz definida no negativa [se suele denotar $\operatorname{Cov}_{\theta}(U) \geq \operatorname{Cov}_{\theta}(T)$ ]

Sea $X \sim N_{p}(\mu, \Sigma) \quad(\Sigma>0)$. Se demuestra que:
\begin{itemize}
\item $\bar{X}$ es un estimador eficiente de $\mu$ en el espacio $\mathbb{R}^{p}$
\item $S_{N-1}$ es un estimador eficiente de $\Sigma$ en el espacio de matrices simétricas definidas positivas de dimensión $p \times p$.
\end{itemize}

\subsubsection{Teorema de Zehna: INVARIANCIA de los estimadores MV}
TEOREMA (Zehna, 1966): Sea $\mathcal{P}=\{P_{\theta}: \theta \in \Theta\}$ una familia de distribuciones de probabilidad sobre el espacio $(\mathbb{R}^{p}, \mathcal{B}^{p})$. Sea $g: \Theta \rightarrow \Lambda$ una función arbitraria dada. Si $\hat{\theta}$ es un estimador máximo-verosímil de $\theta$, entonces $g(\hat{\theta})$ es un estimador máximo-verosímil de $g(\theta)$

[OBSERVACIÓN: Este teorema hace uso del concepto de 'función de verosimilitud inducida por $g$ : Para cada $\lambda \in g(\Theta) \subseteq \Lambda$, se define

$$M(\lambda)=\sup\limits_{\theta \in \Theta_{\lambda}} L(\theta), \quad \text{ con } \Theta_{\lambda}=\{\theta \in \Theta: g(\theta)=\lambda\}$$

Se entiende entonces que `$g(\hat{\theta})$ es un EMV de $g(\theta)$' en el sentido de que $g(\hat{\theta})$ es un máximo para $M$

En el contexto de la inferencia sobre la DNM, se tiene que para distintos coeficientes que dependen funcionalmente de $\mu$ y $\Sigma$ (por ejemplo, coeficientes de correlación, regresión, etc.) los correspondientes estimadores máximo-verosímiles se obtienen mediante sustitución de $\mu$ y $\Sigma$ por $\overline{X}$ y $S_{N}$, respectivamente, en las expresiones que definen dichos coeficientes

\subsubsection{[Complemento] Distribuciones asintóticas de \texorpdfstring{$\hat{\mu}$}) y \texorpdfstring{$\hat{\Sigma}$}), con distribución de referencia no necesariamente normal}
\begin{itemize}
\item MOTIVACIÓN (referida al vector de medias muestral):

Recordemos que, al enunciar y probar el teorema de Fisher, se ha obtenido que para una muestra aleatoria simple $X_{1}, \ldots, X_{N}$ de una distribución $N_{p}(\mu, \Sigma)$ se tiene que
$$\overline{X} \sim N_{p}(\mu, \frac{\Sigma}{N});$$
es decir, normalizando en media y tamaño muestral,
$$N^{\frac{1}{2}}(\overline{X}-\mu)=N^{-\frac{1}{2}} \sum_{\alpha=1}^{N}(X_{\alpha}-\mu) \sim N_{p}(0, \Sigma)$$
(la distribución límite ya no depende de $N$ ).

\item ¿Qué puede afirmarse, al menos como aproximación límite (para $N \rightarrow \infty)$, cuando la distribución de origen no es necesariamente una DNM?

\item Distribución asintótica del vector de medias muestral, $\overline{X}$.

RESULTADO: Sea $X_{1}, \ldots, X_{N}, \ldots$ una sucesión de vectores aleatorios $p$-dimensionales independientes e idénticamente distribuidos, con vector de medias $\mu$ y matriz de covarianzas $\Sigma$ (i. e., $X_{\alpha} \sim(\mu, \Sigma), \alpha=1, \ldots, N, \ldots$ ). Sea $\overline{X}_{N}=\frac{1}{N} \sum_{\alpha=1}^{N} X_{\alpha}$, $\forall N \geq 1$. Entonces, cuando $N \to \infty$, se tiene la distribución asintótica
$$N^{\frac{1}{2}}(\overline{X}_{N}-\mu)=N^{-\frac{1}{2}} \sum_{\alpha=1}^{N}(X_{\alpha}-\mu) \underset{N \to \infty}{\sim} N_{p}(0, \Sigma)$$
(Se trata de una versión del Teorema Central del Límite, para vectores i.i.d. $\sim(\mu, \Sigma))$

\item Demostración (Resultado): Sea:
$$Y_{N} = N^{-\frac{1}{2}} \sum_{\alpha=1}^{N}(X_{\alpha} - \mu) \quad (p \times 1)$$
Por el ``Teorema de continuidad de funciones características'', es suficiente probar que la función característica de este vector $\upphi_{Y_{N}}(t)$ converge a
$$exp\{-\frac{1}{2}t' \Sigma t\}, \quad \forall t \in \mathbb{R}^{t}$$
Sabemos que , para cada $t \in \mathbb{R}^{p}$, se tiene que
$$\upphi_{Y_{N}}(t) = \upphi_{t'Y_{N}}(1)$$
Veamos como es la función característica $\upphi_{t'Y_{N}}(s), s \in \mathbb{R}$.

Tenemos que
$$t'Y_{N} = N^{-\frac{1}{2}} \sum_{\alpha=1}^{N}(t'X_{\alpha} - t'\mu)$$
donde sabes que cada sumando $t'X_{\alpha}-t'\mu \sim (0, t' \Sigma t)$ siendo estas variables i.i.d. Aplicamos el Teorema Central del Límite univariante $\star$.

TCL univariante: \\
Sean $X_{1}, X_{2}, \dots$ v.a. i.i.d. $\sim (\mu, \sigma^{2})$. Sea $W_{N} = \sum_{\alpha=1}^{N} X_{\alpha}, N \geq 1$ ($\overline{X}_{N} = \frac{1}{N}W_{N}$). Entonces:
\begin{itemize}
\item con $\sigma^{2} > 0,$ $$\quad \frac{W_{N} - E[W_{N}]}{\sqrt{Var(W_{N})}} = \frac{W_{N} - N\mu}{N^{\frac{1}{2}}\sigma} = \frac{\overline{X}_{N}-\mu}{N^{-\frac{1}{2}}\sigma} \underset{N \to \infty}{\sim} N(0, 1)$$
\item En general (con $\sigma^{2} \geq 0$), $$W_{N} - N\mu \sim (0, N\sigma^{2})$$
$$\overline{X}_{N} - \mu \sim N(0, \frac{\sigma^{2}}{N})$$
\end{itemize}

$$\star \quad t'Y_{N} \underset{N \to \infty}{\sim} N(0, t' \Sigma t)$$
($\Sigma$ puede ser singular, $\Sigma \geq 0$). Si fuese $\Sigma > 0$, podemos escribir:
$$\frac{t'Y_{N}}{\sqrt{t' \Sigma t}} \underset{N \to \infty}{\sim} N(0, 1)$$
Por el teorema de continuidad de funciones características (caso univariante), tenemos que $\forall s \in \mathbb{R}$,
$$\upphi_{t'Y_{N}}(s) \underset{N \to \infty}{\longrightarrow} exp\{-\frac{1}{2}s^{2}(t' \Sigma t)\}$$
Para $s=1$:
$$\upphi_{Y_{N}}(t) = \upphi_{t'Y_{N}}(1) \underset{N \to \infty}{\longrightarrow} exp\{-\frac{1}{2} t' \Sigma t\}$$
Y esa es justamente la función característica de una distribución $N_{p}(0, \Sigma)$.

\item Distribución asintótica de la matriz de dispersiones muestral, $A$. \\
RESULTADO: Sea $X_{1}, \dots, X_{N}, \dots$ una sucesión de vectores aleatorios $p$-dimensionales indpendientes e idénticamente distribuidos, con vector de medias $\mu$ y matriz de covarianzas $\Sigma$ (i.e., $X_{\alpha} \sim (\mu, \Sigma), \alpha = 1, \dots, N, \dots)$, u con momentos de orden cuatro finitos. Sean $\bar{X}_{N} = \frac{1}{N}\sum_{\alpha=1}^{N} X_{\alpha}$ y $A_{n} = \sum_{\alpha=1}^{N}(X_{\alpha} - \bar{X}_{N})(X_{\alpha} - \bar{X}_{N})', \forall N \geq 1$. Entonces, cuando $N \to \infty$, se tiene la distribución asintótica
$$N^{-\frac{1}{2}}(A_{N} - N\Sigma) \underset{N \to \infty}{\sim} N_{p^{2}}(0, V)$$
(en el sentido de que $N^{-\frac{1}{2}}(\operatorname{Vec}(A_{N} - N\operatorname{Vec}(\Sigma)) \underset{N \to \infty}{\sim} N_{p^{2}}(0, V))$, con $V = \operatorname{Cov}(\operatorname{Vec}((X_{\alpha} - \mu)(X_{\alpha} - \mu)'))$.

(OBSERVACIÓN: La matriz $V$ será necesariamente singular, por lo que la distribución límite se refiere al caso general de la DNM)
\end{itemize}

\subsection{Distribución de Wishart}
\subsubsection{Motivación}
\begin{itemize}
\item Según se vio en el Teorema de Fisher multivariante, dada una muestra aleatoria simple \\ $X_{1}, \dots, X_{N}$ de una distribución $N_{p}(\mu, \Sigma) \quad (\Sigma > 0)$, se tiene que la matriz de dispersiones muestral, $A = \sum_{\alpha=1}^{N}(X_{\alpha}-\bar{X})(X_{\alpha}-\bar{X})'$, se distribuye de igual modo que $\sum_{\alpha=1}^{N-1}Z_{\alpha}Z_{\alpha}'$, con $Z_{1}, \dots, Z_{N-1}$ vectores aleatorios independientes e idénticamente distribuidos según una $N_{p}(0, \Sigma)$ $(\Sigma > 0)$.

\item Interesa, por tanto, conocer la forma explícita de esta distribución, que juega un papel fundamental en el contexto de la inferencia estadística multivariante.

\item La distribución de Wishart (1928), que puede verse como una extensión de la distribución $\chi^{2}$ (entre otras distribuciones), constituye el modelo de referencia en relación con la cuestión planteada.
\end{itemize}

\subsubsection{Definición}
\begin{itemize}
\item DEFINICIÓN (IMPLÍCITA): $Z_{1}, \dots, Z_{n}$ vectores aleatorios independientes e idénticamente distribuidos según una $N_{p}(0, \Sigma) \quad (\Sigma \geq 0)$, con $n \geq p$. Se define la distribución de Wishart centrada con $n$ grados de libertad como la distribución de la matriz $\sum_{\alpha=1}^{n} Z_{\alpha}Z_{\alpha}'$. Se denota $W_{p}(n, \Sigma)$.

\item Como consecuencia, en el caso de la matriz de dispersiones de derivadas del muestro aleatorio simple de una población $N_{p}(\mu, \Sigma) \quad (\Sigma > 0)$ [se puede extender al caso $\Sigma \geq 0$], tendremos que
$$A = NS_{N} = (N-1)S_{N-1} \sim W_{p}(n, \Sigma), \quad \text{con } n = N-1$$

\item En el caso $\Sigma > 0$, la función de densidad de la distribución $W_{p}(n, \Sigma)$ tiene la forma
$$f(A) = \frac{|A|^{\frac{n-p-1}{2}} exp\{-\frac{1}{2} \operatorname{tr}(\Sigma^{-1}A)\}}{2^{\frac{np}{2}}|\Sigma|^{\frac{n}{2}}\pi^{\frac{p(p-1)}{4}}\prod_{i=1}^{p}\Gamma(\frac{n+1-i}{2})}, \quad A > 0$$

\item Recordando que la función \emph{función gamma multivariante de orden} $m$ se define como
$$\Gamma_{m}(z) = \pi^{\frac{m(m-1)}{4}} \prod_{i=1}^{m} \Gamma(z - \frac{i-1}{2})$$
se expresa la densidad de Wishart como
$$f(A) = \frac{|A|^{\frac{n-p-1}{2}} exp\{-\frac{1}{2} \operatorname{tr}(\Sigma^{-1}A)\}}{2^{\frac{np}{2}}|\Sigma|^{\frac{n}{2}}\Gamma_{p}(\frac{n}{2})}, \quad A > 0$$
\end{itemize}

\subsubsection{La distribución de Wishart como extensión de la \texorpdfstring{$\chi^{2}$})}
\begin{itemize}
\item Resultado: Si $a \sim W_{1}(n, \sigma^{2})$, entonces $\frac{a}{\sigma^{2}} \sim \chi_{n}^{2} \quad$ (1).

\item (1) Suponemos $a \sim W_{1}(n, \sigma^{2}), \sigma^{2} > 0$.
$$f(a) \overset{p=1}{=} \frac{a^{\frac{n-2}{2}}e^{\frac{a}{2\sigma^{2}}}}{2^{\frac{n}{2}}(\sigma^{2})^{\frac{n}{2}}\Gamma(\frac{n}{2})}, a > 0$$
$$cv : c := \frac{a}{\sigma^{2}}, a = v\sigma^{2}, da = \sigma^{2}dv$$
$$f(v) = \frac{(\sigma^{2}v)^{\frac{n-2}{2}} e^{-\frac{v}{2}}}{2^{\frac{n}{2}}(\sigma^{2})^{\frac{n}{2}}\Gamma(\frac{n}{2})}\sigma^{2} = \frac{v^{\frac{n-2}{2}}e^{-\frac{v}{2}}}{2^{\frac{n}{2}}\Gamma(\frac{n}{2})}, v > 0$$
$\Rightarrow v \sim \chi_{n}^{2} \Rightarrow \frac{a}{\sigma^{2}} \sim \chi_{n}^{2}$.

También se puede demostrar de forma más sencilla a partir de la función característica, que veremos a continuación.

\item OBSERVACIÓN: En relación con el resultado anterior, si
$$\hat{\sigma^{2}} = s_{N}^{2} := \frac{1}{N} \sum_{\alpha=1}^{N}(x_{\alpha} - \bar{x})^{2}$$
es el EMV de $\sigma^{2}$ en el caso unidimensional ($p=1$), entonces se tiene que
$$\frac{Ns_{N}^{2}}{\sigma^{2}} = \frac{\sum_{\alpha=1}^{N}(x_{\alpha} - \bar{x})^{2}}{\sigma^{2}} \sim \chi_{n}^{2}, \quad \text{con } n=N-1$$
\end{itemize}

\subsubsection{Función característica}
En primer lugar, veamos cómo se define, de forma general, la función característica de una matriz aleatoria (no necesariamente cuadrada):

\begin{itemize}
\item Sea $Y$ una matriz aleatoria $(r \times s)-$dimensional. Se define la función característica de $Y$ como
$$\upphi_{Y}(\Theta) = E[e^{i\operatorname{tr}(\Theta'Y)}]$$
$\forall \Theta$ en el espacio de matrices de dimensión $r \times s$.

\item En realidad, esta expresión proviene de la vectorización de $Y$ u $\Theta$ y el cálculo usual correspondiente de la función característica según la forma vectorial:
$$\upphi_{\operatorname{Vec}(Y)}(\operatorname{Vec}(\Theta)) = E[e^{i\operatorname{Vec}(\Theta)'\operatorname{Vec}(Y)}] = E[e^{i\sum_{l=1}^{s}\sum_{k=1}^{r}\theta_{kl}Y_{kl}}] = E[e^{i\operatorname{tr}(\Theta'Y)}]$$
(es decir, en la doble suma solo intervienen los elementos diagonales de la matriz ($s \times s$)-dimensional $\Theta'Y$).

\item Ahora bien, en el caso de una matriz aleatoria $Y$ ($r \times r$)-dimensional y simétrica, se puede restringir la formulación al conjunto de las $\frac{1}{2}r(r+1)$ variables correspondientes a la diagonal y la parte triangular superior [o inferior] de $Y$, adoptando la forma reducida.
$$\upphi_{Y}^{\star}(\Theta) = E[\exp\{i\sum_{l=1}^{r}\sum_{\underset{k \leq l}{k=1}}^{r}\theta_{kl}Y_{kl}\}]$$
$\forall \Theta$ en el espacio de matrices simétricas de dimensión $r \times r$.

\item Se tiene el siguiente resultado para la distribución de Wishart: \\
RESULTADO: Sea $A \sim W_{p}(n, \Sigma)$. Entonces:
$$\upphi_{A}(\Theta) = |I_{p} - 2i\Theta\Sigma|^{-\frac{n}{2}}$$
$$\upphi_{A}^{\star}(\Theta) = |I_{p} - i\Gamma\Sigma|^{-\frac{n}{2}}$$
$\forall \Theta$ en el espacio de matrices simétricas de dimensión $p \times p$, siendo $\Gamma = (\gamma_{kl})$ (matriz ($p \times p$)-dimensional), con $\gamma_{kl} = (1 + \delta_{kl})\theta_{kl}$, y $\delta_{kl}$ la delta de Kronecker.
\end{itemize}

\subsubsection{Algunas propiedades de la distribución de Wishart}
\begin{itemize}
\item[1] MOMENTO DE PRIMER Y SEGUNDO ORDEN: \\
RESULTADO: Sea $A \sim W_{p}(n, \Sigma)$. Entonces,
$$E[A] = n \Sigma$$
$$\operatorname{Cov}(a_{ij}, a_{kl}) = n(\sigma_{ij}\sigma_{jl} + \sigma_{il}\sigma_{jk}) \quad (i, j, k, l = 1, \dots, p)$$
(Se puede obtener una expresión matricial para la matriz de covarianzas de $A$ en términos de productos de Kronecker).

\item Teorema de Isserlis: En una DNM
$$E[X_{i}X_{j}X_{k}X_{l}] = \sigma_{ij}\sigma_{kl} + \sigma_{ik}\sigma_{kl} + \sigma_{il}\sigma_{jk}$$
(Se usa para la demostración de la propiedad 1)

\item Demostración extra: $A \sim W_{p}(n, \Sigma) \Rightarrow A \overset{d}{=} \sum_{\alpha=1}^{n} Z_{\alpha}Z_{\alpha}', Z_{\alpha} \sim N_{p}(0, \Sigma)$.
$$E[A] = E[\sum_{\alpha=1}^{n} Z_{\alpha}Z_{\alpha}'] = \sum_{\alpha=1}^{n} E[Z_{\alpha}Z_{\alpha}'] = \sum_{\alpha=1}^{n}(\Sigma) = n\Sigma$$
$$Cov[a_{ij}, a_{kl}] = E[a_{ij}a_{kl}] - E[a_{ij}]E[a_{kl}] = \star$$
Como $a_{ij} \overset{d}{=} \sum_{\alpha=1}^{n} Z_{\alpha_{i}}Z_{\alpha_{j}}$:
$$A \overset{d}{=} \sum_{\alpha=1}^{n} Z_{\alpha}Z_{\alpha}' = \sum_{\alpha=1}^{n} \begin{pmatrix} Z_{\alpha_{1}} \\ \vdots \\ Z_{\alpha_{p}} \end{pmatrix} \begin{pmatrix} Z_{\alpha_{1}} & \dots & Z_{\alpha_{p}} \end{pmatrix} = \sum_{\alpha=1}^{n} \begin{pmatrix}
Z_{\alpha_{1}}^{2} & Z_{\alpha_{1}}Z_{\alpha_{2}} & \dots \\
\vdots & \vdots & \ddots
\end{pmatrix}$$
$$E[a_{ij}] = E[\sum_{\alpha=1}^{n}Z_{\alpha_{i}}Z_{\alpha_{j}}] = \sum_{\alpha=1}^{n} E[Z_{\alpha_{i}}Z_{\alpha_{j}}] = \sum_{\alpha=1}^{n} \sigma_{ij} = n\sigma_{ij}$$
$$E[a_{ij}a_{kl}] = E[(\sum_{\alpha=1}^{n} Z_{\alpha_{i}}Z_{\alpha_{j}})(\sum_{\beta=1}^{n}Z_{\beta_{k}}Z_{\beta_{l}})] = E[\sum_{\alpha=1}^{n} \sum_{\beta=1}^{n} Z_{\alpha_{i}}Z_{\alpha_{j}} Z_{\beta_{k}} Z_{\beta_{l}}] =$$
$$= \sum_{\alpha=1}^{n} \sum_{\beta=1}^{n} E[Z_{\alpha_{i}} Z_{\alpha_{j}} Z_{\beta_{k}} Z_{\beta_{l}}] \overset{\text{T. Isserlis}}{=} \sum_{\alpha=1}^{n} \sum_{\beta=1}^{n} (\sigma_{ij}\sigma_{kl} + \sigma_{ik}\sigma_{jl} + \sigma_{il}\sigma_{jk}) =$$
$$= n^{2} (\sigma_{ij}\sigma_{kl} + \sigma_{ik}\sigma_{jl} + \sigma_{il}\sigma_{jk})$$
$$\star = n^{2}(\sigma_{ij}\sigma_{kl} + \sigma_{ik}\sigma_{jl} + \sigma_{il}\sigma_{jk}) - n\sigma_{ij}n\sigma_{kl} = n^{2}(\sigma_{ij}\sigma_{jl} + \sigma_{ij}\sigma_{jk})$$

Otra forma de verlo es razonar igual hasta $E[a_{ij}a_{kl}]$, y luego distinguimos casos:
$$E[Z_{\alpha_{i}} Z_{\alpha_{j}} Z_{\beta_{k}} Z_{\beta_{l}}] = \begin{cases}
E[Z_{\alpha_{i}}Z_{\alpha_{j}}] E[Z_{\alpha_{k}} Z_{\alpha_{l}} & \text{si } \alpha \neq \beta \\
E[Z_{\alpha_{i}} Z_{\alpha_{j}} Z_{\alpha_{k}} Z_{\alpha_{l}}] & \text{si } \alpha = \beta
\end{cases}$$
Entonces:
$$E[a_{ij}a_{kl}] = \begin{cases}
\sigma_{ij}\sigma_{kl} & \alpha \neq \beta \\
\sigma_{ij}\sigma_{kl} + \sigma_{ik}\sigma_{jl} + \sigma_{il}\sigma_{jk} & \alpha = \beta 
\end{cases}$$
$$E[a_{ij}a_{kl}] = \sum_{\alpha=1}^{n} E[Z-{\alpha_{i}} Z_{\alpha_{j}} Z_{\alpha_{k}} Z_{\alpha_{l}}] + \sum_{\underset{\alpha \neq \beta}{\alpha, \beta = 1}}^{n} E[Z_{\alpha_{i}} Z_{\alpha_{j}} Z_{\beta_{k}} Z_{\beta_{l}}] =$$
$$= n(\sigma_{ij}\sigma_{kl} + \sigma_{ik}\sigma_{jl} + \sigma_{il}\sigma_{jk}) + (n^{2}-n)\sigma_{ij}\sigma_{kl}$$
Por lo que
$$Cov(a_{ij}a_{kl}) = n(\sigma_{ij}\sigma_{kl} + \sigma_{ik}\sigma_{jl} + \sigma_{il}\sigma_{jk}) + (n^{2}-n)(\sigma_{ij}\sigma_{kl}) - n^{2}\sigma_{ij}\sigma_{kl} =$$
$$= n(\sigma_{ij}\sigma_{kl} + \sigma_{ik}\sigma_{jl} + \sigma_{il}\sigma_{jk}) - n\sigma_{ij}\sigma_{kl} = n(\sigma_{ij}\sigma_{jl} + \sigma_{il}\sigma_{jk})$$

\item[2] REPRODUCTIVIDAD RESPECTO A GRADOS DE LIBERTAD: \\
RESULTADO: Sean $A_{1}, \dots, A_{q}$ matrices aleatorias independientes, $A_{j} \sim W_{p}(n_{j}, \Sigma), j = 1, \dots, q$. Entonces:
$$A = \sum_{j=1}^{q} A_{j} \sim W_{p}(n, \Sigma), \quad \text{con } n = \sum_{j=1}^{q}n_{j} \quad (2)$$

\item (2) Si $A_{j} \sim W_{p}(n_{j}, \Sigma), j = 1, \dots, q \Rightarrow \upphi_{A_{j}}(\Theta) = |I_{p} - 2i\Theta\Sigma|^{-\frac{n}{2}}$.
$$\upphi_{A}(\Theta) = \upphi_{\sum_{j=1}^{s}n_{j}} = E[e^{it2(\Theta\sum_{j=1}^{k}A_{k})}] = E[e^{it2(\sum_{j=1}^{k}\Theta A_{k})}] =$$ 
$$= E[e^{i\sum_{j=1}^{k}t2(\Theta A_{j})}] = \prod_{j=1}^{k} E[e^{it2(\Theta A_{j})}] = \prod_{j=1}^{k} \upphi_{A_{j}}(\Theta)$$
$$\upphi_{A}(\Theta) = \prod_{j=1}^{k} |I_{p} - 2i\Theta\Sigma|^{-\frac{n_{j}}{2}} = |I_{p} - 2i\Theta\Sigma|^{\frac{1}{2} \sum_{j=1}^{k}n_{j}} = |I_{p} - 2i\Theta\Sigma|^{-\frac{n}{2}}$$

\item[3] REPRODUCTIVIDAD BAJO TRANSFORMACIONES LINEALES `RECTANGULARES': \\
RESULTADO: Sea $A \sim W_{p}(n, \Sigma)$ y sea $M_{k \times p}$ una matriz no aleatoria de rango $k \leq p$. Entonces,
$$MAM' \sim W_{k}(n, M \Sigma M') \quad (3)$$

\item Se prueba también que
$$(MA^{-1}M')^{-1} \sim W_{k}(n-p+k, (N\Sigma^{-1}M')^{-1})$$
(No se verá la demostración, donde se usa el resultado sobre condicionamiento que se enuncia más adelante en la propiedad [5]).

\item (3) Sea $A \sim W_{p}(n, \Sigma) \Rightarrow A \overset{d}{:=} \sum_{\alpha=1}^{n} Z_{\alpha}Z_{\alpha}'$, con $Z_{\alpha} \sim N(0, \Sigma)$ independientes, $\forall \alpha = 1, \dots, n$.

Defino $Y_{\alpha} := MZ_{\alpha}$ que también son independientes $\forall \alpha = 1, \dots, n$ por serlo los $Z_{\alpha}$ y siguen también una distribución normal (por resultado del T1).
$$Y_{\alpha} \sim N_{k}(0, M \Sigma M')$$
$$\sum_{\alpha=1}^{n} Y_{\alpha}Y_{\alpha}' = \sum_{\alpha=1}^{n} (MZ_{\alpha})(MZ_{\alpha})' = \sum_{\alpha=1}^{n} MZ_{\alpha}Z_{\alpha}'M' = M(\sum_{\alpha=1}^{n} Z_{\alpha}Z_{\alpha}')M' = MAM'$$
$\Rightarrow MAM' \sim W_{k}(n, M \Sigma M')$, con $k = \operatorname{rango}(M)$.

\item[4] DISTRIBUCIÓN DE LAS MATRICES DE COVARIANZAS Y CUASI-CONVARIANZAS \\ MUESTRALES: \\
RESULTADO: Sean $S_{N} = \frac{A}{N}$ y $S_{n} = \frac{A}{n}$ (con $n=N-1$) las matrices de covarianzas y cuasi-covarianzas muestrales correspondientes a una muestra aleatoria simple de tamaño $N$ de una distribución $N_{p}(\mu, \Sigma)$. Entonces,
$$\begin{aligned}
S_{N} \sim W_{p}(n, \frac{\Sigma}{N}) \\
S_{n} \sim W_{p}(n, \frac{\Sigma}{n})
\end{aligned} \quad (4)$$
(Este resultado es consecuencia del resultado visto en la propiedad [3]).

\item (4) $A \sim W_{p}(n, \Sigma)$.
$$S_{N} = \frac{1}{N}A$$
$$M = \frac{1}{\sqrt{N}} I \Rightarrow MAM' = \frac{1}{\sqrt{N}} I A \frac{1}{\sqrt{N}} I = \frac{1}{N} A = S_{N} \Rightarrow$$
$$\Rightarrow S_{N} \sim W_{p}(n, M \Sigma M') = W_{p}(n, \frac{1}{N}\Sigma)$$

$$S_{n} = \frac{1}{n}A = \frac{1}{N-1}A$$
$$M = \frac{1}{\sqrt{N-1}} I \Rightarrow MAM' = \frac{1}{\sqrt{N-1}} I A \frac{1}{\sqrt{N-1}} I = \frac{1}{N-1}A = S_{n} \Rightarrow$$
$$\Rightarrow S_{n} \sim W_{p}(n, M \Sigma N') = W_{p}(n, \frac{1}{N-1}\Sigma)$$

\item[5] MARGINALIZACIÓN (DIAGONAL Y POR BLOQUES) Y CONDICIONAMIENTO:
\begin{itemize}
\item RESULTADO 1: Sea $A \sim W_{p}(n, \Sigma)$. Entonces:
$$\frac{a_{ii}}{\sigma_{i}^{2}} \sim \chi_{n}^{2}, \quad i=1,\dots,p \quad (5)$$

\item (5) $M = (0, \dots, 0, \overset{i}{1}, 0, \dots, 0)_{1 \times p} \Rightarrow M \Sigma M' = a_{i} = \sigma_{i}^{2}, i = 1, \dots, p$.
$$MAM' = a_{ii} \Rightarrow a_{ii} \sim W_{1}(n, \sigma_{i}^{2}), i = 1, \dots, p$$
$$\Rightarrow \frac{a_{ii}}{\sigma_{i}^{2}} \sim \chi_{n}^{2}, i = 1, \dots, p$$

\item RESULTADO 2: Sea $A \sim W_{p}(n, \Sigma)$, y sea el particionamiento por cajas:
$$A = \begin{pmatrix}
A_{11} & A_{12} \\
A_{21} & A_{22}
\end{pmatrix}$$
y, correspondientemente,
$$\Sigma = \begin{pmatrix}
\Sigma_{11} & \Sigma_{12} \\
\Sigma_{21} & \Sigma_{22}
\end{pmatrix}$$
con $A_{11}$ de dimensión $q \times q$ y $A_{22}$ de dimensión $(p-q)\times(p-q)$. Entonces:
\begin{itemize}
\item[(i)] $A_{11} \sim W_{q}(n, \Sigma_{11})$ y $A_{22} \sim W_{p-q}(n, \Sigma_{22})$.
\item[(ii)] Si además $\Sigma_{12} = 0$, entonces $A_{11}$ y $A_{22}$ son independientes (6).
\end{itemize}

\item (6) $A \sim W_{p}(n, \Sigma)$
\begin{itemize}
\item[(i)] $M_{1} = \begin{pmatrix} I_{q} \mid 0_{q \times (p-q)} \end{pmatrix}$, $M_{2} = \begin{pmatrix} 0_{(p-q) \times q} \mid I_{p-q} \end{pmatrix}$. \\
Entonces
$$\begin{matrix}
M_{1} A M_{1}' = A_{11}, & M_{2} A M_{2}' = A_{22} \\
M_{1} \Sigma M_{1}' = \Sigma_{11}, & M_{2} \Sigma M_{2}' = \Sigma_{22}
\end{matrix}$$
Aplicamos el resultado 3.
\item[(ii)] $A \overset{d}{=} \sum_{\alpha=1}^{n} Z_{\alpha}Z_{\alpha}'$, con $Z_{\alpha} \sim N_{p}(0, \Sigma)$, con:
$$\Sigma = \begin{pmatrix}
\Sigma_{11} & 0 \\
0 & \Sigma_{22}
\end{pmatrix}$$
Para cada $\alpha$, consideramos el particionamiento $Z_{\alpha} = \begin{pmatrix} Z_{\alpha, [1]} \\ Z_{\alpha, [2]} \end{pmatrix}$ independientes. \\
Puedo escribir:
$$\begin{aligned}
A_{11} = \sum_{\alpha=1}^{n} Z_{\alpha, [1]} Z_{\alpha, [1]}' \\
A_{22} = \sum_{\alpha=1}^{n} Z_{\alpha, [2]} Z_{\alpha, [2]}'
\end{aligned} \text{independientes}$$
\end{itemize}

\item RESULTADO 3: Sea $A \sim W_{p}(n, \Sigma)$, y sea el particionamiento por cajas considerado en el resultado anterior (Resultado 2). Entonces, con
$$\begin{matrix}
A_{11 \cdot 2} = A_{11} - A_{12}A_{22}^{-1}A_{21}, & A_{22 \cdot 1} = A_{22} - A_{21}A_{11}^{-1}A_{12} \\
\Sigma_{11 \cdot 2} = \Sigma_{11} - \Sigma_{12}\Sigma_{22}^{-1}\Sigma_{21}, & \Sigma_{22 \cdot 1} = \Sigma_{22} - \Sigma_{21}\Sigma_{11}^{-1}\Sigma_{12}
\end{matrix}$$
se tiene que
\begin{itemize}
\item[(a)] $A_{11 \cdot 2} \sim W_{q}(n-(p-q), \Sigma_{11 \cdot 2})$ y es independiente de $A_{12}$ y $A_{22}$.
\item[(b)] $A_{22 \cdot 1} \sim W_{p-q}(n-q, \Sigma_{22 \cdot 1})$ y es independiente de $A_{12}$ y $A_{11}$.
\item[(c)] $A_{12} \mid A_{22} \sim N_{q \times (p-q)}(\Sigma_{12}\Sigma_{22}^{-1}A_{22}, \Sigma_{11 \cdot 2} \otimes A_{22})$. \\
donde $A \otimes B = \begin{pmatrix}
a_{11}B & a_{12}B & \dots \\
a_{21}B & a_{22}B & \dots \\
\vdots & \vdots & \ddots
\end{pmatrix}$.
\end{itemize}
\end{itemize}

\item[6] El siguiente resultado será útil en relación con distribución $T^{2}$ de Hotelling: \\
RESULTADO: Sea $A \sim W_{p}(n, \Sigma)$ y sea $Y$ un vector aleatorio $p$-dimensional independiente de $A$ tal que $P[Y=0]=0$. Entonces,
$$\frac{Y'AY}{Y'\Sigma Y} \sim \chi_{n}^{2} \quad (7)$$
siendo independiente de $Y$.

\item (7) Aplicamos el resultado 3, condicionalmente a que $Y = y \neq 0$. Tomamos
$$M = y' \overset{\text{Resultado } 3}{\Longrightarrow} y'Ay \sim W_{1}(n, y' \Sigma y)$$
Tenemos en cuenta que
$$Y'AY \mid Y = y \overset{d}{\equiv} y'(A \mid Y=y)y = y'Ay$$
(por ser $A$ e $Y$ independientes).
$$\frac{y'Ay}{y' \Sigma y} \sim \chi_{n}^{2}$$
no depende de $y$. Por tanto:
$$\frac{Y'AY}{Y' \Sigma Y} \sim \chi_{n}^{2}$$

\item Se prueba también que
$$\frac{Y'\Sigma^{-1}Y}{Y'A^{-1}Y} \sim \chi_{n-p+1}^{2}$$
siendo independiente de $Y$.

\item Una consecuencia interesante, en el contexto de la inferencia, viene dada por el siguiente resultado:

Sea $\{X_{\alpha}: \alpha=1,\dots,N\}$ una muestra aleatoria simple de una distribución $N_{p}(\mu, \Sigma)$. Entonces, con $n = N-1$,
$$n\frac{\bar{X}'S_{n}\bar{X}}{\bar{X}'\Sigma\bar{X}} \sim \chi_{n}^{2} \quad (8)$$
siendo independiente de $\bar{X}$.

\item (8) $\{X_{\alpha}: \alpha=1, \dots, N\}$ m.a.s, $X \sim N_{p}(\mu, \Sigma)$.
$$A = nS_{n} = NS_{N} \sim W_{p}(n, \Sigma)$$
$$\bar{X} \sim N_{p}(\mu, \frac{\Sigma}{N})$$
$A$ y $\bar{X}$ son independientes (teorema de Fisher). Además, $P(\bar{X}=0)=0$. Aplicamos el resultado 6:
$$\frac{\bar{X}'A\bar{X}}{\bar{X}'\Sigma\bar{X}} = n \frac{\bar{X}'S_{n}\bar{X}}{\bar{X}'\Sigma\bar{X}} \sim \chi_{n}^{2}$$
\end{itemize}

\subsubsection{Algunos aspectos complementarios}
Entre otros aspectos, a partir de este punto se podrían estudiar, como complemento:
\begin{itemize}
\item La \emph{distribución de Wishart inversa}, que permite representar de forma directa la distribución de $A^{-1}$ para $A \sim W_{p}(n, \Sigma)$:
$$A^{-1} \sim W_{p}^{-1}(n+p+1, \Sigma^{-1})$$

\item La \emph{distribución de Wishart descentrada}, que permite cubrir el caso en que $A = \sum_{\alpha=1}^{n} Z_{\alpha}Z_{\alpha}'$, con $Z_{\alpha} \sim N_{p}(\mu, \Sigma)$.

\item Formas cuadráticas normales matriciales y su relación con la distribución de Wishart.
\end{itemize}

\subsection{Distribución \texorpdfstring{$T^{2}$}) de Hotelling}
\subsubsection{Distribución \texorpdfstring{$T^{2}$}) de Hotelling}
\begin{itemize}
\item DEFINICIÓN: Sea $X$ un vector aleatorio $p$-dimensional con distribución $N_{p}(\mu, \Sigma) \quad (\Sigma > 0)$. Sea $B$ una matriz aleatoria $(p \times p)$-dimensional con distribución $W_{p}(n, \Sigma), n \geq p$. Supongamos, además, que $X$ y $B$ son independientes. Entonces, se define la distribución $T^{2}$ de Hotelling no centrada, con $p$ y $n$ grados de libertad y parámetro de no centralidad $\delta = \mu'\Sigma\mu$, como la distribución de la variable aleatoria (unidimensional) dada por la forma cuadrática
$$T^{2} = nX'BP{-1}X$$

\item La distribución anterior es conocida, y se relaciona con la distribución $F$ de Snedecor no centrada del siguiente modo: \\
RESULTADO: Bajo las condiciones de la definición anterior, se tiene que
$$\frac{T^{2}}{n} \frac{n-p+1}{p} \sim F_{p, n-p+1}(\delta) \quad (1)$$
con parámetro de no centralidad $\delta = \mu'\Sigma^{-1}\mu$
\begin{itemize}
\item Si $\mu=0$, se tiene entonces una $F_{p, n-p+1}$ centrada.
\item En el caso $p=1$, se tiene que $T^{2} \equiv F_{1, n}(\delta)$, con $\delta = (\frac{\mu}{\sigma})^{2}$.
\end{itemize}

\item (1) $X \sim N_{p}(\mu, \Sigma), \Sigma > 0, B \sim W_{p}(n, \Sigma), n \geq p$ independientes.
$$\frac{T^{2}}{n} \frac{n-p+1}{p} = \frac{nX'B^{-1}X}{n} \frac{n-p+1}{p} = X'B^{-1}X \frac{X'\Sigma^{-1}X}{X'\Sigma^{-1}X} \frac{n-p+1}{p} = \frac{\sfrac{X\Sigma^{-1}X}{p}}{\sfrac{\frac{X'\Sigma^{-1}X}{X'B^{-1}X}}{n-p+1}}$$
Sabemos que, como $X \sim N_{p}(\mu, \Sigma)$ con $\Sigma > 0 \Rightarrow X'\Sigma^{-1}X \sim \chi^{2}_{p}(\delta), \delta = \mu'\Sigma^{-1}\mu$. Además, como $B \sim W_{p}(n, \Sigma)$ y $X$ y $B$ son independientes $\Rightarrow \frac{X'\Sigma^{-1}X}{X'B^{-1}X} \sim \chi_{n-p+1}^{2}$ independiente de $X$, luego independiente de $X'\Sigma^{-1}X$.

Por tanto, $\frac{T^{2}}{n}\frac{n-p+1}{p} \sim F_{p, n-p+1}(\delta)$, con $\delta = \mu'\Sigma^{-1}\mu$.

\item NOTA: $t_{n}^{2} \equiv F_{1, n}$.
\end{itemize}

\subsubsection{Estadístico \texorpdfstring{$T^{2}$}) de Hotelling (inferencia multivariante)}
\begin{itemize}
\item Originalmente, el denominado `estadístico $T^{2}$ de Hotelling' fue introducido por Harold Hotelling (1931) en el contexto de la inferencia estadística a partir del muestreo de una población con DNM (puede verse como una generalización, multivariante y no estandarizada, del estadístico $t$ de Student).

Este uso se explica a continuación, como aplicación de la definición y el resultado anteriores:

\item DEFINICIÓN: Sea $X_{1}, \dots, X_{N}$ una muestra aleatoria simple de una distribución $N_{p}(\mu, \Sigma) \quad (\Sigma > 0)$, y sean $\bar{X}, A, S_{N}$ y $S_{n}$ (con $n=N-1$) los correspondientes estadísticos vector de medias, matriz de dispersiones y matrices de covarianzas y cuasi-covarianzas muestrales. Se define el estadístico $T^{2}$ de Hotelling, basado en la muestra, como
$$T^{2} := n\bar{X}'S_{N}^{-1}\bar{X} = N\bar{X}'S_{n}^{-1}\bar{X} = nN\bar{X}'A^{-1}\bar{X}$$
(Formalmente, se escribiría $T^{2}(X_{1}, \dots, X_{N})$, con $T^{2}: \mathbb{R}^{Np} \to \mathbb{R}_{0}^{+}$ medible, aunque se suele sobreentender de forma implícita el argumento).

\item NOTA: $\bar{X} \sim N_{p}(\mu, \frac{\Sigma}{N}), S_{N} \sim (n, \frac{\Sigma}{N})$.

\item Por el resultado visto al inicio, se tiene el siguiente, donde se caracteriza convenientemente la distribución del estadístico $T^{2}$ en términos de la distribución $F$ de Snedecor no centrada: \\
RESULTADO: Bajo las condiciones anteriores, el estadístico $T^{2}$ se distribuye como sigue:
$$\frac{T^{2}}{n} \frac{n-p+1}{p} \sim F_{p, n-p+1}(\delta)$$
o, equivalentemente,
$$\frac{T^{2}}{N-1} \frac{N-p}{p} \sim F_{p, N-p}(\delta)$$
con parámetro de no centralidad $\delta = N\mu'\Sigma^{-1}\mu$.
\end{itemize}

\subsubsection{Aplicación: Contraste sobre el vector de medias en una DNM}
\begin{itemize}
\item PLANTEAMIENTO: Sea $X_{1}, \dots, X_{N}$ una muestra aleatoria simple extraída de una población con distribución $N_{p}(\mu, \Sigma) \quad (\Sigma > 0)$, siendo $N > p$.

Denotaremos $X = (X_{1}, \dots, X_{N})'$ (matriz correspondiente a la muestra aleatoria) y, correspondientemente, $x = (x_{1}, \dots, x_{N})'$ (matriz de datos muestrales observados).

PROBLEMA DE CONTRASTE: Se consideran las hipótesis:
$$\begin{cases}
H_{0}: \mu = \mu_{0} & \text{(`hipótesis nula')} \\
H_{1}: \mu \neq \mu_{0} & \text{(`hipótesis alternativa')}
\end{cases}$$
donde $\mu_{0} \in \mathbb{R}^{p}$ es un vector fijo dado.

\item DOS ESCENARIOS:
\begin{itemize}
\item $\Sigma$ conocida $\to$ Se resuelve a partir de la distribución $\chi^{2}(\delta)$.
\item $\Sigma$ desconocida $\to$ Se resuelve a partir de la distribución $T^{2}$.
\end{itemize}

\item Caso $\Sigma$ conocida \\
Estadístico de contraste:
$$U := N(\bar{X}-\mu_{0})'\Sigma^{-1}(\bar{X}-\mu_{0}) \sim \chi_{p}^{2}(\delta)$$
con $\delta=N(\mu-\mu_{0})'\Sigma^{-1}(\mu-\mu_{0})$.

Nota:
$$X \sim N_{p}(\mu, \Sigma) \Rightarrow \bar{X} \sim N_{p}(\mu, \frac{\Sigma}{N}) \Rightarrow X-\mu \sim N_{p}(\mu - \mu_{0}, \frac{\Sigma}{N})$$

Función `test': Con $\upphi: \mathbb{R}^{Np} \to \{0, 1\}$
$$\upphi(x) = \begin{cases}
1 & \text{si } u := U(x) > \chi_{p, \alpha}^{2} \\
0 & \text{si } u := U(x) \leq \chi_{p, \alpha}^{2}
\end{cases}$$
para todo $x \in \mathbb{R}^{Np}$, siendo $\chi_{p, \alpha}^{2}$ el valor de una distribución $\chi^{2}$ (centrada) con $p$ grados de libertad que deja a su derecha una masa de probabilidad igual a $\alpha$.

\item Caso $\Sigma$ desconocida: \\
Estadístico de contraste:
$$T^{2} := N(\bar{X} - \mu_{0})' S_{n}^{-1} (\bar{X} - \mu_{0})$$
verificándose que
$$\frac{T^{2}}{n} \frac{n-p+1}{p} \sim F_{p, n-p+1}(\delta)$$
con $\delta = N(\mu-\mu_{0})' \Sigma^{-1} (\mu-\mu_{0})$.

Función `test': Con $\upphi: \mathbb{R}^{Np} \to \{0, 1\}$,
$$\upphi(x) = \begin{cases}
1 & \text{si } u := T^{2}(x) > \frac{np}{n-p+1}F_{p,n-p+1,\alpha} \\
0 & \text{si } u := T^{2}(x) \leq \frac{np}{n-p+1}F_{p,n-p+1,\alpha}
\end{cases}$$
para todo $x \in \mathbb{R}^{Np}$, siendo $F_{p, n-p+1, \alpha}$ el valor de una distribución $F$ (centrada) con ($p, n-p+1$) grados de libertad que deja a su derecha una masa de probabilidad igual a $\alpha$.
\end{itemize}

\subsubsection{Ejemplo}
`Altura' (pulgadas) y `Peso' (libras) para una muestra de $N = 20$ individuos.
\begin{center}
\begin{tabular}{|c|cc||c|cc|}
\hline
& Altura & Peso & & Altura & Peso \\
Individuo & ($X_{1}$) & ($X_{2}$) & Individuo & ($X_{1}$) & ($X_{2}$) \\ \hline
1 & 69 & 153 & 11 & 72 & 140 \\
2 & 74 & 175 & 12 & 79 & 265 \\
3 & 68 & 155 & 13 & 74 & 185 \\
4 & 70 & 135 & 14 & 67 & 112 \\
5 & 72 & 172 & 15 & 66 & 140 \\
6 & 67 & 150 & 16 & 71 & 150 \\
7 & 66 & 115 & 17 & 74 & 165 \\
8 & 70 & 137 & 18 & 75 & 185 \\
9 & 76 & 200 & 19 & 75 & 210 \\
10 & 68 & 130 & 20 & 76 & 220 \\ \hline
\end{tabular}
\end{center}

Estadísticos muestales básicos:
\begin{itemize}
\item Vector de medias muestral:
$$\bar{x} = \begin{pmatrix} 71.45 \\ 164.7 \end{pmatrix}$$
\item Matriz de cuasi-covarianzas muestral ($n = N-1 = 19$(
$$S_{n} = \begin{pmatrix}
14.576 & 128.88 \\
128.88 & 1441.2653
\end{pmatrix}$$
\item Coeficiente de correlación muestral:
$$r_{12} = 0.889$$
\end{itemize}

Se quiere contrastar las hipótesis:
$$\begin{cases}
H_{0}: \mu = \begin{pmatrix} 70 \\ 170 \end{pmatrix} \\
H_{1}: \mu \neq \begin{pmatrix} 70 \\ 170 \end{pmatrix}
\end{cases}$$

Contraste sobre el vector de medias $\mu$, con $\Sigma$ conocida: \\
Supongamos que se conoce que, para la población de referencia, la matriz de covarianzas es
$$\Sigma = \begin{pmatrix}
20 & 100 \\
100 & 1000
\end{pmatrix}$$
\begin{itemize}
\item Valor del estadístico de contraste:
$$u = (20) \begin{pmatrix}
71.45 - 70 \\
164.7 - 170
\end{pmatrix}' \begin{pmatrix}
20 & 100 \\
100 & 1000
\end{pmatrix}^{-1} \begin{pmatrix}
71.45 - 70 \\
164.7 - 170
\end{pmatrix} = 8.4026$$
\item Comparación con el valor teórico bajo $H_{0}$ a un nivel de significación $\alpha = 0.05, \chi^{2}_{2, 0.05} = 5.99$
$$u = 8.4026 > 5.99 = \chi^{2}_{2, 0.05} \to \text{Se rechazaría } H_{0}$$
\item Comparación con el valor teórico bajo $H_{0}$ a un nivel de significación $\alpha = 0.01, \chi^{2}_{2, 0.01} = 9.21$
$$u = 8.4026 < 9.21 = \chi^{2}_{2, 0.01} \to \text{No se rechazaría } H_{0}$$
\end{itemize}

Contraste sobre el vector de medias $\mu$, con $\Sigma$ desconocida:
\begin{itemize}
\item Valor del estadístico de contraste:
$$u = (20) \begin{pmatrix}
71.45 - 70 \\
164.7 - 170
\end{pmatrix}' \begin{pmatrix}
14.576 & 128.88 \\
128.88 & 1441.2653
\end{pmatrix}^{-1} \begin{pmatrix}
71.45 - 70 \\
164.7 - 170
\end{pmatrix} = \dots$$
\item Valor de comparación teórico bajo $H_{0}$ a un nivel de significación $\alpha$:
$$\frac{(19)(2)}{19-2+1}F_{2, 19-2+1, \alpha} = \frac{38}{18}F_{2, 18, \alpha} = \dots$$
\end{itemize}

\end{document}